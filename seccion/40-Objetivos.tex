% !TeX root = ../tfm.tex
%! TEX root = ../tfm.tex
\begin{comment}
Puente entre el estudio y la contribución. Debe contener:
 * objetivo general
 * objetivo específico
 * metodología de trabajo
 
Los objetivos deben ser \textit{SMART}

Specificl (objetivo claro)
Measurable (se pueda medir el éxito o fracaso)
Attainable (viable su conecución con el tiempo y recursos disponibles)
Relevant (que tenga un impacto demostrable)
Time-related (que se pueda realizar en un tiempo determinado)
\end{comment}

\section{Objetivo General}\label{sec:obj:objetivo_general}
%\begin{comment}
%Un poco la descripción a grandes rasgos de qué se espera explicado al público general.
%Ejemplo: Mejorar el servicio de audio gruía de un museo con una guía interactiva por voz valorada positivamente con un 4/5 al menos.
%\end{comment}

Al término de este trabajo se presentará una implementación de un sistema de detección de caídas autónomo sobre un dispositivo vestible del tipo \textit{smartwatch} con una precisión mayor a la disponible en los sistemas vestibles actuales y comparable a la de los sistemas basados en aprendizaje automático.

\section{Objetivos Específicos}\label{sec:obj:objetivos_especificos}

Dividiremos el trabajo a realizar en los siguientes metas con la idea de compartimentar el trabajo y facilitar su posterior evaluación y valoración del cumplimiento de los mismos.

\begin{enumerate}
  \item Identificar los requisitos funcionales que debe cumplir un sistema de detección de caídas.
  \item Diseñar o determinar un algoritmo eficiente para la detección de caídas: Debe tener bajo coste de cómputo sin sacrificar la fiabilidad.
  \item Implementar un modelo basado en el algoritmo anterior.
  \item Comparar y evaluar el modelo implementado frente a otros algoritmos existentes para analizar su eficiencia y viabilidad.
  \item Establecer e implementar un sistema de detección de caídas usando el modelo previamente definido y validado.
  \item Analizar los resultados del sistema y su viabilidad como solución al problema de la detección de caídas.

\end{enumerate}

\begin{comment}
Conjunto de objetivos más específicos alcanzables por separado. suelen ser los diferentes pasos a seguir para conseguir el objetivo general. Han de ser smart, los verbos deberían ser:     • Analizar
Calcular
Clasificar
Comparar
Conocer
Cuantificar
Desarrollar
Describir
Descubrir
Determinar
Establecer
Explorar
Identificar
Indagar
Medir
Sintetizar
Verificar
\end{comment}

\section{Metodología de trabajo}

Se van a abordar muchas y muy variadas tareas, por lo que es fundamental establecer una jerarquía y organización entre estas. Definimos aquí las principales etapas del proceso seguido:

\subsection{Documentación previa}

Una vez definido del objetivo general definido en la sección \ref{sec:obj:objetivo_general} es necesario adquirir un conocimiento del contexto y estado de la tecnología en el área definida. Para ello es necesario recopilar información de diversas fuentes y publicaciones. Validaremos este punto definiendo:
\begin{itemize}
  \item Un modelo y algoritmo que a priori permita realizar la tarea propuesta.
  \item El índice o índices necesarios para alimentar al modelo.
  \item Dados los puntos anteriores, definir unas especificaciones aproximadas (tipo de sensor, posición del mismo, velocidad de lectura, etc) que debe cumplir la plataforma a usar.
\end{itemize}

\subsection{Plataforma}
A partir de los requerimientos impuestos en el punto anterior, analizaremos los dispositivos disponibles y realizaremos la elección de la plataforma sobre la que ejecutaremos la aplicación objeto de este proyecto. Con esta decisión tomada, y una vez adquirido el soporte material, podremos validar que efectivamente cumple los requisitos previos estipulados. Adicionalmente en este punto definiremos el entorno de software para crear el modelo que ha de ser soportado por la plataforma informática usada a la vez que ser capaz de implementar el modelo y/o algoritmo.


\subsection{Bases de datos}
En paralelo a la elección de plataforma y como si de una prolongación de la fase de documentación se tratara, realizaremos un análisis de las bases de datos disponibles que se adecuen al proyecto. Tendremos en cuenta que tanto el tipo de datos, formato y etiquetado se ajuste o pueda ser adaptado a las especificaciones definidas. Así mismo, es esencial que el conjunto escogido permita:
\begin{itemize}
  \item Entrenar un modelo y validar el funcionamiento del algoritmo propuesto de forma experimental.
  \item Generar un modelo utilizable en la implementación de la aplicación.
  \item Comparar la precisión del sistema con otros modelos y algoritmos existentes para su validación
\end{itemize}
En este punto deberemos definir, validar y/o recolectar una serie de propiedades de las señales o series que se usarán como entrada del algoritmo, como por ejemplo:
\begin{itemize}
  \item Estudiar la estacionariedad de la señal.
  \item Duración de los episodios; fijar unos márgenes para el tamaño de ventana para la señal de entrada.
  \item Definir un tamaño orientativo de la señal de salida.
\end{itemize}

\subsection{Implementación de un prototipo del algoritmo}
Una vez hecha la elección de los puntos anteriores pasaremos a la implementación experimental del sistema. Generaremos un modelo y estudiaremos los resultados de variar diversos parámetros:
\begin{itemize}
  \item Fijar las longitudes de las secuencias de entrada y salida, así como la magnitud de la señal de entrada.
  \item Seleccionar los parámetros del modelo analítico.
  \item Definir la topología de red y ajustar los hiperparámetros del modelo recurrente.
  \item Confirmar el funcionamiento del sistema y su ajuste a los requisitos y exigencias previstos.
  \item Comparar los resultados con otros modelos.
\end{itemize}
Una vez todos estos puntos resueltos tendremos las especificaciones y el modelo final que será el punto de partida de la implementación del sistema.

\subsection{Implementación de la aplicación final}
Partiendo del modelo anterior, crearemos una aplicación que capture y procese los datos necesarios para alimentarlo, lo ejecute y genere una salida acorde. Es indispensable conseguir que además de ejecutar satisfactoriamente el algoritmo, los resultados se obtengan dentro de un margen de latencias aceptables (esperamos conseguir resultados en menos de un segundo desde el punto de máximo impacto de la caída).

\subsection{Validación en uso real}
Dada la naturaleza del evento a estudiar es complejo realizar una validación experimental con muestras suficientes. Sin embargo esta etapa debe proveer el punto de vista del usuario final sobre el sistema así, su experiencia de uso, adecuación de la solución al problema y posibles ejes de mejora. Los datos recogidos por la aplicación deben ser analizados para detectar puntos de mejora a proponer para posteriores estudios o evoluciones.

% \end{document}
