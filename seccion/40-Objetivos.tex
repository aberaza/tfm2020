\documentclass[../tfm.tex]{subfiles}

\begin{document}
\info{Puente entre el estudio y la contribución.}

Debe contener:
\begin{enumerate}
  \item objetivo general
  \item objetivo específico
  \item metodología de trabajo
\end{enumerate}

Los objetivos deben ser \textit{SMART}
\begin{itemize}
  \item Specificl (objetivo claro)
  \item Measurable (se pueda medir el éxito o fracaso)
  \item Attainable (viable su conecución con el tiempo y recursos disponibles)
  \item Relevant (que tenga un impacto demostrable)
  \item Time-related (que se pueda realizar en un tiempo determinado)
\end{itemize}


\section{Objetivo General}

Un poco la descripción a grandes rasgos de qué se espera explicado al público general.

Ejemplo: Mejorar el servicio de audio gruía de un museo con una guía interactiva por voz valorada positivamente con un 4/5 al menos.

Implementar un sistema de detección de caídas que se ejecute en un smartwatch con una tasa de detección comparable a los sitemas basados en "thresholds" del acelerómetro y mucho menor ratio de falsos positivos.

\section{Objetivos Específicos}
\begin{itemize}
  \item Desarrollar un conjunto de datos para el experimento
  \item Estudiar el comportamiento y características de las señales temporales del acelerómetro.
  \item Identificar y establecer una magnitud derivada de la lectura del acelerómetro que permita su posterior trabajo
  \item Desarrollar un sistema predictor basado en los puntos anteriores
  \item cuantificar el grado de satisfacción con el sistema obtenido y comparar con los sistemas existentes
\end{itemize}


Conjunto de objetivos más específicos alcanzables por separado. suelen sor los diferentes pasos a seguir para conseguir el objetivo general. Han de ser smart, los verbos deberian ser:     • Analizar
Calcular
Clasificar
Comparar
Conocer
Cuantificar
Desarrollar
Describir
Descubrir
Determinar
Establecer
Explorar
Identificar
Indagar
Medir
Sintetizar
Verificar

\section{Metodología de trabajo}

Debe describir los pasos que se van a dar, el por qué de cada paso, instrumentos a utuilizar y cómo se van a analizar los resultados.

\begin{enumerate}
  \item Generar un nuevo dataset \footnote{Se implementa una app para ello, y se realiza un procesado de los datos obtenidos}
  \begin{enumerate}
    \item Los existentes se basan en detección de actividades
    \item Los pocos basados en detección de caidas se restringen a un subconjunt de caidas muy específico y muchas veces con posiciones del elemento de detección diferente de la usada en el estudio
  \end{enumerate}

  2- Estudiar el dataset obtenido
  a- comprender mejor el comportamiento de las señales capturadas
  b- buscar parámetros importantes para llevar a cabo la detección

  \item Implementar un sistema basado en LSTM y detección de anomalías usando errores de predicción. Mecanismo atencional basado en potencia de señal de entrada?
  \begin{itemize}
    \item Implementar una red LSTM FORWARD, y/o una FORWARD-BACKWARD, Asi como un sistema basado en threshold (usar GRU con RELU para mejorar la eficiencia de cómputo)
    \item Una vez entrenados los modelos con información de actividad ordinaria:
    \begin{itemize}
      \item Implementar una app que o bien en tiempo real o cuando detecte un cierto nivel de actividad lance los modelos de predicción
      \item Usar un sistema que comparando la señal real y la predicha decida si es una "anomalía" o no, lo más básico un RMS del error de predicción.
    \end{itemize}
  \end{itemize}

  \item Comparar los resultados con el estado del arte
  \begin{itemize}
    \item Entrenar el sistema usando al menos un dataset que incluya caídas.
    \item Compararar con al menos métodos de cota fall index y modulo aceleración.
    \item Comparar con estudios previos que usen el mismo dataset.
    \item Analizar los resultados con el dataset propio.
  \end{itemize}

  \item Prueba en uso real. (Si da tiempo)
  \begin{itemize}
    \item Entrenar el sistema con dataset creado.
    \item Usar en interno
    \item Probar en caídas simuladas
    \item Si es posible conseguir ejemplos de caídas reales (app debe enviar todos los eventos registrados como caídas e indicar si han sido confirmados, rechazados o ignorados)
  \end{itemize}


\end{enumerate}


\end{document}
