\documentclass[../tfm.tex]{subfiles}

\begin{document}
\info{Puente entre el estudio y la contribución.}
\warn{
Debe contener:
 objetivo general
 objetivo específico
 metodología de trabajo
 }

 \info{
   Los objetivos deben ser \textit{SMART}

     Specificl (objetivo claro)
     Measurable (se pueda medir el éxito o fracaso)
     Attainable (viable su conecución con el tiempo y recursos disponibles)
     Relevant (que tenga un impacto demostrable)
     Time-related (que se pueda realizar en un tiempo determinado)
 }


\section{Objetivo General}
\info{
Un poco la descripción a grandes rasgos de qué se espera explicado al público general.

Ejemplo: Mejorar el servicio de audio gruía de un museo con una guía interactiva por voz valorada positivamente con un 4/5 al menos.
}

Este trabajo presenta una implementación de un sistema de detección de caídas autónomo sobre un dispositivo llevable del tipo smartwatch con una precisión mayor a la disponible en los sistemas llevables actuales comparable a la de los sitemas basados en aprendizaje automático.

\section{Objetivos Específicos}

Dividiremos el trabajo a realizar en los siguientes pasos o etapas:
\begin{itemize}
  \item Desarrollar un conjunto de datos para el entrenamiento y validación del sistema final.
  \item Estudiar el comportamiento y características de las señales temporales del acelerómetro a partir del conjunto de datos previamente capturado para parametrizar el sistema (tamaño de la ventana de observación y )
  \item Identificar y establecer un índice o métrica derivada de la lectura del acelerómetro que permita su posterior uso para el reconocimiento de caídas.
  \item Desarrollar un modelo de predicción de la evolución temporal del índice elegido usando redes neuronales recurrentes y evaluar su rendimiento.
  \item Implementar un modelo de clasificación de la actividad humana usando el índice basado en la aceleración y el modelo de aprendizaje automático de los puntos anteriores.
  \item Comparar los resultados del modelo implementado frente a los modelos existentes. Analizar y cuantificar la mejoría aportada sobre los métodos analíticos.
  \item Analizar la viabilidad e idoneidad de la solución propuesta y posibles mejoras
\end{itemize}

\todohide{
Conjunto de objetivos más específicos alcanzables por separado. suelen sor los diferentes pasos a seguir para conseguir el objetivo general. Han de ser smart, los verbos deberian ser:     • Analizar
Calcular
Clasificar
Comparar
Conocer
Cuantificar
Desarrollar
Describir
Descubrir
Determinar
Establecer
Explorar
Identificar
Indagar
Medir
Sintetizar
Verificar}

\section{Metodología de trabajo}

\todohide{
Debe describir los pasos que se van a dar, el por qué de cada paso, instrumentos a utuilizar y cómo se van a analizar los resultados.
}

\begin{enumerate}
  \item Generar un nuevo dataset \footnote{Se implementa una app para ello, y se realiza un procesado de los datos obtenidos}
  \begin{enumerate}
    \item Los existentes se basan en detección de actividades
    \item Los pocos basados en detección de caidas se restringen a un subconjunt de caidas muy específico y muchas veces con posiciones del elemento de detección diferente de la usada en el estudio
  \end{enumerate}

  2- Estudiar el dataset obtenido
  a- comprender mejor el comportamiento de las señales capturadas
  b- buscar parámetros importantes para llevar a cabo la detección

  \item Implementar un sistema basado en LSTM y detección de anomalías usando errores de predicción. Mecanismo atencional basado en potencia de señal de entrada?
  \begin{itemize}
    \item Implementar una red LSTM FORWARD, y/o una FORWARD-BACKWARD, Asi como un sistema basado en threshold (usar GRU con RELU para mejorar la eficiencia de cómputo)
    \item Una vez entrenados los modelos con información de actividad ordinaria:
    \begin{itemize}
      \item Implementar una app que o bien en tiempo real o cuando detecte un cierto nivel de actividad lance los modelos de predicción
      \item Usar un sistema que comparando la señal real y la predicha decida si es una "anomalía" o no, lo más básico un RMS del error de predicción.
    \end{itemize}
  \end{itemize}

  \item Comparar los resultados con el estado del arte
  \begin{itemize}
    \item Entrenar el sistema usando al menos un dataset que incluya caídas.
    \item Compararar con al menos métodos de cota fall index y modulo aceleración.
    \item Comparar con estudios previos que usen el mismo dataset.
    \item Analizar los resultados con el dataset propio.
  \end{itemize}

  \item Prueba en uso real. (Si da tiempo)
  \begin{itemize}
    \item Entrenar el sistema con dataset creado.
    \item Usar en interno
    \item Probar en caídas simuladas
    \item Si es posible conseguir ejemplos de caídas reales (app debe enviar todos los eventos registrados como caídas e indicar si han sido confirmados, rechazados o ignorados)
  \end{itemize}


\end{enumerate}


\end{document}
