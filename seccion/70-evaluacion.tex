\documentclass[../tfm.tex]{subfiles}
\begin{document}

\info{al menos una mínima evaluación de usabilidad de la herramienta y su aplicabilidad para resolver el problema resuelto.}

\section{Evaluación del modelo}
textwidth in pt: \the\textwidth


Un sistema de detección de caídas es en el fondo un clasificador con una única clase. Todos los resultados pueden por tanto agruparse en los que pertenecen o no a dicho conjunto. Con el fin de evaluar y comparar los resultados obtenidos usaremos dos métricas estadísticas:
\begin{itemize}
  \item Sensitividad (Capacidad de identificar las caídas), también conocida como \textit{recall}.
  \[
    Sensitividad = \frac{TP}{TP+FN}
  \]
  \item Especificidad o Selectividad (Capacidad de discernir únicamente las caídas).
  \[
    Especificidad = \frac{TN}{TP+FP}
  \]
\end{itemize}

Estas métricas son usadas habitualmente en otros trabajos similares\cite{Noury2007,Chen2005, Bourke2006}
\info{añadir el resto de citas}.



\subsection{Modelos basados en Machine Learning}
\warn{de los trabajos Anita2020, Lim2014 extraer tabla de resultados con sisfall y comparar}

\subsection{Mixto Bourke + GRU}
\todo{Validar la mejora del acercamiento en cuanto a especificidad respecto a Bourke simple. Comparar con los resultados de la tabla anterior}


\end{document}
