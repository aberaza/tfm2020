\documentclass[../tfm.tex]{subfiles}
\begin{document}

\info{al menos una mínima evaluación de usabilidad de la herramienta y su aplicabilidad para resolver el problema resuelto.}

\section{Evaluación del modelo}
textwidth in pt: \the\textwidth


Un sistema de detección de caídas es en el fondo un clasificador con una única clase. Todos los resultados pueden por tanto agruparse en los que pertenecen o no a dicho conjunto. Con el fin de evaluar y comparar los resultados obtenidos usaremos dos métricas estadísticas:
\begin{itemize}
  \item Sensitividad (Capacidad de identificar las caídas), también conocida como \textit{recall}.
  \[
    Sensitividad = \frac{TP}{TP+FN}
  \]
  \item Especificidad o Selectividad (Capacidad de discernir únicamente las caídas).
  \[
    Especificidad = \frac{TN}{TP+FP}
  \]
\end{itemize}

Estas métricas son usadas habitualmente en otros trabajos similares\cite{Noury2007,Chen2005, Bourke2006}\todo{añadir el resto de citas}.

\subsection{Bourke}

Los modelos de Bourke \cite{Bourke2006} se basan en la detección de la estructura típica del vector suma de la aceleración de una caída. Esta estructura está compuesta por un valle seguido de un pico. El valle está asociado al inicio de la caída, momento en el que se contrarresta la aceleración constante de 1G en sentido vertical al que están expuestos los cuerpos con la aceleración en el mismo sentido del cuerpo en caída. El posterior pico es debido a las aceleraciones durante el impacto con el suelo. Así pues Bourke propone establecer dos cotas, una para detectar el valle y otra el pico. Con estas cotas definidas se pueden extraer tres modelos. Uno que observe los picos, otro los valles y un tercero que obseerve valles y picos.
\figura{BourkeLowThresholdsHistogram}{fig:bourke_low_hist}{Histograma valores \textit{valle} modelo Bourke}
\figura{BourkeUpperThresholdsHistogram}{fig:bourke_upper_hist}{Histograma valores \textit{pico} modelo Bourke}
Para establecer las cotas, Bourke propone analizar un corpus de caídas y establecer los niveles de tal forma que el 100\% de las caídas entren dentro del espacio, consiguiendo una sensibilidad del mismo valor por definición. Este modelo tiene la desventaja de tener una especificidad muy baja. Los resultados con el dataset SisFall nos arrojan un valor para las cotas de 0,971G para el valle y 4,164G para el pico. Con estos valores se consigue una especificidad de tan solo $0,9\%$. Si observamos la distribución de los valores de valle de las caídas y resto de actividades en la figura \ref{fig:bourke_low_hist}, se aprecia el alto grado de solapamiento y como pràcticamente todos los supuestos del dataset entran dentro del conjunto seguregado por el modelo. Por contra, el análisis de los valores de pico (figura \ref{fig:bourke_upper_hist}) arroja una dsitribucion que permite obtener un discriminador de mejor calidad.

Efectivamente en la figura \ref{fig:bourke_cfmatrix} se aprecia como usando únicamente la cota superior o del pico el modelo de bourke alcanza una especificidad del 42,84\% que aumenta hasta el 64\% si aceptamos una degradación de la sensibilidad hasta el 87,8\%.


\figura{BourkeCONF_Matrix}{fig:bourke_cfmatrix}{Matrices de confusión para modelos Bourke}

\subsection{Modelos basados en Machine Learning}
\warn{de los trabajos Anita2020, Lim2014 extraer tabla de resultados con sisfall y comparar}

\subsection{Mixto Bourke + GRU}
\todo{Validar la mejora del acercamiento en cuanto a especificidad respecto a Bourke simple. Comparar con los resultados de la tabla anterior}


\end{document}
