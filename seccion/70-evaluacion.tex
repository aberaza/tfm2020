% !TeX root = ../tfm.tex
%! TEX root = ../tfm.tex


Un sistema de detección de caídas es en el fondo un clasificador con una única clase. Todos los resultados pueden por tanto agruparse en los que pertenecen o no a dicho conjunto. Con el fin de evaluar y comparar los resultados obtenidos usaremos dos métricas estadísticas:
\begin{itemize}
  \item Sensitividad (Capacidad de identificar las caídas), también conocida como \textit{recall}.
  \[
    Sensitividad = \frac{TP}{TP+FN}
  \]
  \item Especificidad o Selectividad (Capacidad de discernir únicamente las caídas).
  \[
    Especificidad = \frac{TN}{TN+FP}
  \]
\end{itemize}

Estas métricas son usadas habitualmente en otros trabajos similares\cite{Noury2007,Chen2005, Bourke2006} y permite usar los mismos indicadores para analizar los resultados de diferentes modelos.

\section{Evaluación del algoritmo obre SisFall}\label{sec:eval:hibrido}

De la misma forma que analizamos el comportamiento del algoritmo de Bourke en la sección \ref{sub:imp:model:analitico}analizamos el comportamiento del algoritmo de modelo hibrido de iFell. De nuevo nos encontramos con que debemos encontrar y definir una nivel de error del modelo codificador/decodificador para identificar si se trata de una caída. Tomando de nuevo el conjunto de entrenamiento, esta vez sin aislar las caídas, calculamos la distribución del error RMSE para ambos conjuntos, caídas y resto de actividades como puede observarse en la figura \todo{inssertar histograma modelo híbrdio}. Numéricamente observamos en la tabla \ref{}

\tabla[0.4\linewidth]{tab:eval:hibrid:percentiles}{Percentiles de caídas y actvidades para una misma cota de error RMSE}{lcc}{
  RMSE & Caídas (\%) & Actividades (\%) \\
  0,12 & 0          & 1,9   \\
  0,29 & 1          & 28,4  \\
  0,45 & 3          & 45,8  \\
  0,53 & 5          & 51,8  \\
  0,63 & 10         & 57,8  \\
}{2}

Comparando los resultados con los del modelo de Bourke, se observa un primer fenómeno de compresión del espacio de resultados: mientras que el caso de Bourke la distribución de los valores de pico cubría un rango muy amplio, en el caso del error RMSE el rango obtenido más reducido. Si analizamos los datos de la tabla \ref{tab:eval:hibrid:percentiles} 




\section{Evaluación del algoritmo en uso real}\label{sec:eval:real}

\info{añadir el resto de citas}.



\subsection{Modelos basados en Machine Learning}
\info{De Anita 2020 sabemos que
Putra2017: An Event-triggered machine learning approach for accelerometer based fall detection (sistema híbrido: Eventos + ML)

Hussain2019 "Activity Aware falldetection and recognition based on wearable sensors" IEEE sensors 19 vol 12, solo suscripción :(

}

En la tabla \ref{tab:MLResults} Destacan los buenos resultados obtenidos por los modelos que usan técnicas de aprendizaje automático. Si bien en lo referente a la latencia, \cite{Liu2020} obtiene tiempos que superan el segundo en la mayoría de modelos, llegando incluso a los 8,87s obtenidos con un clasificador de Bayes. Tanto en los trabajos de \cite{Liu2020} como \cite{Musci2020} y \cite{Torti2018} se subraya el hecho de la no uniformidad de los resultados. Los mejores resultados se obtienen con población joven mientras que con población adulta las métricas pueden perder hasta 5 puntos, en parte debido a la falta de datos de entrenamiento.

\tablas{tab:MLResults}{Resultados de sistemas basados en ML}{l|cc|c|c|ccc}{
    & \multicolumn{2}{c}{\textbf{iFell}}  & Musci2020 & Torti2018   & Liu2020  & Liu2020  & Liu2020 \\
                  & S100  & S99  & RNN(LSTM) & RNN(LSTM)   & FD-DNN   & LSTM     & CNN     \\ \midrule
Sensitividad (\%) & 100   & 99,2 &           & 98,73       & 94,09     & 81,47   & 87,50   \\
Especificidad (\%)& 34,4  & 64,7 &           & 97,93       & 99,94     & 99,57   & 99,88   \\
Accuracy (\%)     &       &      &           & 98,33       & 99,17     & 96,88   & 98,13   \\
Tiempo(s)         & 0,065 & 0,06 &           &             &           &         &         \\
}{3}

\figura{CompositeFallNormalRMSHistogram_t-25}{fig:GRU_predictionRMS_Histogram}{Histograma de los errores de predicción del modelo GRU}

En la tabla\ref{tab:analiticResults} se muestran los resultados de los modelos basados en métodos analíticos. Su importancia para este estudio proviene del hecho de que son los métodos más extendidos en los sistemas disponibles hoy en día y establecen por tanto el nivel a superar. Se muestra también los resultados obtenidos para el clasificador bourke utilizado a modo comparativo y de validación del resultado obtenido.

\begin{comment}
\tablas{tab:analiticResults}{Resultados de sistemas basados en métodos analíticos}{l|c|c|c|c}{
              & \emph{iFell} & \emph{iFell} & \emph{SisFALL} & \emph{SisFALL} \\
              & Bourke    & Hibrido & Cotas(SumVect)  & Cotas(SV) 100\%Sens \\ \midrule
Sensitividad (\%) & 99,4  &    91,13   & 94,28 & 100  \\
Especificidad (\%) & 29,7 &   42,88    & 96,13 & 32,9 \\
Accuracy (\%) & 64,55 &     67  & 95,21 & 66,43 \\
Tiempo (s)    & 0     & 2,5     & 0       & \\
}{3}
\warn{Falta extraer Lim2014 con sisfall y comparar}
\end{comment}
\subsection{Mixto Bourke + GRU}


\todo{Validar la mejora del acercamiento en cuanto a especificidad respecto a Bourke simple. Comparar con los resultados de la tabla anterior}


% ########################################

El desafío de consolidar en un sistema portable una aplicación autónoma de detección de caídas debe hacer frente a una serie de limitaciones.

\section{Desafíos}
\todo{esto son requisitos, movel al punto anterior} El objetivo de todo modelo de detección de eventos es lograr un sistema que consiga capturar la totalidad de las realizaciones del mismo con el menos número posible de falsos positivos. En otras palabras, buscamos un sistema con una especificidad y sensibilidad de 100\%\cite{Noury2007}. \todo{Añadir referencias a papers, comentarios sobre sensibilidad y especificidad}.

\subsection{Usabilidad}\todo{de nuevo un requisito, al apartado anterior}
El primer reto de toda aplicación es conseguir una experiencia de usuario adaptada al cliente final. De nada sirve lograr implementar una plataforma que cumpla perfectamente con todos los requisitos y objetivos funcionales si el producto resultante se utiliza.

\subsubsection{Público objetivo}
Como se ha mencionado en la introducción del trabajo, los daños relacionados con las caídas son una de las principales causas de mortaldad entre las personas mayores de 65 años \todo{cita requerida}. Es propio de este grupo de población la desafección por la tecnología y la carencia o desinterés por su uso. Esta condición ha de ser tenida en cuenta para el desarrollo de cualquier producto.

Las personas de edad avanzada suelen padecer así mismo de otras condiciones que pueden limitar su grado de movilidad, atención o memoria que impidan o reduzcan la posibilidad de adaptarse o incorporar nuevas rutinas. Los problemas motores y de percepción reducen notablemente la capacidad de mostrar información así como de interactuar con el usuario cuando se necesite una acción por su parte.

Se entiende por tanto que si se ha de realizar un producto para esta población, es requisito que sea lo menos obtrusivo posible \todo{de nuevo un requisito, al punto anterior}, siendo recomendable incorporar la funcionalidad a un objeto de uso cotidiano para evitar la modificación de rutinas o la reticencia a incorporar nuevos procesos o elementos en su vida diaria. La interfaz de usuario debe ser mínima, usando un lenguaje visual que resulte familiar alejado de los estándares de las aplicaciones modernas. Así mismo, reducir o eliminar los procesos de configuración y manipulación, con un sistema que funcione al salir de la caja \todo{mala traducción de \textit{out of the box}}.

\subsubsection{Localización}

Una de las decisiones con mayor impacto sobre la funcionalidad del prototipo es la elección de la posición del dispositivo de captura ya que influencia en gran medida a la capacidad de detección de caídas \cite{Kangas2008}. Diversos estudios muestran que el mejor lugar para posicionar un sistema de medición de la aceleración para detectar caídas es la cintura, seguida de la cabeza siendo posible también usar un medidor en la muñeca\cite{Chen2005, Kangas2008, Noury2007}. Si bien estos resultados se basan en el análisis de métodos analíticos basados en cotas, se desprende de ellos la actitud o posición del cuerpo es un buen indicador para la predicción de actividades, razón por la que realizar la captura en muñecas o tobillos, las extremidades más alejadas del tronco, sufren de mayores penalizaciones para conseguir buenas estimaciones.

Al optarse por un reloj o \textit{pulsera de actividad} como plataforma para la implementación las opciones para posicionar la unidad de medida quedan reducida a una: la muñeca.


\section{Plataforma}
\todo{justificar cada una de estas decisiones}
Para el servidor optamos por una arquitectura de microservicios usando la plataforma de AWS Lambda con almacenamiento en S3

Para el dispositivo móvil usamos un reloj inteligente Fossil Sport con sistema operativo WearOS y por tanto compatible con el ecosistema Android.

El la generación, entrenamiento, análisis y evaluación de modelos se realiza usando Keras/Tensorflow corriendo en la plataforma Google Colab.

\section{Arquitectura}\label{desc_archi}
\warn{según la profe punto muy interesante, a pulir }

\subsection{Arquitectura del sistema}

\warn{hablando de AWS: (ver siguiente páraffo)}
Resumiendo la estructura del sistema, \textit{AWS S3} (\url{https://aws.amazon.com/es/s3/?c=ser&sec=srv}) es el sistema de almacenamiento de datos en la red. Un disco duro en la nube, escalable en capacidad y fácilmente accesible para poder recuperar los datos. \textit{AWS Lambda} (\url{https://aws.amazon.com/es/lambda/?c=ser&sec=srv}) permite implementar funciones de código y definir una serie de eventos para lanzar su ejecución, al formar parte del ecosistema AWS es fácil conectar estas funciones con el sistema S3 y API Gateway. Finalmente \textit{AWS API Gateway} permite definir unos puntos de entrada, o URLs que conformarán la API del servicio. Cuando una de estas direcciones URL es invocada, inmediatamente se ejecuta la 

