% !TeX root = ../tfm.tex
%\documentclass[../tfm.tex]{subfiles}
%\begin{document}

\begin{comment}
Resumen esquemático de cada una de las partes del trabajo. Leer esta sección ha de dar una idea clara de lo que se pretendía y las concluseiones a las que se han llegado y del proceso seguido. Es uno de los capítulos mas importantes
\end{comment}

\section{Motivación}

\begin{comment}
Problema a tratar, posiles causas, relevancia del problema
\end{comment}

En un momento en el que la población mundial acelera su crecimiento, la mayoría de las sociedades occidentales adolecen de un fenómeno contradictorio: su población se envejece rápidamente. Esta tendencia es consecuencia de la contínua tendencia descendente de la tasa de fertilidad pero sobretodo por el incremento en la esperanza de vida de la población. No obstante este aumento en la longevidad no implica que la población sea más saludable. Más bien al contrario.

En su página web, la Organización Mundial de la Salud recoge que los mayores de 65 años son los más propensos a sufrir caídas, y que se producen aproximadamente 37 millones de caídas que requieren atención médica y recomienda priorizar la investigación relacionada con las caídas\cite{FactsFalls2018}. En este contexto es normal que proliferen estudios y soluciones para mejorar la atención de estos colectivos de riesgo.

Esta tendencia no es nueva, el estudio e implementación de sistemas para la detección de caídas lleva décadas en desarrollo\cite{fallindex00, Chen2005, Noury2007, Bourke2006}. En la actualidad se comercializan multitud de dispositivos para la detección de caídas.

Desde sistemas de propósito específico como las alarmas de detección de caídas en el baño basadas en una correa atada a la muñeca, a sistemas multi-propósito basados en dispositivos de captura y otro de procesado por separado, pasando por complejos métodos de captura y procesado de imágenes. Todos estos sistemas adolecen o bien de restricciones geográficas (su funcionamiento queda limitado a un entorno limitado por el alcance de las cámaras o cobertura del enlace radio con la base de procesado) o bien resultan poco prácticos e incluso invasivos para un grupo de usuarios que no está acostumbrado a lidiar con la tecnología.

\section{Planteamiento del trabajo}

\begin{comment}
¿cómo se puede resolver el problema qué se propone descripción de objetivos en términos generales?
\end{comment}

La rápida evolución tecnológica permite tener a nuestra disposición cada vez más dispositivos \textit{vestibles} (por \textit{wearables} del inglés, dispositivos electrónicos que se incorporan en alguna parte del cuerpo para realizar una función) con sensores conectados a internet. Estos dispositivos mejoran generación tras generación gracias a la creciente capacidad de cálculo de los microprocesadores, la miniaturización de los sensores y la reducción de costes, lo que ha permitido generalizar y extender y popularizar su uso al grueso de la sociedad. Estos avances se aprovechan ya parcialmente en algunas soluciones al problema de la detección de caídas. Se utilizan los datos capturados por acelerómetro triaxiales y/o sistemas inerciales de una pulsera de actividad o reloj inteligente ya sea para realizar un procesado analítico de los parámetros medidos y realizar la detección de una caída \cite{fallindex00, Chen2005,Bourke2006,Kangas2008,Bagala2012} (por ejemplo buscar una aceleración fuerte y abrupta) o bien para enviar este flujo de datos a un segundo sistema\cite{Luque2014,Vilarinho2015}, para realizar la detección mediante modelos complejos con grandes requisitos computacionales\cite{Cho2014, Aziz2017b,Putra2017}. Ambas soluciones tienen sus pros y contras. La primera tiene a su favor la alta disponibilidad al aunar captura y tratamiento en la misma unidad, pero falla al usar algoritmos poco precisos, al contrario que la segunda opción que se aprovecha de la mayor potencia de un segundo centro de cómputo para mejorar la detección a costa de una mayor complejidad en el sistema que impacta negativamente en su usabilidad y disponibilidad. Hay que tener en cuenta que el público objetivo de esta tecnología es un segmento de población con escasos conocimientos de nuevas tecnologías y poco habituado a su uso diario.

Este trabajo propone una solución basada en aprovechar la mejora en rendimiento de los procesadores de sistemas llevables como pueden ser las pulseras de actividad o los relojes inteligentes para realizar la captura, tratamiento y detección de caídas en la misma unidad. La diferencia con los sistemas ya existentes, es que el algoritmo usado se base en los últimos avances en tecnologías de predicción temporal con inteligencia artificial, redes recurrentes GRU y un mecanismos de atención o eventos para reducir los requisitos de capacidad de cómputo. La principal novedad de este acercamiento es la arquitectura codificador-decodificador de la red neuronal para conseguir una una secuencia de predicción en un único paso y a su vez, gracias al proceso de reducción y expansión de la dimensionalidad realizada por las redes de codificación y decodificación y mediante un comparador de la predicción con la señal real para detectar las caídas como anomalías basándonos en la baja capacidad de generalizaciónde las redes neuronales a los casos para los que no ha sido entrenada. Este sistema permite simplificar el clasificador a un único caso: \textit{actividad} (o anomalía)

Estas tres mejoras (detección en dos etapas, predicción de varios pasos a la vez y simplificación del clasificador) permiten ejecutar el sistema con una latencia baja en un sistema discreto, no invasivo o molesto para el susuario final, como es llevar puesto un reloj de pulsera. El objetivo final es implementar una solución para la detección de caídas con gran precisión, mejor que la de los sistemas analíticos y próxima a la del estado del arte de los sistemas basados en aprendizaje automático que se ejecute exclusivamente en un dispositivo llevable.

\section{Estructura de la memoria}
\todo[disable]{qué hay en cada uno de los subsiguientes capítulos}
\todo{Desarrollar}

El siguiente trabajo se estructura con una revisión del estado del arte y literatura relacionada en el capítulo \ref{chap:stateofart}, seguido de la definición de objetivos y de requisitos previos en los capítulos \ref{chap:objetivos} y \ref{chap:requisitos}. Posteriormente nos adentraremos en el desarrollo e implementación de la solución en el capítulo \ref{chap:descripcion} y en el capítulo \ref{chap:eval} evaluaremos el sistema y sus resultados contra soluciones existentes, antes de exponer las conclusiones (capítulo \ref{chap:conclusiones}) y trabajo futuro. Al final en los apéndices detallaremos otros aspectos como el conjunto de datos usado (Ap.\ref{app:dataset}) y la plataforma usada (Apéndice.\ref{app:plataforma}).

/todo{RECOMENDACIÓN: ponerlo en su sección (capítulo 2), aunque es únicamente una sugerencia. Podemos quizás dejarlo y poner algo similar en cada capítulo.} Entre la literatura previa y de referencia a este trabajo introduciremos los sistemas de detección de actividad humana en \ref{sect:sa_har}, los diferentes modelos usados \ref{sa_modelos_analiticos}, \ref{sa_modelos_ml} antes de introducir los modelos híbridos en \ref{sa_modelos_hybridos}. Intruduciremos también las redes neuronales recurrentes como herramienta de procesado de series temporales \ref{sa_rnn} y la problemática de la optimización de modelos \ref{sa_optimizacion} tanto a nivel computacional como de consumo de recursos y memoria.

/todo{RECOMENDACIÓN: ponerlo en su sección (capítulo 4), aunque es únicamente una sugerencia. Podemos quizás dejarlo y poner algo similar en cada capítulo.} Estos conceptos se retoman posteriormente para definir los prerrequisitos: plataforma de desarrollo \ref{req_hardware} como herramientas de software \ref{req_tflite} y conjuntos de datos adecuados para entrenamiento y validación de resultados \ref{req_base de datos}. Trataremos la situación actual y soluciones disponibles así como las limitaciones y sus posibles soluciones. Trataremos de nuevo el problema de los diferentes tipos de modelos de detección de caídas \ref{req_modelos} relacionando sus capacidades con las limitaciones tanto de los conjuntos de datos como del soporte físico y computacional usado y su adecuación a la solución propuesta.

/todo{RECOMENDACIÓN: ponerlo en su sección (capítulo 5), aunque es únicamente una sugerencia. Podemos quizás dejarlo y poner algo similar en cada capítulo.}En las secciones posteriores detallaremos el algoritmo y modelo usado para la detección de caídas \ref{desc_modelo}, estructura \ref{desc_archi}, implementación \ref{desc_impl} e interfaz\todo{añadir referencia} de la aplicación y las optimizaciones implementadas \ref{desc_optim}, para a la postre evaluar tanto la viabilidad del modelo usado respecto a otras implementaciones \ref{eval_modelo}, como la usabilidad de la aplicación y su adecuación al problema a tratar\todo{añadir referencia}.

Finalmente tras presentar los resultados obtenidos analizaremos posibles mejoras futuras al sistema así como alternativas que no ha dado tiempo a explorar en este trabajo

%\end{document}
