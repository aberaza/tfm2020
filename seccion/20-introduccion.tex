% !TeX root = ../tfm.tex
%\documentclass[../tfm.tex]{subfiles}
%\begin{document}
\begin{comment}
Resumen esquemático de cada una de las partes del trabajo. Leer esta sección ha de dar una idea clara de lo que se pretendía y las conclusiones a las que se han llegado y del proceso seguido. Es uno de los capítulos mas importantes
\end{comment}

\section{Motivación}\label{sec:intro:motivación}

\begin{comment}
Problema a tratar, posibles causas, relevancia del problema
\end{comment}

En un momento en el que la población mundial acelera su crecimiento, la mayoría de las sociedades occidentales adolecen de un fenómeno contradictorio: su población se envejece rápidamente. Esta tendencia es consecuencia de la continua tendencia descendente de la tasa de fertilidad pero sobretodo por el incremento en la esperanza de vida de la población. No obstante este aumento en la longevidad no implica que la población sea más saludable. Más bien al contrario.

En su página web, la Organización Mundial de la Salud recoge que los mayores de 65 años son los más propensos a sufrir caídas, y que se producen aproximadamente 37 millones de caídas que requieren atención médica y recomienda priorizar la investigación relacionada con las caídas\cite{FactsFalls2018}. En este contexto es normal que proliferen estudios y soluciones para mejorar la atención de estos colectivos de riesgo.

Esta tendencia no es nueva, el estudio e implementación de sistemas para la detección de caídas lleva décadas en desarrollo\cite{fallindex00, Chen2005, Noury2007, Bourke2006}. En la actualidad se comercializan multitud de dispositivos para la detección de caídas.

Desde sistemas de propósito específico como las alarmas de detección de caídas en el baño basadas en una correa atada a la muñeca, a sistemas multi-propósito basados en dispositivos de captura y otro de procesado por separado, pasando por complejos métodos de captura y procesado de imágenes. Todos estos sistemas adolecen o bien de restricciones geográficas (su funcionamiento queda limitado a un entorno limitado por el alcance de las cámaras o cobertura del enlace radio con la base de procesado) o bien resultan poco prácticos e incluso invasivos para un grupo de usuarios que no está acostumbrado a lidiar con la tecnología.

\section{Planteamiento del trabajo}\label{sec:intro:planteamiento}

\begin{comment}
¿cómo se puede resolver el problema qué se propone descripción de objetivos en términos generales?
\end{comment}

La rápida evolución tecnológica permite tener a nuestra disposición cada vez más dispositivos \textit{vestibles} (por \textit{wearables} del inglés, dispositivos electrónicos que se incorporan en alguna parte del cuerpo para realizar una función) con sensores conectados a Internet. Estos dispositivos mejoran generación tras generación gracias a la creciente capacidad de cálculo de los microprocesadores, la miniaturización de los sensores y la reducción de costes, lo que ha permitido generalizar y extender y popularizar su uso al grueso de la sociedad. Estos avances se aprovechan ya parcialmente en algunas soluciones al problema de la detección de caídas. Se utilizan los datos capturados por acelerómetros triaxiales y/o sistemas inerciales de una pulsera de actividad o reloj inteligente ya sea para realizar un procesado analítico de los parámetros medidos y realizar la detección de una caída \cite{fallindex00, Chen2005,Bourke2006,Kangas2008,Bagala2012} (por ejemplo buscar una aceleración fuerte y abrupta) o bien para enviar este flujo de datos a un segundo sistema\cite{Luque2014,Vilarinho2015}, para realizar la detección mediante modelos complejos con grandes requisitos computacionales\cite{Cho2014, Aziz2017b,Putra2017}. Ambas soluciones tienen sus pros y contras. La primera tiene a su favor la alta disponibilidad al aunar captura y tratamiento en la misma unidad, pero falla al usar algoritmos poco precisos, al contrario que la segunda opción que se aprovecha de la mayor potencia de un segundo centro de cómputo para mejorar la detección a costa de una mayor complejidad en el sistema que impacta negativamente en su facilidad de uso y disponibilidad. Hay que tener en cuenta que el público objetivo de esta tecnología es un segmento de población con escasos conocimientos de nuevas tecnologías y poco habituado a su uso diario.

Este trabajo propone una solución que sea simple de usar, no invasivo y al mismo tiempo ofrezca resultados comparables al estado del arte. Para conseguir estos objetivos nos basaremos en aprovechar la mejora en rendimiento de los microprocesadores de sistemas llevables (como pueden ser las pulseras de actividad o los relojes inteligentes), para el tratamiento y predicción de series temporales para realizar la captura, tratamiento y detección de caídas en la misma unidad usando algoritmos de aprendizaje automático. Esta es la principal diferencia con los sistemas ya existentes: usar un algoritmo basado en los últimos avances en tecnologías de predicción temporal con inteligencia artificial, redes recurrentes GRU (\textit{Gated Recurrent Unit} un tipo de celda para formar redes neuronales recurrentes),  y un mecanismos de atención o eventos para reducir los requisitos de capacidad de cómputo. Este trabajo aporta al campo de la detección de caídas la novedad de usar la arquitectura codificador-decodificador de la red neuronal recurrente para conseguir una una secuencia de predicción en un único paso y a su vez, gracias al proceso de reducción y expansión de la dimensionalidad realizada por las redes de codificación y decodificación y mediante un comparador de la predicción con la señal real para detectar las caídas como anomalías basándonos en la baja capacidad de generalización de las redes neuronales a los casos para los que no ha sido entrenada. Este sistema permite simplificar el clasificador a un único caso: \textit{actividad} (o anomalía)

Estas tres mejoras (detección en dos etapas, predicción de varios pasos a la vez y simplificación del clasificador) permiten ejecutar el sistema con una latencia baja en un sistema discreto, no invasivo o molesto para el usuario final, como es llevar puesto un reloj de pulsera. El objetivo final es implementar una solución para la detección de caídas con gran precisión, mejor que la de los sistemas analíticos y próxima a la del estado del arte de los sistemas basados en aprendizaje automático que se ejecute exclusivamente en un dispositivo llevable.

\section{Estructura de la memoria}\label{seq:intro:estructura}
\begin{comment}
qué hay en cada uno de los subsiguientes capítulos
\end{comment}

Los capítulos que estructuran este documento se organizan de la siguiente forma: en el capítulo \ref{chap:stateofart} realizamos una introducción y revisión del del estado del arte y literatura relacionada con la problemática abordada. Haremos especial hincapié en el tratamiento de la actividad humana (Sección \ref{sect:sa_har}, y los diferentes acercamientos existentes para aboradar la detección de caidas en las secciones \ref{sa_modelos_analiticos,sa_modelos_ml,sa_modelos_hybridos} ya que en los siguientes capítulos haremos muchas referencias a estos trabajos. Habiendo establecido las bases de la situación actual de la tecnología y soluciones disponibles presentaremos el objetivo a abordar por la aplicación y una serie de metas previas e intermedias que faciliten la evaluación de los logros conseguidos en el capítulo \ref{chap:objetivos}.   

Antes de empezar con las explicaciones del trabajo realizado, introduciremos componentes, plataformas (\ref{req_hardware}), entornos (\ref{req_tflite}), bases de datos (\ref{req_base_de_datos}), modelos de detección ya existentes sobre las que trabajaremos (\ref{req_modelos}), sus necesidad, sus limitaciones y posibles alternativas\todo{poner referencias a todas las secciones citadas}, justificando siempre las decisiones en base al cumplimiento de los objetivos previamente descritos. Acto seguido nos adentraremos en el desarrollo e implementación de la solución en el capítulo \ref{chap:descripcion}. En esta parte detallaremos el trabajo realizado, el sistema, las diferentes aplicaciones que lo componen, sus arquitecturas así como detalles del modelo (\ref{eval_modelo}) utilizado, sus parámetros y funcionamiento.

Una vez explicado todo el proceso seguido para implementar toda la plataforma pasaremos a evaluar el sistema en el capítulo \ref{chap:eval}. Analizaremos los resultados obtenidos en cada una de las etapas de desarrollo, ajustaremos los diferentes parámtetros, compararemos con otras soluciones existentes y justificaremos las decisiones tomadas con dichos datos. Volveremos a revisar de forma más frugal estos resultados en el capítulo \ref{chap:conclusiones} para exponer las conclusiones obtenidas y presentar posibles líneas de trabajo futuro que por diversas razones no se han podido explorar. Al final en los apéndices detallaremos otros aspectos como las bases de datos usadas (Ap.\ref{app:dataset}) y la plataforma llevable usada (Apéndice.\ref{app:plataforma}).

%\end{document}
