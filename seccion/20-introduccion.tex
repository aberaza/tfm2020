\documentclass[../tfm.tex]{subfiles}

\begin{document}

Resumen esquemático de cada una de las partes del trabajo. Leer esta sección ha de dar una idea clara de lo que se pretendía y las concluseiones a las que se han llegado y del proceso seguido
Es uno de los capítulos mas importantes

\subsection{Motivación}
\info{Problema a tratar, posiles causas, relevancia del problema}

Este trabajo resulta de especial interés en el contexto actual en el que las sociedades de los países desarrollados sufren un rápido envejecimiento de sus poblaciones. El control contínuo de diferentes parámetros de salud de poblaciones propensas a estos permite ofrecer atención de calidad a menor coste. En este contexto es especialmente importante la detección de caidas. Las caídas son la principal causa de mortaldad de personas mayores o con problemas motores que viven solos.

Existen infinidad de dispositivos para la detección de caídas en el mercado existente. Desde sistemas de propósito específico como las alarmas de detección de caídas en el baño basadas en una correa atada a la muñeca, a sistemas multipropósito basados en un dispositivo de captura y otro de procesado por separado, pasando por complejos métodos de captura y procesado de imágenes. Todos estos sistemas adolecen o bien de restricciones geográficas (funcionan o bien en una habitación o entorno geográfico limitado por el alcance de las cámaras o cobertura del enlace radio con la base de procesado) o bien resultan poco prácticos e incluso obtrusivos en una sociedad no acostumbrada a vivir en un mundo dominado por la tecnología.

\subsection{plantemiento del trabajo}
\info{cómo se puede resolver el problema qué se propone descripción de objetivos en términos generales}

La evolución de la capacidad de cálculo de los microprocesadores y miniaturización de los sensores ha permitido generalizar y extender y popularizar el uso de dispositivos "\textit{llevables}" (por \textit{wearables} del inglés) al grueso de la sociedad. Estos avances se aprovechan ya parcialmente en algunas soluciones a este problema, donde se utilizan los datos capturados por el acelerómetro de una pulsera de actividad o reloj inteligente ya sea para realizar una detección simplista de una caida (una aceleración fuerte y abrupta) con altas tasas de fallos o bien para enviar este flujo de datos a un segundo sistema, normalmente un móvil, para realizar la detección. Ambas soluciones tienen sus pros y contras. La primera tiene a su favor la alta disponibilidad al aunar captura y tratamiento en la misma unidad, pero falla al usar algoritmos poco eficaces, al contrario que la segunda opción que se aprovecha de la mayor potencia de un segundo centro de cómputo para mejorar la detección a costa de una mayor complejidad en el sistema que reduzca su usabilidad y disponibilidad. Hay que tener en cuenta que el público objetivo de esta tecnología es un segmento de población con escasos conocimientos de nuevas tecnologías.

La solución propuesta es aprovechar la mejora en rendimiento de los procesadores de sistemas portables como los usados en relojes inteligentes para, al igual que en la primera solución, realizar tanto captura como tratamiento y detección de caídas en la misma unidad. La diferencia es que el algoritmo usado se base en los últimos avances en tecnologías de predicción temporal con inteligencia artificial, redes recurrentes LSTM bidireccionales con mecanismos de atención y detección de anomalías para implementar un sistema de detección autónomo en quasi-tiempo real que a la vez sea lo menos obtrusivo para el usuario final, como podría ser el simple hecho de llevar un reloj puesto.

El objetivo es por tanto implementar una solución de detección de caídas con alta precisión, siendo esta al menos comparable a la conseguida en sistemas con cómputo externo, que funcione exclusivamente en un reloj inteligente.


\subsection{Estructura de la memoria}
\info{qué hay en cada uno de los subsiguientes capítulos}
\todo{Desarrollar}
Un capítulo de estado del arte, otro de objetivos (qué aportamos sobre el estado del arte).

Un capítulo sobre la descripción del problema, el dataset y su estructura (requisitos). Explicar peculiaridades sobre las señales del acelerómetro usadas tanto a nivel físico (frecuencia, valores) como fisiológico. En este sentido explicar qué se cosiderará un episodio, su longitud y magnitudes elegidas justificar dichas decisiones así como exponer todo tratamiento de datos realizado al dataset. Explicar como se consigue la estacionariedad de la señal o qué efecto tiene la estacionariedad en la decisión del tamaño del episodio.

Otro sobre la aplicación y el algoritmo de detección final usado y como presenta los resultados (IHM).

Finalmente la presentación de resultados, evaluación con respecto a las soluciones existentes (simple detección por "threshold", sistemas basados en RNN para el smartphone e incluso un modelo arima o similar no basado en redes neuronales. Incluir precisión y si es posible latencia. Explicar mecanismos de test y como se ha conseguido el conjunto de datos de test. Si da tiempo a probarlo lo suficiente como para tener un conjunto de datos estadísticamente significativo, añadir los resultados.

Resumen de la conclusión.




\end{document}
