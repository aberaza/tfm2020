\documentclass[../tfm.tex]{subfiles}

\begin{document}

\todohide{Resumen esquemático de cada una de las partes del trabajo. Leer esta sección ha de dar una idea clara de lo que se pretendía y las concluseiones a las que se han llegado y del proceso seguido. Es uno de los capítulos mas importantes
}

\subsection{Motivación}
\info{Problema a tratar, posiles causas, relevancia del problema}

Este trabajo resulta de especial interés en el contexto actual en el que las sociedades de los países desarrollados sufren un rápido envejecimiento de sus poblaciones. Las personas mayores son más propensas a las caídas y estas suponen la principal causa de accidentes y hospitalizaciones de personas mayores de 65 años \todo{dharmita2019}. El control contínuo de la salud de poblaciones propensas a estos accidentes permite ofrecer atención de calidad a menor coste. En este contexto es especialmente importante la detección de caidas dentro de este marco de control y atención a estos grupos de población de riesgo.

Existen infinidad de dispositivos para la detección de caídas en el mercado. Desde sistemas de propósito específico como las alarmas de detección de caídas en el baño basadas en una correa atada a la muñeca, a sistemas multipropósito basados en dispositivos de captura y otro de procesado por separado, pasando por complejos métodos de captura y procesado de imágenes. Todos estos sistemas adolecen o bien de restricciones geográficas (funcionan o bien en una habitación o entorno geográfico limitado por el alcance de las cámaras o cobertura del enlace radio con la base de procesado) o bien resultan poco prácticos e incluso obtrusivos para un grupo de usuarios que no está acostumbrado a lidiar con la tecnología.

\subsection{Plantemiento del trabajo}
\todohide{¿cómo se puede resolver el problema qué se propone descripción de objetivos en términos generales?}

Hoy en día tenemos a nuestra disposición cada vez más dispositivos \textit{llevables} (por \textit{wearables} del inglés) con sensores conectados a internet. La evolución de la capacidad de cálculo de los microprocesadores y miniaturización de los sensores ha permitido generalizar y extender y popularizar el uso de dispositivos \textit{llevables} al grueso de la sociedad. Estos avances se aprovechan ya parcialmente en algunas soluciones al problema de la detección de caidas. Se utilizan los datos capturados por el acelerómetro de una pulsera de actividad o reloj inteligente ya sea para realizar un procesado analítico de los parámetros medidos y realizar la detección de una caida (por ejemplo buscar una aceleración fuerte y abrupta) o bien para enviar este flujo de datos a un segundo sistema, para realizar la detección mediante modelos complejos con grandes requisitos computacionales. Ambas soluciones tienen sus pros y contras. La primera tiene a su favor la alta disponibilidad al aunar captura y tratamiento en la misma unidad, pero falla al usar algoritmos poco precisos, al contrario que la segunda opción que se aprovecha de la mayor potencia de un segundo centro de cómputo para mejorar la detección a costa de una mayor complejidad en el sistema que impacta negativamente en su usabilidad y disponibilidad. Hay que tener en cuenta que el público objetivo de esta tecnología es un segmento de población con escasos conocimientos de nuevas tecnologías y poco habituado a su uso diario.

Este trabajo propone una salución basada en aprovechar la mejora en rendimiento de los procesadores de sistemas portables como los usados en relojes inteligentes para realizar tanto captura como tratamiento y detección de caídas en la misma unidad. La diferencia con los sistemas ya existentes, es que el algoritmo usado se base en los últimos avances en tecnologías de predicción temporal con inteligencia artificial, redes recurrentes GRU y un mecanismos de atención o eventos para reducir los requisitos computacionales y obtener  un sistema de detección autónomo en quasi-tiempo real que a la vez sea lo menos obtrusivo para el usuario final, como podría ser el simple hecho de llevar un reloj puesto.

El objetivo es por tanto implementar una solución de detección de caídas con alta precisión, siendo esta al menos comparable a la conseguida en sistemas con cómputo externo, que funcione exclusivamente en un reloj inteligente.


\subsection{Estructura de la memoria}
\todo[disable]{qué hay en cada uno de los subsiguientes capítulos}
\todo{Desarrollar}

El siguiente trabajo se estructura con una revisión del estado del arte y literatura relacionada en el capítulo \ref{chap:stateofart}, seguido de la definición de objetivos y de requisitos previos en los capítulos \ref{chap:objetivos} y \ref{chap:requisitos}. Posteriormente nos adentraremos en el desarrollo e implementación de la solución en el capítulo \ref{chap:descripcion} y en el capítulo \ref{chap:eval} evaluaremos el sistema y sus resultados contra soluciones existentes, antes de exponer las conclusiones (capítulo \ref{chap:conclusiones}) y trabajo futuro. Al final en los apéndices detallaremos otros aspectos como el conjunto de datos usado (Ap.\ref{app:dataset}) y la plataforma usada (Ap.\ref{app:plataforma}).

Entre la literatura previa y de referencia a este trabajo introduciremos los sistemas de detección de actividad humana en \ref{sect:sa_har}, los diferentes modelos usados \ref{sa_modelos_analiticos}, \ref{sa_modelos_ml} antes de introducir los modelos híbridos en \ref{sa_modelos_hybridos}. Intruduciremos también las redes neuronales recurrentes como herramienta de procesado de series temporales \ref{sa_rnn} y la problemática de la optimización de modelos \ref{sa_optimizacion} tanto a nivel computacional como de consumo de recursos y memoria.

Estos conceptos se retoman posteriormente para definir los prerrequisitos: plataforma de desarrollo \ref{req_hardware} como herramientas de software \ref{req_tflite} y conjuntos de datos adecuados para entrenamiento y validación de resultados \ref{req_corpus}. Trataremos la situación actual y soluciones disponibles así como las limitaciones y sus posibles soluciones. Trataremos de nuevo el problema de los diferentes tipos de modelos de detección de caídas \ref{req_modelos} relacionando sus capacidades con las limitaciones tanto de los conjuntos de datos como del soporte físico y computacional usado y su adecuación a la solución propuesta.

En las secciones posteriores detallaremos el algoritmo y modelo usado para la detección de caídas \ref{desc_modelo}, estructura \ref{desc_archi}, implementación \ref{desc_impl} e interfaz\todo{añadir referencia} de la aplicación y las optimizaciones implementadas \ref{desc_optim}, para a la postre evaluar tanto la viabilidad del modelo usado respecto a otras implementaciones \ref{eval_modelo}, como la usabilidad de la aplicación y su adecuación al problema a tratar\todo{añadir referencia}.

Finalmente tras presentar los resultados obtenidos analizaremos posibles mejoras futuras al sistema así como alternativas que no ha dado tiempo a explorar en este trabajo

\end{document}
