% !TeX root = ../tfm.tex
%! TEX root = ../tfm.tex



\section{Entrenamiento de Modelos}
\figura{evolucionLR}{fig:LRvariable}{Evolución del valor LR exponencial}

\section{Modelos \ifell/}


\section{Matrices de confusión}

\subsection{Modelos Bourke}
Entrenados y validados usando \sisfall/
\figura[0.7]{BourkeCONF_Matrix}{fig:app:confmatrix:bourke}{Matrices de confusión para BourkeU, BourkeL y BourkeUL}

\subsection{Modelos RNN e IFELL}
La denominación sigue el esquema \textit{<RNN|IFELL>-<tipo>(<unidades>)[<atributos>]}. Donde:
\begin{itemize}
  \item \textit{RNN} Denomina a los modelos entrenados y validados con \sisfall/
  \item \textit{IFELL} Designa a los modelos entrenados con \accelcapture/ y validados con \sisfall/
  \item \textit{<tipo>} \textbf{P} para modelos de predicción y \textbf{R} para modelos de reconstrucción
  \item \textit{<celdas>} Es el número de unidades que componen cada celda GRU
  \item \textit{<atributos>} El número de atributos o dimensiones del espacio intermedio
\end{itemize}

\paragraph{Modelos RNN de 175 unidades}
\begin{figure}[H]
  \centering
  \begin{subfigure}[t]{0.48\textwidth}
      \centering
      \pincludegraphics[1.0]{RNNP(175)30CONF.pgf}
      \caption{\footnotesize \label{fig:app:confmatrix:P17530}Modelo RNN-P(175)[30]}
  \end{subfigure}
  \hfill
  \begin{subfigure}[t]{0.48\textwidth}
      \centering
      \pincludegraphics[1.0]{RNNR(175)30CONF.pgf}
      \caption{\footnotesize \label{fig:app:confmatrix:R17530}Modelo RNN-R(175)[30]}
  \end{subfigure}
  \begin{subfigure}[t]{0.48\textwidth}
      \centering
      \pincludegraphics{RNNP(175)50CONF.pgf}
      \caption{\footnotesize \label{fig:app:confmatrix:P17550}Modelo RNN-P(175)[50]}
  \end{subfigure}
  \hfill
  \begin{subfigure}[t]{0.48\textwidth}
      \centering
      \pincludegraphics[1.0]{RNNR(175)50CONF.pgf}
      \caption{\footnotesize \label{fig:app:confmatrix:R17550}Modelo RNN-R(175)[50]}
  \end{subfigure}
  \caption{\label{fig:app:confmatrix:175} Matrices confusión para modelos RNN de 175 unidades por celda}
\end{figure}

\paragraph{Modelos RNN de 350 unidades}

\begin{figure}[H]
  \centering
  \begin{subfigure}[t]{0.48\textwidth}
      \centering
      \pincludegraphics[1.0]{RNNP(350)30CONF.pgf}
      \caption{\footnotesize \label{fig:app:confmatrix:P35030}Modelo RNN-P(350)[30]}
  \end{subfigure}
  \hfill
  \begin{subfigure}[t]{0.48\textwidth}
      \centering
      \pincludegraphics[1.0]{RNNR(350)30CONF}
      \caption{\footnotesize \label{fig:app:confmatrix:R35030}Modelo RNN-R(350)[30]}
  \end{subfigure}
  \begin{subfigure}[t]{0.48\textwidth}
      \centering
      \pincludegraphics[1.0]{RNNP(400)50CONF}
      \caption{\footnotesize \label{fig:app:confmatrix:P35050}Modelo RNN-P(350)[50]}
  \end{subfigure}
  \hfill
  \begin{subfigure}[t]{0.48\textwidth}
      \centering
      \pincludegraphics[1.0]{RNNR(350)50CONF}
      \caption{\footnotesize \label{fig:app:confmatrix:R35050}Modelo RNN-R(350)[50]}
  \end{subfigure}
  \caption{\label{fig:app:confmatrix:350} Matrices confusión para modelos RNN de 350 unidades por celda}
\end{figure}

\paragraph{Modelos IFELL}
\begin{figure}[H]
  \centering
  \begin{subfigure}[t]{0.48\textwidth}
      \centering
      \pincludegraphics[1.0]{IFELLP(175)50CONF}
      \caption{\footnotesize \label{fig:app:confmatrix:ifell:P17550}Modelo IFELL-P(175)[50]}
  \end{subfigure}
  \hfill
  \begin{subfigure}[t]{0.48\textwidth}
      \centering
      \pincludegraphics[1.0]{IFELLR(175)50CONF}
      \caption{\footnotesize \label{fig:app:confmatrix:ifell:R17550}Modelo IFELL-R(175)[50]}
  \end{subfigure}
  \caption{\label{fig:app:confmatrix:ifell:175} Matrices confusión para modelos IFELL de 175 unidades por celda}
\end{figure}
