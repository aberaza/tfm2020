% !TeX root = ../tfm.tex
%! TEX root = ../tfm.tex
En sociedades cada vez más envejecidas aumenta la necesidad de controlar de forma continua la salud de las personas mayores. En especial de identificar las caídas, una de las principales causas de mortalidad en personas mayores. Con este objetivo existen hoy en día soluciones que sirven a este propósito, generalmente sistemas específicos que usan un sensor llevable y una unidad de cómputo independiente. Estos sistemas resultan caros, altamente intrusivos y restringidos pues sólo pueden funcionar dentro del área de alcance de la unidad de cálculo. En este trabajo se presenta un sistema de detección de caídas que funciona de forma autónoma en un reloj inteligente o \textit{smartwatch} integrando en un mismo dispositivo las capacidades de captura, detección y alerta en un componente que no resulte invasivo para el usuario como es un reloj de pulsera.

{\bf Palabras Clave:} Detección de caídas, Detección de actividades, RNN, GRU, Bourke, Detección de anomalías, codificador/decodificador

