\documentclass[11pt,a4paper,spanish]{book}
\usepackage{unir} %\usepackage{estilo_unir}

%para gráficos
\usepackage{standalone}
\usepackage{tikz}
%\usepackage{import}
\usepackage{pgf}
\usepackage{pgfplots}
\pgfplotsset{compat=1.16}
\graphicspath{{img/}{../img/}} % Paths por defecto para las imagenes
\usepackage{smartdiagram}
\usepackage{definitions}
\usepackage{verbatim} % to use \begin{comment} for block comments

%tablas más presentablesi 
\usepackage{booktabs}
\usepackage{graphicx}
\usepackage{rotating}
% Siguientes 3 para imágenes compuestas
\usepackage{float}
\usepackage[caption = false]{subfig}
%\usepackage[export]{adjustbox}

% para usar \currenttime
\usepackage{datetime}

%\usepackage{amsmath}% ya se define en unir.sty
\DeclareMathOperator*{\argmax}{arg\,max}
\DeclareMathOperator*{\softmax}{\textup{softmax}}


% Fix warning \headheigt is to changeso small
%\setlength{\headheight}{14pt}%

% Permitir modularizar el trabajo
\usepackage{subfiles} % recomiendan cargarlo al final del preambulo

%---------------------------
%título del trabajo y autor
%---------------------------
\title{Detección de caídas con Dispositivos Vestibles y Redes Neuronales Recurrentes}
\author{Aritz Beraza Garayalde}
\date{\currenttime}
%\director{Sonia Valladares Rodriguez}
\director{Claudia Villalonga Palliser}
\nombreciudad{Paris}

%---------------------------
%margenes
%---------------------------
%\usepackage[margin=1.9cm]{geometry}
%---------------------------
%---------------------------
%---------------------------
%--------------------------
\begin{document}
%----------------
% overfull hbox : hide messages
%\hbadness=10000
%\the\tolerance,
%\the\pretolerance,
%\the\hbadness,
%\the\hfuzz,
%\the\emergencystretch,
% Second test
%\pretolerance=150
\tolerance=530
%\hbadness=752
%\hfuzz0pt
%\emergencystretch=0em
% underfull vbox fixes::
%\setlength\parskip{1em plus 2pt}
%\setlength\parskip{\baselineskip}



%\selectlanguage{spanish}
\renewcommand{\listfigurename}{Índice de Ilustraciones}
\renewcommand{\listtablename}{Índice de Tablas}
\renewcommand{\contentsname}{Índice de Contenidos}
\renewcommand{\figurename}{Figura}
\renewcommand{\tablename}{Tabla}

\maketitle
\frontmatter
\tableofcontents
\listoffigures
\listoftables

% Posicion original de mainmatter
\mainmatter

%\chapter{Resumen}
\section*{Resumen}
%\subfile{seccion/10-abstract.es.tex}
\documentclass[../tfm.tex]{subfiles}
\begin{document}
En sociedades cada vez más envejecidas aumenta la necesidad de controlar de forma contínua la salud de las personas mayores, y en concreto identificar las caídas, una de las principales causas de mortalidad en personas mayores. A este respecto existen hoy en día soluciones que sirven únicamente a este propósito, altamente intrusivas, o soluciones basadas en un reloj inteligente y una unidad de procesado independiente que reduce la autonomía. En este trabajo se presenta un sistema de detección de caídas que funciona de forma autónoma en un reloj inteligente o smartwatch ofreciendo un alto grado de precisión en un sistema no intrusivo y autónomo como es un reloj de pulsera.

{\bf Palabras Clave:} Detección de caídas, RNN, LSTM, Detección Anomalías 

\end{document}



%\chapter{Abstract}
\section*{Abstract}
%\subfile{seccion/10-abstract.en.tex}
% !TeX root = ../tfm.tex
%\documentclass[../tfm.tex]{subfiles}
%\begin{document}
\begin{otherlanguage}{english}
The need to continuously monitor elder people's health, and specially detect falls, one of the main death causes of this population, increases in an aging society. With this objective, many solutions exist nowadays, but they are either based in expensive propietary single purpose hardware, usually a wearable sensor and an external computer. These devices are expensive, highly obtrusive and restricted to the an area where the sensor can maintain a link with the main computer. This work will present a fall detection system based purely in a smartwatch, integrating in the same unit the capture, detection and alert components. This solution is presented to the user as an unobtrusive and friendly wristwatch.

{\bf Keywords:} Fall Detection, Human Activity Detection, RNN, GRU, Bourke, Outlier detection
\end{otherlanguage}
%\end{document}




\chapter{Introducción}\label{chap:intro}
%\subfile{seccion/20-introduccion}
%\documentclass[../tfm.tex]{subfiles}

\begin{document}

\todohide{Resumen esquemático de cada una de las partes del trabajo. Leer esta sección ha de dar una idea clara de lo que se pretendía y las concluseiones a las que se han llegado y del proceso seguido. Es uno de los capítulos mas importantes
}

\subsection{Motivación}
\info{Problema a tratar, posiles causas, relevancia del problema}

Este trabajo resulta de especial interés en el contexto actual en el que las sociedades de los países desarrollados sufren un rápido envejecimiento de sus poblaciones. Las personas mayores son más propensas a las caídas y estas suponen la principal causa de accidentes y hospitalizaciones de personas mayores de 65 años \todo{dharmita2019. Añadir referencia}. El control contínuo de la salud de poblaciones propensas a estos accidentes permite ofrecer atención de calidad a menor coste. En este contexto es especialmente importante la detección de caidas dentro de este marco de control y atención a estos grupos de población de riesgo.

Existen infinidad de dispositivos para la detección de caídas en el mercado. Desde sistemas de propósito específico como las alarmas de detección de caídas en el baño basadas en una correa atada a la muñeca, a sistemas multipropósito basados en dispositivos de captura y otro de procesado por separado, pasando por complejos métodos de captura y procesado de imágenes. Todos estos sistemas adolecen o bien de restricciones geográficas (funcionan o bien en una habitación o entorno geográfico limitado por el alcance de las cámaras o cobertura del enlace radio con la base de procesado) o bien resultan poco prácticos e incluso obtrusivos para un grupo de usuarios que no está acostumbrado a lidiar con la tecnología.

\subsection{Plantemiento del trabajo}
\todohide{¿cómo se puede resolver el problema qué se propone descripción de objetivos en términos generales?}

Hoy en día tenemos a nuestra disposición cada vez más dispositivos \textit{llevables} (por \textit{wearables} del inglés) con sensores conectados a internet. La evolución de la capacidad de cálculo de los microprocesadores y miniaturización de los sensores ha permitido generalizar y extender y popularizar el uso de dispositivos \textit{llevables} al grueso de la sociedad. Estos avances se aprovechan ya parcialmente en algunas soluciones al problema de la detección de caidas. Se utilizan los datos capturados por el acelerómetro de una pulsera de actividad o reloj inteligente ya sea para realizar un procesado analítico de los parámetros medidos y realizar la detección de una caida (por ejemplo buscar una aceleración fuerte y abrupta) o bien para enviar este flujo de datos a un segundo sistema, para realizar la detección mediante modelos complejos con grandes requisitos computacionales. Ambas soluciones tienen sus pros y contras. La primera tiene a su favor la alta disponibilidad al aunar captura y tratamiento en la misma unidad, pero falla al usar algoritmos poco precisos, al contrario que la segunda opción que se aprovecha de la mayor potencia de un segundo centro de cómputo para mejorar la detección a costa de una mayor complejidad en el sistema que impacta negativamente en su usabilidad y disponibilidad. Hay que tener en cuenta que el público objetivo de esta tecnología es un segmento de población con escasos conocimientos de nuevas tecnologías y poco habituado a su uso diario.

Este trabajo propone una salución basada en aprovechar la mejora en rendimiento de los procesadores de sistemas portables como los usados en relojes inteligentes para realizar tanto captura como tratamiento y detección de caídas en la misma unidad. La diferencia con los sistemas ya existentes, es que el algoritmo usado se base en los últimos avances en tecnologías de predicción temporal con inteligencia artificial, redes recurrentes GRU y un mecanismos de atención o eventos para reducir los requisitos computacionales y obtener  un sistema de detección autónomo en quasi-tiempo real que a la vez sea lo menos obtrusivo para el usuario final, como podría ser el simple hecho de llevar un reloj puesto.

El objetivo es por tanto implementar una solución de detección de caídas con alta precisión, siendo esta al menos comparable a la conseguida en sistemas con cómputo externo, que funcione exclusivamente en un reloj inteligente.


\subsection{Estructura de la memoria}
\todo[disable]{qué hay en cada uno de los subsiguientes capítulos}
\todo{Desarrollar}

El siguiente trabajo se estructura con una revisión del estado del arte y literatura relacionada en el capítulo \ref{chap:stateofart}, seguido de la definición de objetivos y de requisitos previos en los capítulos \ref{chap:objetivos} y \ref{chap:requisitos}. Posteriormente nos adentraremos en el desarrollo e implementación de la solución en el capítulo \ref{chap:descripcion} y en el capítulo \ref{chap:eval} evaluaremos el sistema y sus resultados contra soluciones existentes, antes de exponer las conclusiones (capítulo \ref{chap:conclusiones}) y trabajo futuro. Al final en los apéndices detallaremos otros aspectos como el conjunto de datos usado (Ap.\ref{app:dataset}) y la plataforma usada (Ap.\ref{app:plataforma}).

Entre la literatura previa y de referencia a este trabajo introduciremos los sistemas de detección de actividad humana en \ref{sect:sa_har}, los diferentes modelos usados \ref{sa_modelos_analiticos}, \ref{sa_modelos_ml} antes de introducir los modelos híbridos en \ref{sa_modelos_hybridos}. Intruduciremos también las redes neuronales recurrentes como herramienta de procesado de series temporales \ref{sa_rnn} y la problemática de la optimización de modelos \ref{sa_optimizacion} tanto a nivel computacional como de consumo de recursos y memoria.

Estos conceptos se retoman posteriormente para definir los prerrequisitos: plataforma de desarrollo \ref{req_hardware} como herramientas de software \ref{req_tflite} y conjuntos de datos adecuados para entrenamiento y validación de resultados \ref{req_corpus}. Trataremos la situación actual y soluciones disponibles así como las limitaciones y sus posibles soluciones. Trataremos de nuevo el problema de los diferentes tipos de modelos de detección de caídas \ref{req_modelos} relacionando sus capacidades con las limitaciones tanto de los conjuntos de datos como del soporte físico y computacional usado y su adecuación a la solución propuesta.

En las secciones posteriores detallaremos el algoritmo y modelo usado para la detección de caídas \ref{desc_modelo}, estructura \ref{desc_archi}, implementación \ref{desc_impl} e interfaz\todo{añadir referencia} de la aplicación y las optimizaciones implementadas \ref{desc_optim}, para a la postre evaluar tanto la viabilidad del modelo usado respecto a otras implementaciones \ref{eval_modelo}, como la usabilidad de la aplicación y su adecuación al problema a tratar\todo{añadir referencia}.

Finalmente tras presentar los resultados obtenidos analizaremos posibles mejoras futuras al sistema así como alternativas que no ha dado tiempo a explorar en este trabajo

\end{document}

\documentclass[../tfm.tex]{subfiles}

\begin{document}

\todohide{Resumen esquemático de cada una de las partes del trabajo. Leer esta sección ha de dar una idea clara de lo que se pretendía y las concluseiones a las que se han llegado y del proceso seguido. Es uno de los capítulos mas importantes
}

\subsection{Motivación}
\info{Problema a tratar, posiles causas, relevancia del problema}

Este trabajo resulta de especial interés en el contexto actual en el que las sociedades de los países desarrollados sufren un rápido envejecimiento de sus poblaciones. Las personas mayores son más propensas a las caídas y estas suponen la principal causa de accidentes y hospitalizaciones de personas mayores de 65 años \todo{dharmita2019. Añadir referencia}. El control contínuo de la salud de poblaciones propensas a estos accidentes permite ofrecer atención de calidad a menor coste. En este contexto es especialmente importante la detección de caidas dentro de este marco de control y atención a estos grupos de población de riesgo.

Existen infinidad de dispositivos para la detección de caídas en el mercado. Desde sistemas de propósito específico como las alarmas de detección de caídas en el baño basadas en una correa atada a la muñeca, a sistemas multipropósito basados en dispositivos de captura y otro de procesado por separado, pasando por complejos métodos de captura y procesado de imágenes. Todos estos sistemas adolecen o bien de restricciones geográficas (funcionan o bien en una habitación o entorno geográfico limitado por el alcance de las cámaras o cobertura del enlace radio con la base de procesado) o bien resultan poco prácticos e incluso obtrusivos para un grupo de usuarios que no está acostumbrado a lidiar con la tecnología.

\subsection{Plantemiento del trabajo}
\todohide{¿cómo se puede resolver el problema qué se propone descripción de objetivos en términos generales?}

Hoy en día tenemos a nuestra disposición cada vez más dispositivos \textit{llevables} (por \textit{wearables} del inglés) con sensores conectados a internet. La evolución de la capacidad de cálculo de los microprocesadores y miniaturización de los sensores ha permitido generalizar y extender y popularizar el uso de dispositivos \textit{llevables} al grueso de la sociedad. Estos avances se aprovechan ya parcialmente en algunas soluciones al problema de la detección de caidas. Se utilizan los datos capturados por el acelerómetro de una pulsera de actividad o reloj inteligente ya sea para realizar un procesado analítico de los parámetros medidos y realizar la detección de una caida (por ejemplo buscar una aceleración fuerte y abrupta) o bien para enviar este flujo de datos a un segundo sistema, para realizar la detección mediante modelos complejos con grandes requisitos computacionales. Ambas soluciones tienen sus pros y contras. La primera tiene a su favor la alta disponibilidad al aunar captura y tratamiento en la misma unidad, pero falla al usar algoritmos poco precisos, al contrario que la segunda opción que se aprovecha de la mayor potencia de un segundo centro de cómputo para mejorar la detección a costa de una mayor complejidad en el sistema que impacta negativamente en su usabilidad y disponibilidad. Hay que tener en cuenta que el público objetivo de esta tecnología es un segmento de población con escasos conocimientos de nuevas tecnologías y poco habituado a su uso diario.

Este trabajo propone una salución basada en aprovechar la mejora en rendimiento de los procesadores de sistemas portables como los usados en relojes inteligentes para realizar tanto captura como tratamiento y detección de caídas en la misma unidad. La diferencia con los sistemas ya existentes, es que el algoritmo usado se base en los últimos avances en tecnologías de predicción temporal con inteligencia artificial, redes recurrentes GRU y un mecanismos de atención o eventos para reducir los requisitos computacionales y obtener  un sistema de detección autónomo en quasi-tiempo real que a la vez sea lo menos obtrusivo para el usuario final, como podría ser el simple hecho de llevar un reloj puesto.

El objetivo es por tanto implementar una solución de detección de caídas con alta precisión, siendo esta al menos comparable a la conseguida en sistemas con cómputo externo, que funcione exclusivamente en un reloj inteligente.


\subsection{Estructura de la memoria}
\todo[disable]{qué hay en cada uno de los subsiguientes capítulos}
\todo{Desarrollar}

El siguiente trabajo se estructura con una revisión del estado del arte y literatura relacionada en el capítulo \ref{chap:stateofart}, seguido de la definición de objetivos y de requisitos previos en los capítulos \ref{chap:objetivos} y \ref{chap:requisitos}. Posteriormente nos adentraremos en el desarrollo e implementación de la solución en el capítulo \ref{chap:descripcion} y en el capítulo \ref{chap:eval} evaluaremos el sistema y sus resultados contra soluciones existentes, antes de exponer las conclusiones (capítulo \ref{chap:conclusiones}) y trabajo futuro. Al final en los apéndices detallaremos otros aspectos como el conjunto de datos usado (Ap.\ref{app:dataset}) y la plataforma usada (Ap.\ref{app:plataforma}).

Entre la literatura previa y de referencia a este trabajo introduciremos los sistemas de detección de actividad humana en \ref{sect:sa_har}, los diferentes modelos usados \ref{sa_modelos_analiticos}, \ref{sa_modelos_ml} antes de introducir los modelos híbridos en \ref{sa_modelos_hybridos}. Intruduciremos también las redes neuronales recurrentes como herramienta de procesado de series temporales \ref{sa_rnn} y la problemática de la optimización de modelos \ref{sa_optimizacion} tanto a nivel computacional como de consumo de recursos y memoria.

Estos conceptos se retoman posteriormente para definir los prerrequisitos: plataforma de desarrollo \ref{req_hardware} como herramientas de software \ref{req_tflite} y conjuntos de datos adecuados para entrenamiento y validación de resultados \ref{req_corpus}. Trataremos la situación actual y soluciones disponibles así como las limitaciones y sus posibles soluciones. Trataremos de nuevo el problema de los diferentes tipos de modelos de detección de caídas \ref{req_modelos} relacionando sus capacidades con las limitaciones tanto de los conjuntos de datos como del soporte físico y computacional usado y su adecuación a la solución propuesta.

En las secciones posteriores detallaremos el algoritmo y modelo usado para la detección de caídas \ref{desc_modelo}, estructura \ref{desc_archi}, implementación \ref{desc_impl} e interfaz\todo{añadir referencia} de la aplicación y las optimizaciones implementadas \ref{desc_optim}, para a la postre evaluar tanto la viabilidad del modelo usado respecto a otras implementaciones \ref{eval_modelo}, como la usabilidad de la aplicación y su adecuación al problema a tratar\todo{añadir referencia}.

Finalmente tras presentar los resultados obtenidos analizaremos posibles mejoras futuras al sistema así como alternativas que no ha dado tiempo a explorar en este trabajo

\end{document}


\chapter{Contexto y Estado del Arte}\label{chap:stateofart}
%\subfile{seccion/30-estado-arte}
\documentclass[../tfm.tex]{subfiles}

\begin{document}

Históricamente, los modelos de predicción de caídas están íntimamente ligados al estudio del reconocimiento de actividades humanas (\textit{HAR} del inglés \textit{Human Activity Recognition}): discernir a partir de los datos de los sensores qué acción (\textit{AVC} por \textit{Actividad Vida Cotidiana}, o \textit{ADL} por sus siglas en inglés) estaba realizando el individuo. Según \cite[p.10692]{Ozdemir2014} una caída o golpe no es más que otra actividad y deben considerarse una parte de este problema ya que "son eventos que suceden de forma involuntaria e inesperada durante la realización de las mismas".


\section{Detección de actividad humana y caídas}\label{sect:sa_har}

En \cite[p.2]{Anita2020} se recopilan y definen los principios básicos que debería cumplir un buen sistema de deteción de caídas y que replicamos a continuación:

\begin{itemize}
  \item No ser obtrusivos
  \item No restringir la mobilidad del sujeto
  \item Tener baja latencia
  \item Ser capaz de diferenciar caídas de otras actividades humanas
\end{itemize}

En \cite{Musci2020,Anita2020}, se introducen varias clasificaciones de estos sistemas. En el contexto del presente trabajo, resultan de especial interés la rama de los sistemas \textit{llevables}: "Aquellos en los que los sensores usados para la detección se encuentran embebidos en un dispositivo que debe llevar pueto el sujeto, como por ejemplo en una pulsera" \cite[p.3]{Anita2020}. A su vez podemos dividir estos según el tipo de modelo utilizado para analizar los datos \cite{Anita2020,Lim2014}, en "dos acercamientos principales, los basados en cotas o límites y los basados en aprendizaje automático"\cite[p.1]{Lim2014}

\subsection{Modelos Analíticos: cálculos basados en cotas}\label{sa_modelos_analiticos}

Los primeros acercamientos (\cite{fallindex00, Kangas2008}) al problema de la detección de caídas utilizaron modelos matemáticos basados en el análisis de diferentes parámetros físicos como la aceleración vertical, módulo de la aceleración y postura. Como observa \cite{Bagala2012}, estos métodos puedan conseguir buenos resultados aplicados a pruebas de laboratorio o experimentos controlados, pero no en experimentos de uso real. En parte por un sobreajuste de la calibración de los valores al experimento. En general estos métodos contraponen sensibilidad a especificidad o a la inversa, como puede deducirse de los resultados de \cite{Chen2005,Bourke2006,Kangas2008,Vilarinho2015} y verifica \cite{Anita2020}. A pesar de ello, estos métodos siguen siendo ampliamente utilizados hoy en día por la baja complejidad computacional que requieren, lo que permite ser integrados en pequeños dispositivos poco obtrusivos y con bajas latencias, cumpliendo tres de los cuatro requisitos antes mencionados.


Sobre cálculos basados en cotas, tenemos \cite{fallindex00}, Kangas\cite{Kangas2008} lo compara con el uso ya sea de aceleración vertical o modulo de la aceleración y de nuevo obtiene resultados dispares según el tipo de caída simulada.

La posición de los sensores es también importante, como recalcan \cite{Bagala2012, Vilarinho2015}. Ambos usan smartphones o pulseras de actividad en diferentes posiciones para la captura de la aceleración con resultados muy dispares según el tipo de caída o el tipo de experimento. En el caso de \cite{Vilarinho2015} a pesar de introducir las pulseras como instrumento de captura, delega en un equipo externo (en su caso un teléfono) el análisis y detección de caídas, de manera similar a los experimentos conducidos por \cite{Luque2014} quien añade la posibilidad de realizar el cómputo en un servidor conectado a internet.

\info{REFERENCIA PERDIDA:
Luque2014??
Realmente el monitorizado básico no está tan lejos del resto de métodos e incluso son mejores que algunas implementaciones propietarias comerciales.
Recalca: alta variabilidad de los resultados segúl el tipo de caída (están demasiado optimizados para un tipo concreto, no son de uso general), también hace hincapié en los problemas de usabilidad por parte de usuarios no expertos.}

\todohide{
1- Monitorizado básico (usar módulo del vector aceleración y un threshold X )
2- Fall Index (Yoshida,  T.;  Mizuno,  F.;  Hayasaka,  T.;  Tsubota,  K.;  Wada,  S.;  Yamaguchi,  T.  A  Wearable  Computer System for a Detection and Prevention of Elderly Users from Falling. In Proceedings of  the  12th  International  Conference  on  Biomedical  and  Medical  Engineering  (ICBME),  Singapore, Singapore, 7–10 December 2005.  )
3- PerFallD (Dai, J.; Bai, X.; Yang, Z.; Shen, Z.; Xuan, D. PerFallD: A Pervasive Fall Detection System using Mobile   Phones.   In   Proceedings   of   the   8th   IEEE   International   Conference   on   Pervasive   Computing  and  Communications  Workshops  (PERCOM  Workshops),  Mannheim,  Germany,    29 March–2 April 2010; pp. 292–297) (usa también un giroscopio)
4- iFall (Sposaro,  F.;  Tyson,  G.  IFall:  An  Android  Application  for  Fall  Monitoring  and  Response.    In Proceedings of the Annual International Conference of the IEEE Engineering in Medicine and Biology Society (EMBC 2009), Minneapolis, MN, USA, 2–6 September 2009; pp. 6119–6122. )}




\subsection{Modelos de Aprendizaje Automático}\label{sa_modelos_ml}

La segunda gran familia de modelos utiliza técnicas como \textit{K-Nearest Neighbours}, LSM (\textit{Least Square Method} o \textit{Método de Mínimos Cuadrados}), SVM, Detección Bayesiana, DTW y Redes Neuronales aplicadas a la detección con muy buenos resultados como se extrae de \cite{Ozdemir2014}. En \cite[p.6]{Musci2020} se llega a la conclusión de que el uso de la información de la aceleración únicamente es una fuente de información suficientemente fiable para obtener métricas de detección satisfactorias usando una red neuronal recurrente.

Usando inteligencia artificial, Shi \cite{Shi2020} describe un método basado en acelerómetro triaxial llevado en la cintura para predecir caidas, tratando una caida como una actividad humana más y usando métodos similares a los usados para la predicción de actividad humana (CNN). Casilari en \cite[p.17]{Casilari2020} usa una red de 4 capas convolucionales para el mismo fin, y compara el rendimiento sobre multitud de datasets obteniendo que si bien se puede optimizar el rendimiento para un conjunto en concreto logrando resultados de precisión y especificidad del orden del 100\%, es muy costoso generar un modelo que generalice a todos los datasets. A su vez analiza el impacto de otros parámetros como el tamaño de la ventana temporal \cite[p.16]{Casilari2020}, considerando que a partir de 5 segundos no hay ningún beneficio extra en aumentar el tamaño de la ventana. En \cite[p.59]{Hassan2019} usan una ventana de 1s consiguiendo mejorar los clasificadores existentes. Resultado que coincide con \cite[p.2]{Liu2018} que deslimita la duración de una caída entre los 0,4 y 0,8 segundos, demostrando que una frecuencia de muestreo de 21,3Hz es suficiente cuando se utilizan modelos de aprendizaje automático. Madrano \cite{tfall} usa K-NearestNeighbours y SVM para detectar actividad y varios tipos de caídas, mostrando buenos resultados con SVM que además parece ser capaz de distinguir caídas para las que no ha sido entrenado.

Finalmente, en \cite{Anita2020} tenemos un resumen con los últimos avances y resultados. Introduce otras fuentes de información como es el riesgo biológico, o predisposicion a la caída debido a la edad y otros factores de deterioro de la salud.

Varios estudios \cite{Luque2014,Aziz2017,Aziz2017b} llegan a conclusiones similares respecto a la alta variabilidad de los resutlados según el conjunto de datos de entrenamiento usado, especialmente en los métodos basados en cotas. Sin embargo en \cite[p.9]{Aziz2017b} confirma el resultado de \cite[tfall] sobre la mayor capacidad de generalización de los modelos basados en SVM. A la ver que introduce  encuentra que SVM llega a generalizar incluso con tipos de caídas para las que no había sido entrenado. En la segunda obra analiza en más detalle la detección con SVM usando caídas reales y sugiere usar una combinación de métodos basados en cotas junto con técnicas de aprendizaje automático para obtener mejores estimadores.

\subsection{Modelos basados en eventos o híbridos}\label{sa_modelos_hybridos}

Putra \cite{Putra2017} y Lim \cite{Lim2014} definen modelos híbridos que ofrecen una alternativa a los complejos y computacionalmente costosos modelos de aprendizaje automático y a los simples pero poco precisos modelos analíticos aunando las ventajas de ambos. En sus trabajos usan una etapa de detección basada en las propiedades analíticas de una caída y una posterior etapa de clasificación o decisión basada en modelos más complejos obteniendo "Mejores resultados que con el método de cotas simple (...) y una fuerte reducción del coste computacional respecto al uso del modelo HMM únicamente"\cite[p.5]{Lim2014}.

\section{Predicción de series temporales}

El uso de redes neuronales recurrentes para la predicción de series temporales es ampliamente respondido hoy en dia \todo{citar estudios}. Basaremos la etapa de clasificación en la capacidad de predecir episodios conocidos de este tipo de redes.

\section{GRU vs LSTM para predicción de series}\label{sa_rnn}

Respecto al uso de LSTM para detectar anomalías en series temporales, Wang \cite{Wang2020} usa LSTM para identificar anomalías en la señal de un motor (aunque el usa el error de recostrucción de la descomposición y recomposición wavelet de la señal como entrada a una triple red LSTM, nuestro enfoque es el contrario: usar LSTM a modo de transformada wavelet y luego comparar el error de recomposición).

in\cite{Qin2019} estudia el comportamiento de varias redes recurrentes para la predicción de la saturación de oxígeno en el agua y obtiene que las GRU son las que mejores predicciones (menos error) obtienen (Sobre LSTM e incluso RNN bidireccional). Kofi \cite{Koffi2020} compara LSTM y GRU para predecir mercado de valores con topologías muy variadas (número de celdas, de capas, stateless o no) y encuentra que GRU tiene mejor tasa de aciertos teniendo en cuenta el coste computacional (y muchas veces sin tenerlo) y que no siempre tener dos capas de RNN da mejores resultados.

Alicado al problema de la predicción de la actividad humana, Li\cite{Li2019} usa redes con varias capas LSTM bidireccionales usando los datos obtenidos por un acelerómetro y un radar con un dispositivo de captura en la muñeca mientras consigue resultados comparables a las mejores técnicas de clasificación existentes.

\section{Pruning y técnicas para reducir complejidad de modelos}\label{sa_optimizacion}

A pesar de la creciente capacidad de cálculo de los dispositivos llevables, será necesario optimizar el rendimiento del modelo con el fin de conseguir la mayor capacidad de predicción por Hz del sistema. Para ello dispondremos de dos técnicas: \textit{Pruning} y \textit{Cuantificación}.

\subsection{Pruning}
Un modelo de redes neuronales se puede entender como una matriz de pesos que indican la contribción de cada nodo o neurona al resultado final. Es normal que en esa red haya nodos que aporten más información que otros. Este efecto no es necesariamente negativo, la regularización o normalización L1 busca precisamente este efecto para reducir el sobreajuste de la red a los datos de entrenamiento.

Esta diferencia de aportación nos permite realizar una compresión del modelo. Si aceptamos sufrir unas ciertas pérdidas en la capacidad predictiva del mismo, podemos eliminar los nodos de menor peso reduciendo el tamaño de la red y su consumo de memoria. Esta reducción permite además acelerar el modelo y adaptarlo al uso en sistemas embebidos.\todo{Buscar bibliografía formal para: https://www.machinecurve.com/index.php/2020/09/23/tensorflow-model-optimization-an-introduction-to-pruning/}

\todo{Este artículo cita referencias y es ligeramente más técnico.
https://www.machinecurve.com/index.php/2020/09/29/tensorflow-pruning-schedules-constantsparsity-and-polynomialdecay/}

Una de las técnicas más sencillas para realizar esta eliminación de nodos consiste en definir una aportación mínima objetivo y comparar la magnitud de la aportación individual al resultado final, eliminando aquellos que no alcanzan el nivel $\lambda$ deseado
\[
\hat{w_i} = \left\{ \begin{matrix} w_i & :\mbox{si} |w_i|>\lambda\\
  0 & :\mbox{si} |w_i|\leq\lambda\end{matrix} \right.
\]

Esta compresión se puede hacer sobre un modelo ya entrenado (sendo recomendable realizar un nuevo proceso de entrenamiento para balancear los pesos de los nodos existentes y reducir el error introducido) o realizarlo durante el entrenamiento como si de una regularización más se tratara.

En el caso de TensorFlow, el entorno usado para este proyecto, se permite realizar esta regularización durante el entrenamiento de dos formas:

\begin{itemize}
\item Eliminando nodos de forma lineal, o a tasa constante
\item Eliminando nodos de forma polinomial, o a tasa creciente
\end{itemize}

\subsection{Cuantificación}
\warn{citas necesarias}
Entendemos por cuantificación al proceso de discretizar un grupo de valores. En este caso hace referencia precisamente al descenso en la precisión numérica usada para representar los pesos de nodos y valores de entrada y salida de una red neuronal.

La cuantificación de los valores de entrada y salida a la red consiste en reducir el peso de estos valores. Normalmente se normaliza la entrada y se convierte en valores de tipo entero. Tras el paso por la red, a la salida se aplicac la operación inversa para escalar la salida al rango de valores esperado.

Por su parte la cuantificación de la red neuronal realiza una operación similar con los pesos de los enlaces de las neuronas. Reduce la precisión de las representaciones de estos valores consiguiendo un ahorro en recursos y una mejora en tiempo de cálculo al eliminar el requisito de usar unidades de coma flotante en sistemas embebidos.


\end{document}


\chapter{Objetivos}\label{chap:objetivos}
%\subfile{seccion/40-Objetivos}

\chapter{Identificación de requisitos}\label{chap:requisitos}
%\subfile{seccion/50-requisitos}

\chapter{Descripción de la herramienta desarrollada}\label{chap:descripcion}
%\subfile{seccion/60-descripcion-software}

\chapter{Evaluación}\label{chap:eval}
%\subfile{seccion/70-evaluacion}

\chapter{Conclusiones y Trabajo Futuro}\label{chap:conclusiones}
%\subfile{seccion/80-conclusiones}


%\bibliographystyle{unsrt}
%\bibliographystyle{apalike}
\newpage %necesario? parece que con apacite si
\bibliographystyle{apacite} % usa apacite y parece no funcionar, pero es el bueno
\bibliography{tfm}


\appendix

\chapter{Captura de datos y construcción del dataset}\label{app:dataset}
\subfile{seccion/app-accelcapture}

\chapter{Plataforma Hardware}\label{app:plataforma}
\subfile{seccion/app-fossil}

\chapter{Artículo}
%\includepdf[pages=-]{articulo.pdf} %descomentar cuando esté el artículo terminado.

% ELIMINAR EN VERSIÓN FINAL
\listoftodos
\end{document}
