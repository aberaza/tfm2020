\documentclass[11pt,a4paper,spanish]{book}
\usepackage{unir}
%\usepackage{estilo_unir}
\usepackage[T1]{fontenc}

% USe Helvetica font (similar to arial)
\usepackage{helvet}
\renewcommand{\familydefault}{\sfdefault}

% Mejorar el rendido de texto: ligaturas, microespacioes, etc
\usepackage{microtype}

%para gráficos
\usepackage{standalone}
\usepackage{tikz}
%\usepackage{import}
\usepackage{pgf}
\usepackage{pgfplots}
\pgfplotsset{compat=1.16}
\graphicspath{{img/}{../img/}} % Paths por defecto para las imagenes
\usepackage{smartdiagram}
\usepackage{definitions}

%tablas más presentables
\usepackage{booktabs}
\usepackage{graphicx}
\usepackage{rotating}
% Siguientes 3 para imágenes compuestas
\usepackage{float}
\usepackage[caption = false]{subfig}
%\usepackage[export]{adjustbox}

% para usar \currenttime
\usepackage{datetime}

\usepackage{amsmath}
\DeclareMathOperator*{\argmax}{arg\,max}
\DeclareMathOperator*{\softmax}{\textup{softmax}}


% Fix warning \headheigt is tochangeso small
\setlength{\headheight}{14pt}%

% Permitir modularizar el trabajo
\usepackage{subfiles} % recomiendan cargarlo al final del preambulo

%---------------------------
%título del trabajo y autor
%---------------------------
\title{Detección de caídas con Wearables y GRU}
\author{Aritz Beraza Garayalde}
\date{\currenttime}
\director{Sonia Valladares Rodriguez}
\nombreciudad{Paris}

%---------------------------
%margenes
%---------------------------
%\usepackage[margin=1.9cm]{geometry}
%---------------------------
%---------------------------
%---------------------------
%--------------------------
\begin{document}
%\selectlanguage{spanish}
\renewcommand{\listfigurename}{Índice de Ilustraciones}
\renewcommand{\listtablename}{Índice de Tablas}
\renewcommand{\contentsname}{Índice de Contenidos}
\renewcommand{\figurename}{Figura}
\renewcommand{\tablename}{Tabla}

\maketitle
\frontmatter
\tableofcontents
\listoffigures
\listoftables

% Posicion original de mainmatter
\mainmatter

%\chapter{Resumen}
\section*{Resumen}
\subfile{seccion/10-abstract.es.tex}

%\chapter{Abstract}
\section*{Abstract}
\subfile{seccion/10-abstract.en.tex}


% olvidar chapter y usar section/subsection \chapter{Introducción}
\chapter{Intro}
\section{Introducción}
\subfile{seccion/20-introduccion}

\chapter{Contexto y Estado del Arte}
\subfile{seccion/30-estado-arte}

\chapter{Objetivos}
\subfile{seccion/40-Objetivos}

\chapter{Identificación de requisitos}
\subfile{seccion/50-requisitos}

\chapter{Descripción de la herramienta desarrollada}
\subfile{seccion/60-descripcion-software}

\chapter{Evaluación}
\subfile{seccion/70-evaluacion}

\chapter{Conclusiones y Trabajo Futuro}
\subfile{seccion/80-conclusiones}

%\bibliographystyle{unsrt}
\bibliographystyle{apalike}
%\bibliographystyle{apacite} % usa apacite y parece no funcionar, pero es el bueno
\bibliography{tfm}


\appendix

\chapter{Captura de datos y construcción del dataset}
\subfile{seccion/app-accelcapture}

\chapter{Plataforma Hardware}
\subfile{seccion/app-fossil}

\chapter{Artículo}
%\includepdf[pages=-]{articulo.pdf} %descomentar cuando esté el artículo terminado.

% TODO: Borrar esta sección
\newpage
%\listoftodos[Notes]
\end{document}
