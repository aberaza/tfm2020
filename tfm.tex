\documentclass[11pt,a4paper,spanish]{book}
\usepackage{unir} %\usepackage{estilo_unir}

%para gráficos
\usepackage{standalone}
\usepackage{tikz}
%\usepackage{import}
\usepackage{pgf}
\usepackage{pgfplots}
\pgfplotsset{compat=1.16}
\graphicspath{{img/}{../img/}} % Paths por defecto para las imagenes
\usepackage{smartdiagram}
\usepackage{definitions}

% \usepackage{verbatim}
% to use \begin{comment} for block comments

%tablas más presentables 
\usepackage{booktabs}
\usepackage{graphicx}
\usepackage{rotating}
% Siguientes 3 para imágenes compuestas
\usepackage{float}
% \usepackage[caption = false]{subfig}
%\usepackage[export]{adjustbox}

% para usar \currenttime
\usepackage{datetime}

%\usepackage{amsmath}% ya se define en unir.sty
\DeclareMathOperator*{\argmax}{arg\,max}
\DeclareMathOperator*{\softmax}{\textup{softmax}}


% Fix warning \headheigt is to changeso small
%\setlength{\headheight}{14pt}%

% Permitir modularizar el trabajo
\usepackage{subfiles} % recomiendan cargarlo al final del preámbulo

%---------------------------
%título del trabajo y autor
%---------------------------
\title{Detección de caídas con Dispositivos Vestibles y Redes Neuronales Recurrentes}
\author{Aritz Beraza Garayalde}
\date{\currenttime}
%\director{Sonia Valladares Rodriguez}
\director{Carlos Agüero Iglesias}
\nombreciudad{Paris}

%---------------------------
%margenes
%---------------------------
%\usepackage[margin=1.9cm]{geometry}
%---------------------------
%---------------------------
%---------------------------
%--------------------------
\begin{document}
%---------------- 
% overfull hbox : hide messages
\hbadness=15000
%\the\tolerance,
%\the\pretolerance,
%\the\hbadness,
%\the\hfuzz,
%\the\emergencystretch,
% Second test
%\pretolerance=150
\tolerance=530
%\hbadness=752
%\hfuzz0pt
%\emergencystretch=0em
% underfull vbox fixes::
%\setlength\parskip{1em plus 2pt}
%\setlength\parskip{\baselineskip}



%\selectlanguage{spanish}
\renewcommand{\listfigurename}{Índice de Ilustraciones}
\renewcommand{\listtablename}{Índice de Tablas}
\renewcommand{\contentsname}{Índice de Contenidos}
\renewcommand{\figurename}{Figura}
\renewcommand{\tablename}{Tabla}
%% Necesita hyperref
\newcommand{\fullref}[1]{\hyperref[#1]{\autoref{#1}: \textit{\nameref*{#1}}}}
\newcommand{\apartado}[1]{\hyperref[#1]{Apartado~\ref{#1}}}

\maketitle
\frontmatter
\tableofcontents
\listoffigures
\listoftables

% Posicion original de mainmatter
\mainmatter

%\chapter{Resumen}
\section*{Resumen}
%\subfile{seccion/10-abstract.es.tex}
% !TeX root = ../tfm.tex
%\documentclass[../tfm.tex]{subfiles}
%\begin{document}
En sociedades cada vez más envejecidas aumenta la necesidad de controlar de forma contínua la salud de las personas mayores. En especial de identificar las caídas, una de las principales causas de mortalidad en personas mayores. Con este objetivo existen hoy en día soluciones que sirven a este propósito, generalmente sistemas específicos que usan un sensor llevable y una unidad de cómputo independiente. Estos sistemas resultan caros, altamente intrusivos y restringidos pues sólo pueden funcionar dentro del área de alcance de la unidad de cálculo. En este trabajo se presenta un sistema de detección de caídas que funciona de forma autónoma en un reloj inteligente o smartwatch integrando en un mismo dispositivo las capacidades de captura, detección y alerta en un componente que no resulte invasivo para el usuario como es un reloj de pulsera.

{\bf Palabras Clave:} Detección de caídas, Human Activity Recognition, RNN, GRU, Bourke, Detección de anomalías

%\end{document}



%\chapter{Abstract}
\section*{Abstract}
% \subfile{seccion/10-abstract.en.tex}
\documentclass[../tfm.tex]{subfiles}

\begin{document}
\begin{otherlanguage}{english}
The need to continuously monitor elder people's health, and specially detect falls, one of the main death causes of this population, increases in an aging society. To solve this many solutions exist nowadays, but they are either based in expensive propietary one pourpose hardware which is highly obtrusive or in the capture of data through a wearable and a linked device that processes it making it less portable and not quite autonomous. This work will present a fall detection system based purely in a smartwatch, offering high precission detection in an unobtrusive and autonomous wristwatch.

{\bf Keywords:} Fall Detection, Human Activity Detection, RNN, LSTM, Outlier detection
\end{otherlanguage}
\end{document}


\chapter{Introducción}\label{chap:intro}
% !TeX root = ../tfm.tex
%! TEX root = ../tfm.tex
%\documentclass[../tfm.tex]{subfiles}
%\begin{document}
\begin{comment}
Resumen esquemático de cada una de las partes del trabajo. Leer esta sección ha de dar una idea clara de lo que se pretendía y las conclusiones a las que se han llegado y del proceso seguido. Es uno de los capítulos mas importantes
\end{comment}

\section{Motivación}\label{sec:intro:motivación}

\begin{comment}
Problema a tratar, posibles causas, relevancia del problema
\end{comment}

En un momento en el que la población mundial acelera su crecimiento, la mayoría de las sociedades occidentales adolecen de un fenómeno contradictorio: su población se envejece rápidamente. Esta tendencia es consecuencia de la continua tendencia descendente de la tasa de fertilidad pero sobretodo por el incremento en la esperanza de vida de la población. No obstante este aumento en la longevidad no implica que la población sea más saludable. Más bien al contrario.

En su página web, la Organización Mundial de la Salud recoge que los mayores de 65 años son los más propensos a sufrir caídas, y que se producen aproximadamente 37 millones de caídas que requieren atención médica y recomienda priorizar la investigación relacionada con las caídas\cite{FactsFalls2018}. En este contexto es normal que proliferen estudios y soluciones para mejorar la atención de estos colectivos de riesgo.

Esta tendencia no es nueva, el estudio e implementación de sistemas para la detección de caídas lleva décadas en desarrollo\cite{fallindex00, Chen2005, Noury2007, Bourke2006}. En la actualidad se comercializan multitud de dispositivos para la detección de caídas.

Desde sistemas de propósito específico como las alarmas de detección de caídas en el baño basadas en una correa atada a la muñeca, a sistemas multi-propósito basados en dispositivos de captura y otro de procesado por separado, pasando por complejos métodos de captura y procesado de imágenes. Todos estos sistemas adolecen o bien de restricciones geográficas (su funcionamiento queda limitado a un entorno limitado por el alcance de las cámaras o cobertura del enlace radio con la base de procesado) o bien resultan poco prácticos e incluso invasivos para un grupo de usuarios que no está acostumbrado a lidiar con la tecnología.

\section{Planteamiento del trabajo}\label{sec:intro:planteamiento}

\begin{comment}
¿cómo se puede resolver el problema qué se propone descripción de objetivos en términos generales?
\end{comment}

La rápida evolución tecnológica permite tener a nuestra disposición cada vez más dispositivos \textit{vestibles} (por \textit{wearables} del inglés, dispositivos electrónicos que se incorporan en alguna parte del cuerpo para realizar una función) con sensores conectados a Internet. Estos dispositivos mejoran generación tras generación gracias a la creciente capacidad de cálculo de los microprocesadores, la miniaturización de los sensores y la reducción de costes, lo que ha permitido generalizar y extender y popularizar su uso al grueso de la sociedad. Estos avances se aprovechan ya parcialmente en algunas soluciones al problema de la detección de caídas. Se utilizan los datos capturados por acelerómetros triaxiales y/o sistemas inerciales de una pulsera de actividad o reloj inteligente ya sea para realizar un procesado analítico de los parámetros medidos y realizar la detección de una caída \cite{fallindex00, Chen2005,Bourke2006,Kangas2008,Bagala2012} (por ejemplo buscar una aceleración fuerte y abrupta) o bien para enviar este flujo de datos a un segundo sistema\cite{Luque2014,Vilarinho2015}, para realizar la detección mediante modelos complejos con grandes requisitos computacionales\cite{Cho2014, Aziz2017b,Putra2017}. Ambas soluciones tienen sus pros y contras. La primera tiene a su favor la alta disponibilidad al aunar captura y tratamiento en la misma unidad, pero falla al usar algoritmos poco precisos, al contrario que la segunda opción que se aprovecha de la mayor potencia de un segundo centro de cómputo para mejorar la detección a costa de una mayor complejidad en el sistema que impacta negativamente en su facilidad de uso y disponibilidad. Hay que tener en cuenta que el público objetivo de esta tecnología es un segmento de población con escasos conocimientos de nuevas tecnologías y poco habituado a su uso diario.

Este trabajo propone una solución que sea simple de usar, no invasivo y al mismo tiempo ofrezca resultados comparables al estado del arte. Para conseguir estos objetivos nos basaremos en aprovechar la mejora en rendimiento de los microprocesadores de sistemas llevables (como pueden ser las pulseras de actividad o los relojes inteligentes), para el tratamiento y predicción de series temporales para realizar la captura, tratamiento y detección de caídas en la misma unidad usando algoritmos de aprendizaje automático. Esta es la principal diferencia con los sistemas ya existentes: usar un algoritmo basado en los últimos avances en tecnologías de predicción temporal con inteligencia artificial, redes recurrentes GRU (\textit{Gated Recurrent Unit} un tipo de celda para formar redes neuronales recurrentes),  y un mecanismos de atención o eventos para reducir los requisitos de capacidad de cómputo. Este trabajo aporta al campo de la detección de caídas la novedad de usar la arquitectura codificador-decodificador de la red neuronal recurrente para conseguir una una secuencia de predicción en un único paso y a su vez, gracias al proceso de reducción y expansión de la dimensionalidad realizada por las redes de codificación y decodificación y mediante un comparador de la predicción con la señal real para detectar las caídas como anomalías basándonos en la baja capacidad de generalización de las redes neuronales a los casos para los que no ha sido entrenada. Este sistema permite simplificar el clasificador a un único caso: \textit{actividad} (o anomalía)

Estas tres mejoras (detección en dos etapas, predicción de varios pasos a la vez y simplificación del clasificador) permiten ejecutar el sistema con una latencia baja en un sistema discreto, no invasivo o molesto para el usuario final, como es llevar puesto un reloj de pulsera. El objetivo final es implementar una solución para la detección de caídas con gran precisión, mejor que la de los sistemas analíticos y próxima a la del estado del arte de los sistemas basados en aprendizaje automático que se ejecute exclusivamente en un dispositivo llevable.

\section{Estructura de la memoria}\label{sec:intro:estructura}
\begin{comment}
qué hay en cada uno de los subsiguientes capítulos
\end{comment}

Los capítulos que estructuran este documento se organizan de la siguiente forma: en el capítulo \ref{chap:stateofart} realizamos una introducción y revisión del del estado del arte y literatura relacionada con la problemática abordada. Haremos especial hincapié en el tratamiento de la actividad humana (\ref{sec:arte:detección_caídas}, y los diferentes acercamientos existentes para aboradar la detección de caidas en las secciones \ref{sec:arte:modelos_analiticos},\ref{sec:arte:modelos_ml} y \ref{sec:arte:modelos_hybridos} ya que en los siguientes capítulos haremos muchas referencias a estos trabajos. Habiendo establecido las bases de la situación actual de la tecnología y soluciones disponibles presentaremos el objetivo a abordar por la aplicación y una serie de metas previas e intermedias que faciliten la evaluación de los logros conseguidos en el capítulo \ref{chap:objetivos}.   

Antes de empezar con las explicaciones del trabajo realizado, introduciremos componentes(\ref{sub:req:hardware}, plataformas (\ref{sec:req:plataforma}), entornos (\ref{sec:req:tflite}), bases de datos (\ref{sec:req:bases_datos}), modelos de detección (\ref{sec:req:modelos}) ya existentes sobre las que trabajaremos, sus necesidad, sus limitaciones y posibles alternativas, justificando siempre las decisiones en base al cumplimiento de los objetivos previamente descritos. Acto seguido nos adentraremos en el desarrollo e implementación de la solución en el capítulo \ref{chap:descripcion}. En esta parte detallaremos el trabajo realizado, el sistema, las diferentes aplicaciones que lo componen, sus arquitecturas así como detalles del modelo (\ref{eval_modelo}) utilizado, sus parámetros y funcionamiento.

Una vez explicado todo el proceso seguido para implementar toda la plataforma pasaremos a evaluar el sistema en el capítulo \ref{chap:eval}. Analizaremos los resultados obtenidos en cada una de las etapas de desarrollo, ajustaremos los diferentes parámtetros, compararemos con otras soluciones existentes y justificaremos las decisiones tomadas con dichos datos. Volveremos a revisar de forma más frugal estos resultados en el capítulo \ref{chap:conclusiones} para exponer las conclusiones obtenidas y presentar posibles líneas de trabajo futuro que por diversas razones no se han podido explorar. Al final en los apéndices detallaremos otros aspectos como las bases de datos usadas (Ap.\ref{app:dataset}) y la plataforma llevable usada (Apéndice.\ref{app:plataforma}).

%\end{document}


\chapter{Contexto y Estado del Arte}\label{chap:stateofart}
% !TeX root = ../tfm.tex
%! TEX root = ../tfm.tex

Dada la extensión del contexto trabajo previo sobre el que se asienta esta solución, esta sección se organiza en diversos bloques para facilitar su lectura y comprensión. En una primera sección analizaremos la historia y evolución de los sistemas de detección de caídas, introduciremos las principales familias y presentaremos el estado actual. Saltaremos a una nueva sección para introducir varias técnicas y modelos de aprendizaje automático que si bien están relacionadas con los modelos previamente introducidos es necesario exponer de forma aislada para poder justificar a lo largo de este desarrollo los componentes usados y decisiones tomadas. En una última sección de este capítulo trataremos la problemática de la optimización de modelos, técnicas y bibliotecas de desarrollo de modelos neuronales para sistemas embebidos.


\section{Detección de actividad humana y caídas}\label{sec:arte:detección_caídas}

Históricamente, los modelos de predicción de caídas están íntimamente ligados al estudio del reconocimiento de actividades humanas (\textit{HAR} del inglés \textit{Human Activity Recognition}): discernir a partir de los datos de diferentes sensores qué acción (\textit{AVC} por \textit{Actividad Vida Cotidiana}, o \textit{ADL} por sus siglas en inglés) estaba realizando el individuo. Según \citeA[p.10692]{Ozdemir2014} una caída o golpe no es más que otra actividad y deben considerarse una parte de este problema ya que "son eventos que suceden de forma involuntaria e inesperada durante la realización de las mismas [actividades]".

Los sistemas primigenios de detección de caídas se basaban en la capacidad del usuario de activar un mecanismo que a su vez activase algún tipo de alerta. Consciente de las limitaciones de estos sistemas, especialmente dada la naturaleza incapacitante para cualquier acción tras el impacto, \citeA{Williams1998} presenta un sistema automático, sin intervención del usuario, de detección de caídas graves. Con un sensor de impacto y otro de posición y un algoritmo analítico basado en cotas de los parámetros medidos, siendo uno de los primeros modelos analíticos de detección de actividad humana. Veremos en los siguientes puntos (\ref{sec:arte:modelos_analiticos},\ref{sec:arte:modelos_ml}) la evolución de estos sistemas.

Con la mejora de las capacidades de estos sistemas, empiezan a proliferar las implementaciones basadas en ellos. En \citeA{Anita2020} se recopilan los resultados de diferentes estudios que usan sistemas llevables para la detección de caídas,  las técnicas y modelos usadas a la vez que se establece unas directrices que debería cumplir todo sistema de detección de caídas:
\begin{itemize}
  \item No ser invasivo
  \item No restringir la movilidad del sujeto
  \item Tener baja latencia en la detección
  \item Ser capaz de distinguir caídas de otras actividades humanas cotidianas
\end{itemize}
\citeA{Anita2020} y otros autores \cite{Musci2020,Lim2014} introducen varias clasificaciones de estos sistemas. Si bien hay variaciones según los textos la mayoría coinciden en establecer una segregación según el tipo de algoritmo utilizado en \cita[{\citeNP[p.1]{Lim2014}}]{dos acercamientos principales, los basados en cotas o límites y los basados en aprendizaje automático}. También se definen diferentes sistemas según el tipo y situación de los sensores usados, aunque aquí es donde más variabilidad hay según el artículo, resulta de interés la categoría de los dispositivos \textit{llevables} definidos como \cita[{\citeNP[p.3]{Anita2020}}]{aquellos sistemas en los que los sensores usados para la detección se encuentran embebidos en un dispositivo que debe llevar puesto el sujeto, como por ejemplo en una pulsera}. Tanto para entender la evolución como para una mayor comprensión del desarrollo de este trabajo introduciremos las dos grandes familias de algoritmos de detección usadas.

\subsection{Modelos Analíticos: cálculos basados en cotas}\label{sec:arte:modelos_analiticos}

Los primeros acercamientos\cite{Williams1998,fallindex00,Chen2005,Kangas2008} al problema de la detección de caídas utilizaron modelos matemáticos basados en el análisis de diferentes parámetros físicos como la aceleración vertical, módulo de la aceleración y postura. A partir de todos estos parámetros, se establece una métrica o \textit{índice} que se calcula periódicamente. Según el algoritmo, se compara o analiza uno o varios índices con uno o varios niveles predeterminados, que de superarse indicaría que se ha producido una caída\cite{Luque2014}. En \citeA[p.~286-289]{Kangas2008} se presentan varios índices y algoritmos. Como observa \citeA{Bagala2012}, estos métodos puedan conseguir buenos resultados aplicados a pruebas de laboratorio o experimentos controlados, pero no en experimentos de uso real, en parte debido a un sobreajuste de la calibración de los valores al experimento, proceso que a su vez incremente la dificultad de configuración por parte de usuarios no expertos\cite{Kangas2008, Luque2014}. En general estos métodos contraponen sensibilidad a especificidad o a la inversa, como puede deducirse de los resultados de \citeA{Chen2005,Bourke2006,Kangas2008,Vilarinho2015} y verifica \citeA{Anita2020}. A pesar de ello, estos métodos siguen siendo ampliamente utilizados hoy en día por la baja complejidad computacional que requieren, lo que permite ser integrados en pequeños dispositivos poco obtrusivos y con bajas latencias, cumpliendo tres de los cuatro requisitos antes mencionados. Así mismo cabe destacar que ya en estos primeros estudios se recomienda evitar el uso de sensores en los brazos ya que \cita[{\citeNP[p.~3552]{Chen2005}}]{el frecuente y marcado movimiento del brazo durante la actividades cotidiana dificulta el uso de las fuerzas de aceleración medidas en el}.

Con la madurez, las lineas de investigación se centran en el impacto de la posición de los sensores sobre los resultados. Tal es el caso de \citeA{Bagala2012, Vilarinho2015, Luque2014}. Usan \textit{smartphones} o pulseras de actividad en diferentes posiciones para la captura de la aceleración con resultados muy dispares según el tipo de caída o el tipo de experimento. En el caso de \citeA{Vilarinho2015} a pesar de introducir las pulseras como instrumento de captura, delega en un equipo externo (en su caso un teléfono) el análisis y detección de caídas, de manera similar a los experimentos conducidos por \citeA{Luque2014} quien añade la posibilidad de realizar el cómputo en un servidor conectado a Internet.

A pesar de sus inconvenientes, su simplicidad y suficiente capacidad de detección permiten a estos algoritmos resultados similares o en algunos casos mejores a las de las soluciones comerciales existentes \cite{Kangas2008}. La evolución de estos índices sigue dos caminos aparentemente contradictorios: la simplificación o generalización y la mejora de la capacidad detectora del índice (entendemos por simplificación o generalización aquellos intentos por buscar algoritmos con mayor capacidad de generalización a la hora de detectar caídas en contraposición al uso de índices centrados en resolver un subconjunto concreto de estos eventos). Estos métodos se basan en establecer una métrica o \textit{índice} y buscar un  Partiendo del método usado por\citeA{Williams1998}, basada en establecer un \textit{índice de actividad} para el individuo a partir de su historial de caídas, género, edad, medicación, capacidades cognitivas, motoras y otros muchos parámetros, y un \textit{índice de predicción de caídas}. Este último parámetro es un contador del total de veces que se ha superado el índice de actividad en un determinado periodo que de rebasar un determinado nivel indicaría la posibilidad de que el sujeto hubiese sufrido una caída. Una última etapa en el algoritmo mide la actividad posterior al posible evento y de ser inferior a un segundo nivel, determinar que se ha producido una caída. 

Como veremos más adelante en el trabajo, cabe recalcar el acercamiento propuesto por \citeA{Bourke2006}, basado únicamente en el uso del módulo del vector aceleración capturado por un acelerómetro triaxial. En esta obra, \citeauthorNP{Bourke2006} determinan que la curva descrita por la evolución del módulo del vector aceleración (consistente en un valle al inicio de la caída sucedido de un pico en el momento del impacto) permite la segregación de diferentes actividades y de caídas estableciendo unos valores límite para estos valores valle y pico) este método tiene por definición una sensibilidad del 100\% aunque una muy baja especificidad \cite{Aziz2017,Bagala2012}. Posteriormente en \citeA{Bourke2008, Bourke2010} ajusta el algoritmo para añadir una etapa posterior de medición de actividad y postura consiguiendo métricas de precisión, sensibilidad y especificidad entre las mejores en esta categoría, añadiendo por contra cierta latencia en la alarma.
% ####################################################################

\iffalse

Sobre cálculos basados en cotas, tenemos \cite{fallindex00}, Kangas\cite{Kangas2008} lo compara con el uso ya sea de aceleración vertical o modulo de la aceleración y de nuevo obtiene resultados dispares según el tipo de caída simulada.

\todohide{
1- Monitorizado básico (usar módulo del vector aceleración y un threshold X )
2- Fall Index (Yoshida,  T.;  Mizuno,  F.;  Hayasaka,  T.;  Tsubota,  K.;  Wada,  S.;  Yamaguchi,  T.  A  Wearable  Computer System for a Detection and Prevention of Elderly Users from Falling. In Proceedings of  the  12th  International  Conference  on  Biomedical  and  Medical  Engineering  (ICBME),  Singapore, Singapore, 7–10 December 2005.  )
3- PerFallD (Dai, J.; Bai, X.; Yang, Z.; Shen, Z.; Xuan, D. PerFallD: A Pervasive Fall Detection System using Mobile   Phones.   In   Proceedings   of   the   8th   IEEE   International   Conference   on   Pervasive   Computing  and  Communications  Workshops  (PERCOM  Workshops),  Mannheim,  Germany,    29 March–2 April 2010; pp. 292–297) (usa también un giroscopio)
4- iFall (Sposaro,  F.;  Tyson,  G.  IFall:  An  Android  Application  for  Fall  Monitoring  and  Response.    In Proceedings of the Annual International Conference of the IEEE Engineering in Medicine and Biology Society (EMBC 2009), Minneapolis, MN, USA, 2–6 September 2009; pp. 6119–6122. )}

\fi


\subsection{Modelos de Aprendizaje Automático}\label{sec:arte:modelos_ml}

Con el aumento en la capacidad de cálculo sumado al auge en investigación relacionado con la inteligencia artificial, aparecen los primeros estudios que aplican técnicas como \textit{K-Nearest Neighbours}, LSM (\textit{Least Square Method} o \textit{Método de Mínimos Cuadrados}), SVM, Detección Bayesiana, DTW y Redes Neuronales aplicadas a la detección de caídas. \citeA{Yang2010} analiza los resultados de aplicar \textit{SVM}, \textit{Redes Neuronales} y \textit{Naive Bayes} entre otros, con unos resultados muy prometedores. Especialmente para el caso de Naive Bayes con el que consigue una precisión del 88\%. Varios trabajos \cite{Luque2014,ReyesOrtiz2014,tfall,Ozdemir2014} prosiguen por esta vía estudiando en detalle algoritmos basados en Modelos Ocultos de Markov, SVM y Bayes, K-NN, llegando a obtener resultados cercanos al 100\% de precisión usando K-NN \cite{Ozdemir2014} y destacan la capacidad para generalizar el aprendizaje a casos no entrenados de SVM\cite{tfall, Aziz2017b}.

En los últimos años parecen destacar los estudios que usan algún tipo de red neuronal \cite{Musci2020,Shi2020,Casilari2020,Liu2020} para realizar la clasificación de actividades humanas y detección de caídas. Por ejemplo, \citeA{Shi2020} describe un método basado en acelerómetro triaxial llevado en la cintura para predecir caídas, tratando una caída como una actividad humana más y usando redes convolucionales; en \citeA[p.17]{Casilari2020} usa una red de 4 capas convolucionales para el mismo fin, y compara el rendimiento sobre multitud de bases de datos obteniendo que si bien se puede optimizar el rendimiento para un conjunto en concreto logrando resultados de precisión y especificidad del orden del 100\%, es muy costoso generar un modelo que generalice a todos los conjuntos de datos. Resultado verificado por \citeA{Luque2014,Aziz2017,Aziz2017b}.

Al ser una rama tan prolífica en resultados, aparecen metaestudios con el objetivo de aunar criterios y comparar resultados de los diferentes métodos propuestos como son las obras de \citeA{Anita2020} y \citeA{Aziz2017}. Estos estudios vuelven a poner de manifiesto la superioridad de los métodos basados en aprendizaje automático (con sensibilidades y especificidades superiores al 90\% en todos los algoritmos probados, frente al compromiso entre sensibilidad y especificidad que se da en los algoritmos analíticos, especialmente en los más simples como Bourke \cite{Aziz2017}.

De igual forma que pasó con sus hermanos los algoritmos basados en modelos analíticos\ref{sec:arte:modelos_analiticos} al llegar a la madurez los estudios empieza a poner énfasis en otros parámetros que impactan la efectividad de estos sistemas. Se centran en conocer la fisionomía de una caída para así poder mejorar los modelos. En \cite[p.~6]{Musci2020} se llega a la conclusión de que el uso de la información de la aceleración únicamente es una fuente de información suficientemente fiable para obtener métricas de detección satisfactorias usando una red neuronal recurrente. Analizando la duración y componentes de la señal de la aceleración durante una caída, \citeA[p.~16]{Casilari2020} llega a la conclusión que a partir de 5 segundos no hay ningún beneficio extra en aumentar el tamaño de la ventana. En \cite[p.~59]{Hassan2019} usan una ventana de 1s consiguiendo mejorar los clasificadores existentes (modelos basados en recurrentes) con un modelo basado en una red CNN-LSTM. Resultado que coincide con \cite[p.~2]{Liu2018} que acota la duración de una caída entre los 0,4 y 0,8 segundos, demostrando que una frecuencia de muestreo de 21,3Hz es suficiente cuando se utilizan modelos de aprendizaje automático. \citeA{Li2019} usa redes LSTM bidireccionales para realizar el análisis y captura contínua de actividad humana eliminando la necesidad de uso de una ventana y tratando los datos de actividad con técnicas de procesado de series temporales. Usando los datos obtenidos por un acelerómetro situado en la muñeca y un radar con un dispositivo de captura  mientras consigue mejorar los resultados obtenidos con SVM en similares cirtunstancias\cite[p.~9]{Li2019}. 


\subsection{Modelos basados en eventos o híbridos}\label{sec:arte:modelos_hybridos}

Si bien los algoritmos basados en aprendizaje automático consiguen estimadores muy precisos, adolecen de requerir sistemas con gran capacidad de cómputo y de grandes consumos energéticos, que son precisamente las grandes ventajas de los modelos analíticos. El uso de un sistema combinado es aparece descrito por \citeA[p.~9]{Aziz2017b} que recomienda usar una combinación de métodos basados en cotas, como generador de eventos, junto con técnicas de aprendizaje automático para obtener mejores y más eficientes estimadores. Sin embargo podemos observar un primer sistema de hibridación en \citeA{Lim2014}, quiienes proponen un método basado en una primera etapa de pre-detección usando un algoritmo basado en cotas para activar una segunda etapa. Esta segunda fase, usa la postura o ángulo del cuerpo para mediante un modelo HMM decidir si se trata de una caída o no. Llega a la conclusión de que obtiene \cita[{\citeNP[p.~5]{Lim2014}}]{Mejores resultados que con el método de cotas simple (\dots) y una fuerte reducción del coste computacional respecto al uso del modelo HMM únicamente}.

\citeA{Putra2017} refina la técnica usando un autómata de estados finitos para determinar a partir de de los resultados de varios índices analíticos si se debe activar el modelo basado en aprendizaje automático (frente al uso de enventanado fijo con o sin solapamiento (\textit{FOSW} y \textit{FNSW} respectivamente). Compara los resultados usando K-NN y SVM entre otros y obtiene, al igual que \citeauthorNP{Lim2014} que \cita[{\citeNP[p.~15]{Lim2014}}]{usando EvenT-ML arroja mejores resultados de precisión y F-score que usando FOSW y FNSW (...). Con la ventaja adicional que de requerir considerablemente menos capacidad de cálculo}. Confirma así la viabilidad de la técnica y como puede ayudar a mejorar la precisión de los sistemas basados puramente en aprendizaje automático.

\section{Predicción de series temporales: RNN}

De entre todas la técnicas de aprendizaje automático disponibles, este trabajo se centra en el uso de redes neuronales recurrentes. Introducidas por primera ven en el trabajo de \citeA{RNN1986} y designan toda una familia de topologías y celdas que presentan una capacidad de memoria temporal gracias al uso de la retroalimentación. Hoy en día las celdas más usadas son las \textit{LSTM} (\textit{Long Short-Term Memory}, celdas recurrentes que controlan el estado o memoria y su transmisión mediante unas entradas y salidas especiales llamadas puertas, \citeA{LSTM1997}) y una versión simplificada de estas: las GRU (\textit{Gated Recurrent Unit}, al igual que las LSTM tienen la capacidad de controlar la transmisión del estado o memoria a la etapa siguiente, pero con un número inferior de parámetros). Rápidamente se extiende su uso con muy buenos resultados en tratamiento del habla, traducción, reconocimiento de caracteres, segmentación, y en general tareas de tratamiento o predicción de series temporales.\todo{citar wikipedia o poner ejemplos de cada trabajo, dispo en wikipedia también, página inglesa claro}. Estas topologías de celda arrojan resultados comparables, incluso a veces con GRU superando a LSTM a pesar de su mayor simplicidad \cite{Chung2014,Su2018}.

Enseguida aparecen nuevas topologías de redes para mejorar o solventar algunos de los problemas. Por ejemplo, as topologías bidreccionales que aunan dos redes que recorren en sentidos contrarios las secuencias temporales que alimentan el modelo dando al modelo la capacidad de conocer tanto el contexto previo como el futuro \cite{Schuster1997}. Esta arquitectura además demuestra su superioridad en áreas donde se requiera mantener memoria a muy largo plazo, como por ejemplo la traducción de textos largos y tareas complejas de procesado del lenguaje \cite{Su2018}. Esta mayor capacidad de memoria temporal la aprovecha \citeA{Zhao2017} para mejorar la capacidad de detección de actividad humana en un 4,78\% respecto a redes recurrentes ordinarias.

%\subsection{Codificador-Decodificador}

\citeauthor{Cho2014} presentan la estructura \textit{codificador-decodificador} como: \cita[{\citeNP[p.~9]{Cho2014}}]{Una arquitectura de redes neuronales, capaz de comprender la relación entre dos secuencias de longitud arbitraria de dos conjuntos diferentes y evaluar la similitud de ambas en términos de probabilidad condicional o de proveer esa segunda secuencia a partir de a primera.}. Esta arquitectura arroja muy buenos resultados en los campos de traducción y generación de textos \cite{Cho2014,cho2014b,Serban2017,Tran2017} pero rápidamente pasa a aplicarse al campo de la detección de anomalías o problemas usando una medida de distancia entre la salida del modelo y la serie real para discernir\cite{Wang2018,malhotra2016,Park2018,Wang2020}. Nos interesa destacar los trabajo de \citeA{malhotra2016, Park2018} que mediante una red codificador-decodificador para regenerar la señal de entrada, de manera similar a un autocodificador, para posteriormente comparando la salida con la entrada calculando distancia euclídea poder discriminar anomalías e incluso predecir fallos en maquinaria mediante el análisis de señales temporales como el sonido.

  Esta misma arquitectura tiene una segunda virtud a explotar: la capacidad de realizar predicciones o reconstrucciones de varios pasos o muestras\cite{Peng2018}. La posibilidad de generar secuencias completas no solo simplifica la formulación de predicciones de varios pasos \cite{Kao2020}, sino que además muestra una excelente capacidad para encontrar correlaciones no lineales y realizar predicciones mejores que los métodos existentes hasta la fecha\cite{Peng2018,Du2019,Kao2020}. La combinación de la generación de secuencias de varios pasos a la vez así como sus aplicaciones en la detección de anomalías jugarán un papel clave en este trabajo.



\iffalse


\section{GRU vs LSTM para predicción de series}\label{sa_rnn}
\warn{corto y título de poco agrado: se esperaría una conclusión}

Respecto al uso de LSTM para detectar anomalías en series temporales, Wang \cite{Wang2020} \todo{sólo un artículo usando LSTM's?} usa LSTM para identificar anomalías en la señal de un motor (aunque el usa el error de recostrucción de la descomposición y recomposición wavelet de la señal como entrada a una triple red LSTM, nuestro enfoque es el contrario: usar LSTM a modo de transformada wavelet y luego comparar el error de recomposición).

En\cite{Qin2019} estudia el comportamiento de varias redes recurrentes para la predicción de la saturación de oxígeno en el agua y obtiene que las GRU son las que mejores predicciones (menos error) obtienen (Sobre LSTM e incluso RNN bidireccional). Kofi \cite{Koffi2020} compara LSTM y GRU para predecir mercado de valores con topologías muy variadas (número de celdas, de capas, stateless o no) y encuentra que GRU tiene mejor tasa de aciertos teniendo en cuenta el coste computacional (y muchas veces sin tenerlo) y que no siempre tener dos capas de RNN da mejores resultados.
\fi


 
\chapter{Objetivos}\label{chap:objetivos}
\documentclass[../tfm.tex]{subfiles}

\begin{document}
\info{Puente entre el estudio y la contribución.}
\warn{
Debe contener:
\begin{enumerate}
  \item objetivo general
  \item objetivo específico
  \item metodología de trabajo
\end{enumerate}
}

Los objetivos deben ser \textit{SMART}
\begin{itemize}
  \item Specificl (objetivo claro)
  \item Measurable (se pueda medir el éxito o fracaso)
  \item Attainable (viable su conecución con el tiempo y recursos disponibles)
  \item Relevant (que tenga un impacto demostrable)
  \item Time-related (que se pueda realizar en un tiempo determinado)
\end{itemize}


\section{Objetivo General}
\infor{
Un poco la descripción a grandes rasgos de qué se espera explicado al público general.

Ejemplo: Mejorar el servicio de audio gruía de un museo con una guía interactiva por voz valorada positivamente con un 4/5 al menos.
}

Implementar un sistema de detección de caídas que se ejecute en un smartwatch con una tasa de detección comparable a los sitemas basados en "thresholds" del acelerómetro y mucho menor ratio de falsos positivos.

\section{Objetivos Específicos}
\begin{itemize}
  \item Desarrollar un conjunto de datos para el experimento
  \item Estudiar el comportamiento y características de las señales temporales del acelerómetro.
  \item Identificar y establecer una magnitud derivada de la lectura del acelerómetro que permita su posterior trabajo
  \item Desarrollar un sistema predictor basado en los puntos anteriores
  \item cuantificar el grado de satisfacción con el sistema obtenido y comparar con los sistemas existentes
\end{itemize}


Conjunto de objetivos más específicos alcanzables por separado. suelen sor los diferentes pasos a seguir para conseguir el objetivo general. Han de ser smart, los verbos deberian ser:     • Analizar
Calcular
Clasificar
Comparar
Conocer
Cuantificar
Desarrollar
Describir
Descubrir
Determinar
Establecer
Explorar
Identificar
Indagar
Medir
Sintetizar
Verificar

\section{Metodología de trabajo}

Debe describir los pasos que se van a dar, el por qué de cada paso, instrumentos a utuilizar y cómo se van a analizar los resultados.

\begin{enumerate}
  \item Generar un nuevo dataset \footnote{Se implementa una app para ello, y se realiza un procesado de los datos obtenidos}
  \begin{enumerate}
    \item Los existentes se basan en detección de actividades
    \item Los pocos basados en detección de caidas se restringen a un subconjunt de caidas muy específico y muchas veces con posiciones del elemento de detección diferente de la usada en el estudio
  \end{enumerate}

  2- Estudiar el dataset obtenido
  a- comprender mejor el comportamiento de las señales capturadas
  b- buscar parámetros importantes para llevar a cabo la detección

  \item Implementar un sistema basado en LSTM y detección de anomalías usando errores de predicción. Mecanismo atencional basado en potencia de señal de entrada?
  \begin{itemize}
    \item Implementar una red LSTM FORWARD, y/o una FORWARD-BACKWARD, Asi como un sistema basado en threshold (usar GRU con RELU para mejorar la eficiencia de cómputo)
    \item Una vez entrenados los modelos con información de actividad ordinaria:
    \begin{itemize}
      \item Implementar una app que o bien en tiempo real o cuando detecte un cierto nivel de actividad lance los modelos de predicción
      \item Usar un sistema que comparando la señal real y la predicha decida si es una "anomalía" o no, lo más básico un RMS del error de predicción.
    \end{itemize}
  \end{itemize}

  \item Comparar los resultados con el estado del arte
  \begin{itemize}
    \item Entrenar el sistema usando al menos un dataset que incluya caídas.
    \item Compararar con al menos métodos de cota fall index y modulo aceleración.
    \item Comparar con estudios previos que usen el mismo dataset.
    \item Analizar los resultados con el dataset propio.
  \end{itemize}

  \item Prueba en uso real. (Si da tiempo)
  \begin{itemize}
    \item Entrenar el sistema con dataset creado.
    \item Usar en interno
    \item Probar en caídas simuladas
    \item Si es posible conseguir ejemplos de caídas reales (app debe enviar todos los eventos registrados como caídas e indicar si han sido confirmados, rechazados o ignorados)
  \end{itemize}


\end{enumerate}


\end{document}


\chapter{Identificación de requisitos}\label{chap:requisitos}
% !TeX root = ../tfm.tex
%! TEX root = ../tfm.tex

\iffalse
info{Indicar el trabajo previo realizado para guiar el desarrollo del software.
Debe identificar adecuadamente el problema a tratar, contexto adecuado de uso y funcionamiento de la aplicación. Idealmente se debería realizar con expertos en la materia a tratar.}

todo{referenciar los requisitos de un sistema de detección de caídas, en detalle si es posible}
\fi

En este capítulo presentaremos al lector una serie de desarrollos previos sobres que serán utilizados o sobre los que se asentará el sistema de detección de caídas. Esta selección es fruto de las decisiones tomadas tras una etapa de estudio del estado de la tecnología en el área de detección de actividad humana y análisis de diferentes opciones de materiales, servicios, técnicas, modelos y datos disponibles. Todos los desarrollos aquí presentes fueron seleccionados a priori, antes de iniciar cualquier otra etapa de desarrollo, basándonos únicamente en los requisitos y necesidades del sistema. En los capítulos sucesivos se hará mención y justificación del uso o ausencia en la implementación final de todas estas herramientas.

\section{Plataforma y Entorno de Desarrollo}\label{sec:req:plataforma}

Tras adquirir información de entorno y con el fin de acotar y restringir las opciones disponibles de cara a la elaboración del algoritmo a emplear es necesario determinar sobre qué plataforma o dispositivo va a ejecutarse la aplicación. Esta decisión impondrá nuevas restricciones tales como tipos de fuentes de información/sensores disponibles, frecuencias de muestreo, precisión y magnitud, posición del dispositivo, entorno de programación a utilizar.

Entre los requisitos que debe cumplir el sistema destacamos principalmente:

\begin{itemize}
  \item Autónomo: Integrar en una única unidad la capacidad de adquisición, cómputo y alerta.
  \item No ser obtrusivo: El dispositivo no debe interferir o limitar la actividad cotidiana del usuario final.
  \item Programable: El dispositivo debe permitir la ejecución de código de terceros, no debe ser una plataforma cerrada.
  \item Capaz de ejecutar modelos de redes neuronales/aprendizaje automático.
\end{itemize}

Se tendrán en cuenta además de cara al cribado final otros criterios como el coste y la preferencia del autor sobre los entornos de programación a usar así como la adaptación al público objetivo: personas mayores de 65 años. Este grupo de población es más reacio al uso de sistemas computerizados por no estar familiarizados con su uso. Así se buscará ofrecer un sistema que reemplace un objeto de uso cotidiano, con una experiencia de usuario equiparable a la de este objeto y que evite cualquier tipo de interacción que no existiera previamente para la manipulación del objeto original.

\subsection{Dispositivo físico}\label{sub:req:hardware}

Analizamos en primera instancia los

\subsubsection{Tipos de dispositivos analizados}

\paragraph{Teléfono} o \textit{Smartphone}. Un aparato que se ha vuelto ubicuo en nuestra sociedad. Con gran capacidad de cálculo y multitud de sensores. Según la firma \textit{Kantar}(\url{https://www.kantarworldpanel.com/global/smartphone-os-market-share/}) el 99,99\% de los teléfonos vendidos en españa en los 9 primeros meses de 2020 fueron smartphones con IOS o Android y por ende fácilmente programables y con soporte para ejecutar múltiples entornos de modelado de sistemas de aprendizaje automático. Sin embargo adolecen de una complejidad extra: no es un dispositivo que se lleve vestido, y es fácil que el usuario final no lo lleve siempre consigo. Además, en los grupos de poblaciones de edad muy avanzada donde su uso no está generalizado, se convierte en un elemento ajeno y que puede generar rechazo o olvido en su uso.

\paragraph{Chalecos, cinturones y bandas} y otras prendas de vestir han sido usadas con anterioridad \cite{Liu2020, MobiFall} para situar los sensores. La falta de soluciones comerciales disponibles obligaría a desarrollar una plataforma personalizada y por tanto adaptada a los requisitos. Sin embargo este acercamiento se considera altamente intrusivo tanto por limitar al usuario en su vestimenta así como por servir únicamente cuando el usuario lleve puesto el complemento de ropa especificado. Además, la manipulación del mismo no es evidente, pues requiere de una manipulación muy diferente a la prenda de vestir original.

\paragraph{Pulseras de actividad y Relojes inteligentes}. Hace tiempo que las podómetros evolucionaron gracias a la mejora en la miniaturización de sensores, baterías y microprocesadores en sistemas avanzados de medida no ya de la cantidad de pasos sino de otros muchos parámetros como pulso, oxígeno en sangre, temperatura corporal y orientación. Por su parte, el paso de los relojes computadoras interconectadas y programables se ha realizado más recientemente. Actualmente puede resultar difícil distinguir en la oferta disponible entre un \textit{smartwarch} (reloj inteligente) y una pulsera. Muchas veces es la propia definición del fabricante la que especifica su tipo, por criterios tan dispares como la producción tradicional de la empresa (así pues, Fossil llamará a sus productos relojes inteligentes, mientras otro dispositivo de similares características y formato de la marca FitBit será llamado pulsera por la marca). Al existir tal variedad de oferta es fácil encontrar dispositivos que cumplan con los requisitos de sensores, capacidad de cálculo y de ser programado, así como soporte para entornos de aprendizaje automático. Al tener apariencia de un reloj de pulsera, resultan un soporte ideal: son dispositivos de uso cotidiano, con una manipulación sencilla, y que se llevan siempre puestos. Con el interés de buscar un sistema lo más transparente de cara al usuario posible nos decantamos por esta opción, a pesar de conocer las limitaciones del uso de la muñeca como fuente de información de la aceleración para la detección de actividades \cite{Chen2005} .

\subsubsection{Relojes Inteligentes WatchOS}

A la hora de definir las familias en las que subdividir este tipo de dispositivos encontramos que por norma general cada fabricante ha decidido crear su plataforma única. Tan solo el sistema operativo WearOS parece ser utilizado por varios fabricantes diferentes. No descartamos la opción de implementar una solución propietaria, por lo que también analizamos algunas de los sistemas de microcontroladores más extendidos.

% \tablan{tab:familias_smartband}{Familias de Relojes Inteligentes y Pulseras de Actividad}{llll}{
\tablan{tab:familias_smartband}{Familias de Relojes Inteligentes y Pulseras de Actividad}{p{0.20\textwidth}p{0.1\textwidth}p{0.20\textwidth}p{0.35\textwidth}}{
  \emph{Plataforma} & \emph{Progr.} & \emph{Soporte AA} & \emph{Notas} \\ \midrule
  Garmin     & No & No &  \\
  FitBit     & Si & No & \\
  Huawei     & No & No & \\
  TizenOS(Samsung) & Si & \makecell[l]{uTensorfFlow-lite} & Puede soportar otros entornos como Caffe2, PyTorch pero han de ser portados por el desarrollador \\
  WatchOS(Apple) & Si & \makecell[l]{CoreML\tnote{1}} & PyTorch, Tensorflow, Caffe, LibSVM y otros modelos pueden ser importados a CoreML. \\
  Arduino/ESP32 & Si & \makecell[l]{TensorFlow-lite} & Muy baja potencia de cálculo y memoria impiden la ejecución de modelos complejos. \\
  WearOS(Google)  & Si & \makecell[l]{TensorfFlow-lite\\Caffe2\\NNAPI\tnote{1}} & Mediante NNAPI permite realizar cómputos intensivos. \\
}{
  \item [1] API para cálculo intensivo que permite portar bibliotecas de Aprendizaje Automático
}{2}

Comparando los requisitos con los resultados de la tabla \ref{tab:familias_smartband} descartamos en primera instancia los dispositivos Garmin, FitBit y Huawei por no permitir el desarrollo de aplicaciones que usen modelos aprendizaje automático. Aunque en teoría sí hay soporte para usar modelos entrenados en las plataformas ESP32/Arduino, al disponer de cantidades de memoria y procesadores poco potentes su uso real queda muy limitado, prácticamente restringido a una mera curiosidad.

De los tres grupos restantes de dispositivos, la familia de dispositivos WatchOS queda relegada a la tercera posición por su alto precio (hasta 5 veces más caros que los dispositivos de la competencia). La decisión final se decanta finalmente por usar dispositivos WearOS por la preferencia y conocimiento de la plataforma del que ya dispone el autor.

\subsection{Entorno de Desarrollo}

Nos referimos por entorno de desarrollo al conjunto de herramientas usadas para implementar el sistema. Tanto para el la generación, entrenamiento y análisis de modelos como para la implementación y adaptación del sistema de detección de caídas.



\subsection{Soporte en la Nube}\label{sec:req:nube}

Si bien el sistema a desarrollar es autocontenido, preveemos la necesidad de implementar una infraestructura de servidor en la nuve para ampliar la funcionalidad y permitir la recopilación de datos, especialmente durante la etapa de recopilación de muestras para crear la base de datos. En concreto buscamos una plataforma de servicios que permita:

\begin{itemize}
  \item Implementar servicios/APIs de forma sencilla.
  \item Sin necesidad de infraestructura para reducir el mantenimiento.
  \item Permita almacenar grandes cantidades de datos.
\end{itemize}

Con estas necesidades en mente seleccionamos la plataforma de servicios en la nube de \textit{Amazon AWS}(\url{https://aws.amazon.com}) dado que además de cumplir con los requisitos mínimos, su amplia popularidad ha permitido crear una comunidad que ofrece un amplio soporte. De toda la oferta de servicios disponibles nos centraremos en el servicio de almacenamiento \textit{S3}, \textit{AWS lambda} para implementar servicios \textit{sin servidor} (sin infraestructura dedicada) y \textit{Amazon API Gateway} para exponer los servicios en Internet. Como extra a valorar cabe destacar la gran cantidad de lenguajes de programación soportados por AWS lambda, lo cual facilitará la implementación al no requerir aprender nuevos lenguajes de programación.

\section{Bases de datos de actividad}\label{sec:req:bases:datos}

Tras la elección tanto de la plataforma física, entorno de desarrollo así como del algoritmo usado, iniciamos la búsqueda de una base de datos que cumpla con los criterios siguientes:

\begin{enumerate}
  \item Contenga datos de la aceleración. Idealmente las 3 componentes espaciales por separado.
  \item Contenga datos tanto de caídas como de otras actividades cotidianas.
  \item La medida de la aceleración se toma desde un sensor en la muñeca.
  \item Contenga resultados de varios sujetos
\end{enumerate}

Tras analizar como se ve en la \autoref{tab:bases_datos} varias bases de datos relacionadas con la detección de actividades cotidianas y caídas. Por la localización de los sensores de medida, la única que cumple los requisitos es UMAFall, sin embargo tiene dos problemas:
\tablan{tab:bases_datos}{Bases de datos de actividad humana}{p{0.2\textwidth}ccccm{0.2\textwidth}m{0.2\textwidth}}{
  \emph{BBDD}    & \emph{Sujetos} & \emph{AHC} & \emph{Caídas} & \emph{Muestras} & \emph{Sensores}      & \emph{Posición} \\ \midrule
  MobiFall\tnote{1}& 24      & 9   & 4      & 630      & Acelerómetro\newline Giroscopio & Pantalón \\
  MobiAct\tnote{2}& 57      & 9   & 4      & 2526     & Acelerómetro  & Pantalón \\
  tFall\tnote{3}  & 10      & *   & 8      & 10909     & Acelerómetro  & Bolsillo\newline Bolso \\
  SisFall\tnote{4}& 38      & 19  & 15     & 4505      & Acelerómetro\newline Giroscopio & Cintura \\
  UMAFall\tnote{5}& 17      & 8   & 3      & 531       & Acelerómetro\newline Giroscopio\newline Magnetómetro & Pecho\newline Cintura\newline Pantalón\newline Tobillo\newline Muñeca \\
}{
  \item [1]\citeA{MobiFall}
  \item [2]\citeA{MobiAct}
  \item [3]\citeA{tfall}
  \item [4]\citeA{Sucerquia2017}
  \item [5]\citeA{Edu/UMA/2017}
  }{2}
\begin{itemize}
  \item Escasez de eventos: tan solo 531 eventos, de 25 categorías reducen el número de instancias por categorías a un promedio de 21. Claramente insuficiente para obtener una cifra significativa.
  \item Poco extendida entre la comunidad. En parte por ser la compilación más reciente, no hemos encontrado estudios suficientes que utilizasen UMAFall para poder realizar una comparación de resultados.
\end{itemize}
Concluimos esta etapa considerando desierta la oferta disponible. Tomamos la decisión de dividir la tarea a realizar en dos ejercicios:
\begin{enumerate}
  \item Implementar, validar, y comparar la viabilidad del algoritmo (independientemente de la plataforma elegida).
  \item Portar el modelo y entrenarlo para la plataforma escogida.
\end{enumerate}

Esta separación de objetivos permite reducir los requisitos de las bases de datos necesarios para cada uno tal y como recogemos a continuación.


\subsection{Bases de Datos para estudiar el algoritmo}

Al querer únicamente estudiar cómo se comporta el modelo diseñado, no es exigencia la restricción impuesta por el uso de un reloj o pulsera como dispositivo de captura de que los datos sean medidos en la muñeca. Por tanto, estudiando de nuevo la tabla \todo{crear la tabla y poner la referencia} observamos que ahora sí que hay varios candidatos viables.

%SisFALL (Tras considerar otros muchos como umafall, mobiact/mobifall, etc): mayor cantidad de caídas y muchas de las actividades son consideradas \textit{quasi-caidas} y por tanto son casos límite. Si el modelo es capaz de dar buenos resultados en estas condiciones, es un buen modelo. Además, es usado por \cite{Musci2020} y varios de los estudios presentes en \cite{Anita2020} por lo que resulta ideal para comparar los resultados.

\subsection{Base de Datos para la implementación final}

Tras validar el funcionamiento y compararlo con otros modelos en igualdad de condiciones, esta etapa puede realizarse con conjuntos de datos que no dispongan de caídas. Dado que por la naturaleza del modelo propuesto el entrenamiento se realiza únicamente con muestras de actividades cotidianas. A pesar de la existencia de Bases de Datos que cumplen los criterios, se decide finalmente optar por generar un conjunto propio. Al formar parte del desarrollo realizado, trataremos este punto más adelante, en \fullref{sec:imp:accelcapture}.




\warn{revisar esto}
Para poder generar un modelo con el mismo tipo de datos y capturado de la misma forma a los que deberá procesar posteriormente, optamos por realizar un base de datos de datos propio usando el mismo dispositivo sobre el que se ejecutará el prototipo como plataforma de captura.

Este base de datos se realiza usando únicamente información del sensor de aceleración triaxial del reloj Fossil Sport Gen3 \ref{app:sec:fossil} a una frecuencia de muestreo de 50Hz. Los valores de aceleración medidos están expresados en $m/s^2$ dentro de los límites de la plataforma. Este es un base de datos no etiquetado dado que el objetivo es entrenar un modelo de predicción de series temporales y no un clasificador como suele ser habitual.

Para generar las capturas de los acelerómetros se implementa la aplicación \accelcapture/ (\autoref{sec:imp:accelcapture}). En el  \autoref{chap:generar:dataset} se encuentra una descripción del proceso de generación y especificaciones detalladas del base de datos generado.
\iffalse
\todo{no ponerlo como apéndice, es parte importante, ponerlo aquí (referenciar ciertas cosas al apéndice que sean de menor interés)}
\fi
A la hora de parametrizar el sistema de captura de datos, optamos por mantener los 50Hz de frecuencia de muestreo. \cite{Liu2018} demuestra que se pueden conseguir buenos modelos basados en aprendizaje automático con tan solo 5,8Hz, aunque generaliza en 21,3Hz la frecuencia de muestreo mínima necesaria. Nos mantenemos por encima de este límite para eliminar una posible fuente de errores en el modelo.





\section{Modelos y Algoritmo}\label{sec:req:modelos}
\iffalse
\todo{nada de vaguedades, qué modelos, cuales voy a usar, qué valores he calculado yo y cuales vienen de fuera}
Estudios similares han mostrado la mayor capacidad de los sistemas basados en aprendizaje automático. Sin embargo estos algoritmos no han conseguido desplazar en los sistemas comerciales a los métodos analíticos por la gran capacidad de cálculo que requieren. Algunos trabajos han demostrado la eficacia de hibridar los modelos usando un modelo computacionalmente simple para realizar una tarea de detección de candidatos a caídas de fondo y un modelo complejo que actúe únicamente sobre este subconjunto de episodios. Este es el acercamiento que toma este trabajo.
\fi
\subsection{Modelos Analíticos}

En el apartado \ref{sec:arte:modelos_analiticos} hemos introducido el contexto de este conjunto de algoritmos. Para implementar el sistema necesitamos un modelo simple y de bajo coste computacional y que se base principalmente en los datos captados por un acelerómetro de tres ejes. Dada la situación del sensor, toda medida de la postura queda descartada por no ser aplicable. Finalmente, dado que su función es principalmente la de mecanismo de atención o generador de eventos de posibles caídas es primordial que su sensibilidad sea alta para no perder caídas. La tasa de especificidad modulará la eficiencia del sistema en los planos del tiempo de cálculo y consumo de energía, siendo preferible usar algoritmos con alta especificidad, siempre y cuando no se comprometa la sensibilidad. Tras analizar varios modelos y recopilar los resultados en la tabla \ref{tab:resumen_mod_anal} decidimos optar por la familia de algoritmos propuestos por \cite{Bourke2006} por ser el único que de forma consistente ofrece una sensibilidad del 100\%. Si se observan los resultados de los diferentes algoritmos analizados, se observa la gran varianza de resultados según el estudio que realizase el análisis. Por ejemplo, el algoritmo \textit{Bourke 1} pasa de ser un algoritmo perfecto con una precisión del 100\% a ser inservible por su bajísima especificidad según el conjunto de datos usado.

\tablan[0.95\linewidth]{tab:resumen_mod_anal}{Características de los modelos analíticos estudiados}{lccccccccll}{
          &       &     & \multicolumn{2}{c}{\emph{Estudio 1\tnote{1}}} &\multicolumn{2}{c}{\emph{Estudio 2\tnote{2}}} & \multicolumn{2}{c}{\emph{Estudio 3\tnote{3}}} & \\
 {\small Modelo}  & {\small Sens.\%} & {\small Esp.\%} & {\small Sens.\%} & {\small Esp.\%} & {\small Sens.\%} & {\small Esp.\%} & {\small Sens.\%} & {\small Esp.\%} & {\small Sensores } & {\small Postura} \\ \midrule
 Kangas 1\tnote{4}  & 100 & -- & 48 & 94 &   &    &   &  & ACC & Si \\
 Kangas 2           & 100 & -- & 48 & 94 & 86 & 98 &   &  & ACC & Si \\
 Kangas 3           & 71  & -- & 31 & 97 & 94 & 94 &   &  & ACC & Si \\
 Bourke 1\tnote{5}  & 100 & 100& 100& 19 & 100& 79 &   &  & ACC & No \\
 Bourke 2\tnote{6}  & 97  & 99 & 72 & 87 & 70 & 99 &   &  & ACC & Si \\
 Bourke 3\tnote{7}  & 100 & 100& 87 & 97 &    &    &   &  & ACC + GYRO & Si \\
 Chen \tnote{8}     &     &    & 76 & 94 &    &    &   &  & ACC & No \\ % ACC & Si
 FallIndex\tnote{9} &     &    &    &    &    &    & 42&95& ACC & No \\
}{
  \item [1] \citeA{Bagala2012}
  \item [2] \citeA{Aziz2017}
  \item [3] \citeA{Vilarinho2015}
  \item [4] \citeA{Kangas2008}
  \item [5] \citeA{Bourke2006}
  \item [6] \citeA{Bourke2008}
  \item [7] \citeA{Bourke2010}
  \item [8] \citeA{Chen2005}
  \item [9] \citeA{fallindex00}
  }{3}



\paragraph{Algoritmo Bourke 1} El modelo de Bourke 1\cite{Bourke2006} se basa en el estudio de la evolución del módulo de la aceleración. Partimos del principio de que todo cuerpo sufre una aceleración constante de $9,8m/s$ o 1G correspondiente a la gravedad terrestre. Como podemos apreciar en la figura \ref{fig:bourke_thresholds}, al principio de una caída, típicamente se produce un descenso en el módulo de la aceleración ya que al caer, la componente vertical contrarresta el efecto de la gravedad. Al impactar posteriormente contra el suelo se produce un pico en la aceleración. El algoritmo de Bourke utiliza esta característica forma para establecer dos niveles para determinar la detección de caídas. Estas cotas se denominan \textit{LFT} (Nivel de caída inferior) para el nivel mínimo de aceleración medido y \textit{UFT} (Nivel de caída superior) para el nivel superior. El algoritmo de Bourke considera que se ha producido una caída cuando en un intervalo determinado de tiempo se rebasan ambos límites.

El proceso para elegir estos límites garantiza que por construcción la sensibilidad del algoritmo sea del 100\% sobre la base de datos elegida. Para ambos límites, se recorre todo el conjunto de datos y se busca entre todas las caídas:
\begin{itemize}
  \item El mayor de los mínimos de aceleración de las caídas (para definir \textbf{LFT})
  \item El menor de los máximos de aceleración de las caídas (para definir \textbf{UFT})
\end{itemize}

Garantizando así que todas las caidas del conjunto de datos serán detectadas, y por tanto la sensibilidad siempre sea máxima. Por contra esto hace que el valor de especificidad del algoritmo sea altamente dependiente del resto de actividades incluidas en la base de datos. Esta característica explicaría la gran disparidad de resultados de la tabla \ref{tab:resumen_mod_anal}.

\figura{BourkeThresholds}{fig:bourke_thresholds}{Detalle del modelo de detección de Bourke}

Recogemos en la tabla \ref{tab:bourke_threshold_values} una lista con los diferentes valores obtenidos para LFT y UFT según diversos estudios. Su valor es claramente orientativo dada la dependencia de la base de datos usada, que nos obligará a obtener estos valores nuevamente durante la implementación para adaptarlos a SisFall.

\tabla[0.43\linewidth]{tab:bourke_threshold_values}{Valores UFT y LFT de Bourke por estudio}{llcc}{
  \emph{Estudio} & \emph{Posición} & \emph{UFT} & \emph{LFT} \\ \midrule
  \citeA{Bourke2006} & Pecho & 3,52g & 0,41g \\
                     & Cintura & 2,74g & 0,60g \\
  \citeA{Bourke2010} & Cintura & 2,80g & 0,65g \\
  \citeA{Bagala2012} & -- & 1,79g & 0.73g \\
}{2}


\subsection{Modelos Basados en Aprendizaje Automático}

\subsection{Modelos Híbridos}

\section{Biblioteca de Modelos Neuronales} \label{sec:req:tflite}

Una de las piezas angulares de todo el desarrollo es la generación y entrenamiento del modelo. Si bien es factible realizar una imlementación directa tanto del modelo como de las funciones de entrenamiento, no es óptimo ni recursos ni en resultados. Existen multitud de de bibliotecas para el desarrollo, entrenamiento y despliegue de modelos de los que citamos brevemente los más extendidos:

\paragraph{Caffe2 y PyTorch}

Caffe2 (\url{https://caffe2.ai/}) Es en su origen una biblioteca de modelos neuronales centrada en la velocidad con soporte para el lenguaje de programación \textit{python}. Desde hace 2018 forma parte de la biblioteca PyTorch (\url{https://pytorch.org/}). PyTorch auna un módulo de cálculo de tensores y uno de modelado de redes neuronales con gran énfasis en la aceleración mediante GPU. Es, junto con TensorFlow, la biblioteca de Modelos Neuronales más usada en el ámbito académico.

\paragraph{TensorFlow + Keras}

Similar al caso anterior, Keras (\url{https://keras.io/}) es un módulo que ofrece una API de alto nivel para facilitar la generación y entrenamiento de modelos. No es en si una biblioteca de redes neuronales sino una interfaz de alto nivel a varias de estas como TensorFlow, Theano, CNTK y otros. TensorFlow es al igual que PyTorch una biblioteca de cálculo de tensores y de modelado de redes si bien sus algoritmos están optimizados para ejecutarse sobre unidades TPU también soportan aceleración en GPUs. Desde la versión 2.0 de TensorFlow, se incluye Keras para ofrecer tanto una API de bajo nivel como otra de alto para la generación de modelos. Al ser propiedad de Google, la integración con sus productos es de las mejores. Dispone de un módulo para la ejecución de modelos en plataformas embebidas.

\paragraph{SciKit-Learn}

Scikit-learn (\url{https://scikit-learn.org/}) se presenta como una biblioteca de funciones de tratamiento de datos. Ofrece imlementaciones de funciones y algoritmos de aprendizaje automático muy variados como \textit{random forest}, \textit{SVM}, textit{K-NN} y otros, así como funciones para el pre-procesado de datos. Soporta también modelos de redes neuronales, aunque su API de desarrollo des de muy alto nivel y dificilmente industrializable.

La elección del sistema WearOS para imlementar la aplicación, unida a la popularidad tanto en el ámbito académico como profesional así como el soporte y optimización para sistemas embebidos hace que nos decantemos por la opción TensorFlow + Keras.

\section{Optimización de modelos}\label{sec:req:optimizacion}

Los modelos basados en redes neuronales adolecen generalmente por su gran tamaño y la necesidad realizar infinidad de cálculos. Estos problemas se acrecentan cuando el objetivo es usar el modelo en un sistema embebido con poca memoria y microprocesadores optimizados para el consumo de energía en lugar de potencia de cálculo. Esta escasez de recursos obliga a la \textit{optimización} del algoritmo y modelo usados. Entendiendo por optimización la reducción de complejidad controlada para conseguir un modelo que requiera menor capacidad de cálculo y memoria. Estas técnicas tendrán todas un impacto en la precisión del sistema por lo que el objetivo es conseguir un equilibrio entre la degradación de la capacidad predictiva y la reducción de la latencia debida al menor coste computacional. Establecemos dos criterios previos a cumplir por el sistema final:

\begin{itemize}
  \item Arbitrariamente escogemos una latencia máxima de 1 segundo desde el instante del primer impacto.
  \item La elección de la plataforma nos limita tanto en velocidad de cálculo como en tamaño al disponer únicamente de 512MBytes de memoria RAM.
\end{itemize}

\subsection{Técnicas de Optimización: Pruning y Cuantificación}

  A pesar de la creciente capacidad de cálculo de los dispositivos vestibles, será necesario optimizar el rendimiento del modelo con el fin de conseguir la mayor capacidad de predicción por Hz del sistema. Para ello dispondremos de dos técnicas: \textit{Pruning} y \textit{Cuantificación}.

\subsubsection{Pruning}

Si añadimos que indican la contribción de cada nodo o neurona al resultado final. Es normal que en esa red haya nodos que aporten más información que otros. Este efecto no es necesariamente negativo, la regularización o normalización L1 busca precisamente este efecto para reducir el sobreajuste de la red a los datos de entrenamiento.


Si pensamos en reducir el tamaño de una red neuronal, probablemente una de las soluciones que nos venga a la cabeza sea sea hacer un \textit{tree-shacking}, eliminar aquello que no se necesita. El término, usado en ciencias de la computación se aplica para reducir el tamaño de aplicaciones al eliminar el código que no es utilizado. Podemos pensar en realizar un cribado similar en la red eliminando aquellos pesos menos relevantes para la salida. Propuesto en \citeA{Mozer1989} como una herramienta para simplificar las redes de neuronas y facilitar así la comprensión de su funcionamiento, acelerar el aprendizaje al reducir el tamaño de la red y por la misma razón facilitar la capacidad de la red de generalizar. La técnica propuesta consiste en medir la influencia de los nodos en la señal de salida: comparar el resultado con y sin el nodo activado. Sabiendo la influencia de la aportación de cada nodo podemos determinar un umbral y eliminar los nodos con menor contribución reduciendo la red. \citeA{Yann1989} confirma que además de conseguir reducir el tamaño de la red a la mitad, el aprendizaje de la misma puede llegar a mejorar.


Recordemos, una capa de una red neuronal se puede modelar mediante la ecuación $H_i = f_i(W_ix+B_i)$ donde tenemos una matriz de conexiones $W_{i}$ y una matriz de sesgos $B_{i}$ y la función de activación $f_i(x)$. Si añadimos varias capas sucesivas podemos deducir que el impacto de un nodo, especialmente en las primeras capas de la red crecerá con cada capa posterior. Sin embargo eso no es así ya que en muchos casos la aportación puede quedar neutralizada por sucesivas inversiones de signo en el resto de capas, o resultar que el nodo se activa prácticamente con el mismo valor o que las capas posteriores estén en el límite de la función de activación y su influencia sea prácticamente irrelevante\cite{Mozer1989}. Es más, posiblemente el lector haya rápidamente asociado por su similitud esta técnica con una regularización mediante \textit{dropout}. Las similitudes son evidentes, pero hay sin embargo diferencias importantes:
\begin{itemize}
  \item Los nodos anulados o eliminados se pierden de forma definitiva
  \item Los nodos se seleccionan de acuerdo a su aportación y no de forma aleatoria
\end{itemize}

A la hora de aplicar una técnica de pruning sobre la red es necesario entender que tras cada etapa de eliminación de nodos debe realizarse una etapa de reajuste de la red resultante para corregir el efecto de la pérdida de las neuronas. Este reajuste o aplanamiento de la red resulta especialmente efectivo en redes con múltiples capas como demuestra \citeA{Cai2020} logrando reducir el tamaño de \textit{ResNet110} en un 30\% y un descenso de la precisión de la red del 0,15\% y la reducción del número de operaciones necesarias a la mitad. Otros estudios\cite{Lee2019,Bartoldson2019,Han2015} confirman estos resultados y presentan una novedad: Las grandes reducidas mediante pruning presentan mejores resultados que los modelos pequeños (ResNet32 tras ser comprimida es mejor clasificando imágenes que WRN16, a igual número de nodos). Básicamente, se mantiene la mayor capacidad de generalización o aprendizaje del uso de mayor número de capas incluso al reducir el número de neuronas.

Esta técnica, junto con la cuantificación de los valores, se presenta como una herramienta imprescindible para lograr los objetivos de nuestro sistema. Tensorflow permite aplicar prunning a los modelos, aunque el único algoritmo disponible, \textit{prune low magnitude} sea similar al básico definido por \citeA{Mozer1989}: eliminar aquellos nodos cuyo peso sea menor que un determinado nivel.
\[
  pruned(w_{i})=\left\{
    \begin{array}{lcl}
      w_i & si & |w_i| > \lambda \\
      0 & si & |w_i|\leq  \lambda
    \end{array}
    \right.
\]

Este proceso se va realizando durante un número determinado de iteraciones de entrenamiento durante el cual se van anulando progresivamente una cantidad determinada (ya sea un número fijo o una evolución polinomial) de nodos en cada etapa de entrenamiento. \cite{Yann1989} Los modelos son cada vez más comlejos y con el aumento en complejidad de problemas seguirán crecienco. Un método que elimina nodos permite no solo mejorar la capacidad de generalizar

\iffalse

  https://www.machinecurve.com/index.php/2020/09/23/tensorflow-model-optimization-an-introduction-to-pruning/
  ligeramente más técnico.
https://www.machinecurve.com/index.php/2020/09/29/tensorflow-pruning-schedules-constantsparsity-and-polynomialdecay/
\fi


\subsubsection{Cuantificación}

Además de eliminando nodos de la red, hay otra técnica que permite reducir el tamaño y requisitos de esta: reducir el número de bits usados para representar las conexiones y sesgos. Entendemos por cuantificación al proceso de discretizar un grupo de valores. En este caso en concreto hace referencia al mencionado descenso en la precisión numérica usada para representar los pesos de nodos y valores de entrada y salida de una red neuronal.

Una de las primeras aproximaciones a la cuantificación de los parámetros de redes neuronales aparece en \citeA{Courbariaux2015}. \citeauthor{Courbariaux2015} busca una solución al problema de los crecientes requisitos de potencia de cálculo necesarios por los modelos de redes de neuronas. Esta optimización espera reducir la dependencia en la mejora de las unidades GPU(\textit{Graphics Processing Unit}) y CPU, hasta ahora uno de los principales factores que limitan evolución en este área. \cita[{\citeNP[p.~1]{Courbariaux2015}}]{La mayoría de los cálculos realizados en el entrenamiento y aplicación de una red neuronal consisten en multiplicar un peso determinado por valor real por una activación definida por un valor real}, así pues decide binarizar tanto pesos como activaciones y reducirlos a solo dos valores: $0$ y $1$ o $+1$ y $-1$. Los resultados obtenidos son esperanzadores, una reducciónd el tamaño a 1/16 del original (de 16 bits usados para representar pesos y activaciones pasa a uno único) y una mejora en la velocidad de respuesta de la red tres veces menor.

La reducción en el tamaño de la red es evidente. Si tenemos $N$ parámetros en una red representados con $m$ bits cada uno, el tamaño final de la red es $S(red)=Nm$ es por tanto una dependencia lineal. La mejora en los tiempos de cómputo es más complejo. Trabajar con números reales, o de coma flotante, requiere de unidades de cálculo especializadas, ya sean GPUs o unidades de coma flotante en CPUs. Este requisito de unidades especializadas es determinante en el rendimiento, pues será determinante en la velocidad de cálculo del modelo. Esta problemática la estudia en detalle \citeA{Vanhoucke2011} buscando optimizar la velocidad de los modelos. Se fija en la lentitud de las unidades CPU para realizar grandes cantidades de cálculos en coma flotante y propone en este caso usar valores enteros. Predice que el impacto negativo será mínimos pues \cita[{\citeNP[p.~5]{Vanhoucke2011}}]{las activaciones son probabilidades, limitadas al rango $[0:1]$ así pues no hay riesgos relacionados con el escalado al usar enteros. Y como la entrada de cada nodo es la salida de una activación esto asegura que el rango de valores permanecerá acotado}. Aplicando cuantización para convertir los pesos de reales a enteros, y otras optimizaciones en la implementación, consigue reducir a menos de un tercio el tiempo de respuesta de una red de procesado del habla con un error de precisión añadido inferior al 0,1\%.

Hasta ahora hemos tratado pesos y activaciones como un todo: aplicamos la misma regularización o reducción de precisión a ambos conjuntos de valores por igual. A esta técnica se la conoce como \textit{Cuantificación de precisión uniforme}, en contraste con la \textit{Cuantificación de precisión mixta}. Este última propone tratar la reducción de tamaño de cada conjunto de valores por separado, consiguiendo una mayor reducción del tamaño con un impacto menor en la precisión del modelo \cite{Jin2019} e incluso a veces mejorando el rendimiento.

El resultado del último párrafo, de nuevo, no habrá pillado desprevenido al lector que rápidamente asocia la reducción en la precisión con un aumento del ruido, al que llamamos \textit{ruido de cuantificación} y por tanto similar a una regularización de la red por ruido \cite{Bishop1995,Noh2017}. Esta capacidad de regularización de la red resultante de la cuantificación y entrenamiento de los pesos de la red es utilizado directamente por \citeA{Wu2019} no ya con el objetivo de mejorar el tamaño y velocidad del modelo sino para mejorar su capacidad de aprendizaje y generalización mejorando la capacidad de clasificación de imágenes de una red en hasta un 10\%.

TensorFlow Lite ofrece diversas opciones para realizar la cuantificación de modelos como presenta la tabla \ref{tab:tflite_quantization}. En resumidas cuentas ofrece dos opciones: realizar la cuantificación sobre modelos ya entrenados o realizarlo durante el entrenamiento del modelo. Este segundo tipo ofrece mejor granularidad y fiabilidad ya que permite definir secciones específicas del modelo que no sean alteradas y asegura que los rangos y escalados de se realizan teniendo el cuenta todos los datos disponibles.


\tablan{tab:tflite_quantization}{Detalle de los tipos de cuantificación disponibles en TensorFlow Lite}{llcccl}{
  Técnica                   & Datos Requeridos  & Compresión  & Precisión & Velocidad & Hardware Soportado \\ \hline
  Float16\tnote{1}          & -                 & \~50\%      & 0         & --        & CPU, GPU \\
  Rango Dinámico\tnote{1}   & -                 & \~75\%      & ---       & x2,5      & CPU, GPU \\
  Enteros\tnote{1}          & Si\tnote{3}       & \~75\%      & --        & x3        & CPU, GPU, TPU, DSP \\
  QAT\tnote{2}              & Si\tnote{4}       & \~75\%      & -         & x3        & CPU, GPU, TPU, DSP \\
}{
  \item [1] Cuantificación post-entrenamiento.
  \item [2] \textit{Quantification Aware Training} (Entrenamiento Para Cuantificación), cuantificación se prepara durante el entrenamiento.
  \item [3] Es necesario proveer un subconjunto representativo de datos.
  \item [4] Se usa todo el conjunto de entrenamiento.
  }{3}

La precisión y número de bits usado para representar los pesos y activaciones puede verse en la tabla \ref{tab:tflite_weights_activations}. La cuantificación a enteros y OAT resultan en redes con el mismo peso, la diferencia entre ambos es cómo se calculan y escalan los valores (Enteros usa un subconjunto de datos, lo cual penaliza su precisión, contra el uso de toda la base de datos de entrenamiento como hace OAT), y en que OAT al realizarse durante el entrenamiento permite compensar el error introducido, funcionando, como ya hemos mencionado, como una regularización.

\tablan[0.55\linewidth]{tab:tflite_weights_activations}{Tipos de representación usados en cada técnica}{lcc}{
  Técnica & Pesos & Activaciones \\ \hline
  Float16 & Float(16bit) & Float(16bit) \\
  Rango Dinámico & Entero(8bit) & Entero(8bit)\tnote{1} \\
  Enteros & Entero(8bit) & Entero(8bit) \\
  Enteros Experimental & Entero(8bit) & Entero(16bit) \\
  OAT     & Entero(8bit) & Entero(8bit) \\
}{
  \item [1] Las activaciones se convierten de coma flotante a entero al vuelo y el resultado se almacena en coma flotante
  }{2}

  La combinación de estas técnicas, no solo es posible, sino deseable para conseguir reducir el consumo de recursos, memoria, consumo energético y tiempo de proceso, en órdenes de magnitud. El uso de Pruning y Cuantificación permite reducir grandes redes a una fracción: VGG16 se comprime de los 552MBytes originales a apenas 11,3MBytes, Alex-Net adelgaza de 240MBytes hasta los 6,9MBytes ambas sin pérdida de precisión\cite{Han2015}. Respecto a los tiempos, según \citeauthor{Han2015}, en un chip integrado para móviles los tiempos de ejecución son en promedio 4,2 veces más rápidos. Dados estos resultados previos, esperamos que estas técnicas tengan un gran impacto a la hora de permitir la ejecución de modelos de calidad de manera eficiente.














La cuantificación de los valores de entrada y salida a la red consiste en reducir el peso de estos valores. Normalmente se normaliza la entrada y se convierte en valores de tipo entero. Tras el paso por la red, a la salida se aplicac la operación inversa para escalar la salida al rango de valores esperado.

Por su parte la cuantificación de la red neuronal realiza una operación similar con los pesos de los enlaces de las neuronas. Reduce la precisión de las representaciones de estos valores consiguiendo un ahorro en recursos y una mejora en tiempo de cálculo al eliminar el requisito de usar unidades de coma flotante en sistemas embebidos.




% \end{document}



\chapter{Generación de la base de datos}\label{chap:generar:dataset}
% !TeX root = ../tfm.tex
%! TEX root = ../tfm.tex

En este capítulo expondremos las tareas realizadas para obtener una base de datos propia para el posterior entrenamiento del sistema, así como la descripción de la misma y de los estudios previos de las tramas obtenidas que nos ayudarán en el capítulo siguiente a la definición del modelo y parámetros del algoritmo.

Como ya hemos presentado anteriormente en la sección \ref{sec:req:bases_datos}, nos encontramos ante la falta de una base de datos para entrenar el modelo del producto final. Al tratarse de un modelo experimental es necesario conseguir un conjunto de datos lo más parecido posible a los capturados por la plataforma física utilizada. A su vez es necesario validar la capacidad de esta plataforma para servir al propósito de este trabajo. Así, con esta doble tarea de generar una bases de datos y confirmar que la plataforma cubre nuestras necesidades funcionales realizamos el primer desarrollo propio del trabajo.

\section{Captura de datos: AccelCapture}\label{sub:imp:accelcapture}

\textit{AccelCapture} es el nombre del sistema de captura de tramas de la aceleración capturada por el sensor situado en un reloj inteligente con sistema operativo \textit{WearOS}. Consta de dos componentes: una unidad de adquisición de datos, el propio reloj, y una segunda unidad de almacenamiento situada en la nube. Tal y como puede apreciarse en el diagrama de clases de la figura \ref{fig:accelcaptureClasesUml} la aplicación en si misma se compone de tres bloques funcionales.

\figura[0.47]{accelcaptureClassUML}{fig:accelcaptureClasesUml}{Diagrama de clases de AccelCapture}

\paragraph{MainWearActivity}
El punto de entrada de la aplicación. Genera la interfaz de usuario y permite introducir el nombre el usuario a la vez que verificar el estado del servicio de captura de datos e iniciar o parar su ejecución.

\paragraph{SensorsReader}
Un servicio que al ser creado se registra para recibir eventos y lecturas del sensor de aceleración triaxial del reloj. Posee una cola circular donde almacena los últimos 5 segundos de muestras (si por ejemplo la velocidad de lectura del sensor es de $1/50$, el buffer posserá $50 * 5 = 250$ muestras).
Este servicio realiza el análisis de la aceleración para detectar movimiento según el algoritmo definido en la figura \ref{fig:capturaFlow}. Tras cada muestra el algoritmo evalúa la variación de la aceleración según la fórmula: 

\[
  \Delta A_{i}=\left\{
    \begin{array}{lcl}
      |A_i - g| & si & i = 0 \\
      |A_i - A_{i-1}| & si & i > 0 \\
    \end{array}
    \right.
\]

Donde $g$ corresponde a la aceleración de la gravedad terrestre o $9,8m/s^2$. Este valor $\Delta A_i$ se utiliza para realizar la detección de actividad mediante dos mecanismos:

\figura[0.55]{capturaFlujo}{fig:capturaFlow}{Flujo de trabajo de la aplicación AccelCapture}

\begin{enumerate}
  \item \textbf{Movimiento repentino} Se activa si la muestra leida es mayor de $2g$.
  \item \textbf{Movimiento prolongado} Activado cuando la aceleración promedio del contenido del buffer es superior a $0,3m/s^2$
\end{enumerate}

La cota de detección de movimiento repentino se calcula a partir de los resultados obtenidos por \citeA{Bourke2006} para la medición de la cintura por ser la que menor aceleración registra. La diferencia entre la aceleración de pico y de valle usada por bourke es de $\Delta A = (2,74 - 0,60)g = 2,14g$ redondeando a la baja a $2g$ para garantizar la detección de todos los eventos de este tipo. Finalmente, la cota del detector de actividad prolongada, fijada en $0,3m/s^2$, se mide de forma experimental resultado de promediar el vector para realizar el movimiento de leer la hora en el reloj (figura\ref{fig:accelMirarReloj}) y volver a la posición de reposo durante la ventana de 5 segundos que usa la aplicación. En caso de activarse alguno de estos mecanismos, el servicio considera que ha detectado una actividad y manda el contenido del buffer al servidor para ser guardado. 

\figura{MirarRelojAccel}{fig:accelMirarReloj}{$|A|$ y $\Delta A$ de resultantes de mirar el reloj}

El código de las funciones lambda puede consultarse el el apéndice \ref{app:code:accelcapturelambda} mientras que el de la aplicación AccelCapture está disponible para descarga en \url{https://github.com/aberaza/accelCapture}.

\subsection{Interfaz de Usuario}

La interfaz de usuario de la aplicación permite gestionar y observar el estado del servicio de captura así como definir el identificador de usuario al que pertenece la sesión. Como puede observarse en las capturas de pantalla de la figura\ref{fig:accelcapture:UI} se muestra en la parte inferior de la pantalla un boton con información del estado actual del servicio. El botón al ser pulsado alterna entre los estados \textit{activado} y \textit{desactivado}. En la parte superior de la pantalla se muestra la información del identificador de usuario o \textit{nombre} al que se asociarán todas las sesiones registradas. Este identificador puede modificarse desde el mismo reloj. Una vez activado el servicio de captura, puede esconderse interfaz y volver al modo reloj ya que el servicio sigue trabajando de fondo. 
\begin{figure}[htb!]
  \centering
  \begin{subfigure}[b]{0.4\textwidth}
      \centering
      \includegraphics[width=\textwidth]{accelCaptureAct.png}
      \caption{AccelCapture Activado}
      \label{fig:accelCapture:UI1}
  \end{subfigure}
  \hfill
  \begin{subfigure}[b]{0.4\textwidth}
      \centering
      \includegraphics[width=\linewidth]{accelCaptureDes.png}
      \caption{AccelCapture Desactivado}
      \label{fig:accelCapture:UI2}
  \end{subfigure}
  \caption{\label{fig:accelcapture:UI} Interfaz de usuario de AccelCapture}
\end{figure}

\subsection{Almacenamiento de datos}

\tabla[0.65\linewidth]{tab:accelcapture:api}{API del servidor AccelCapture}{llll}{
  Método  & URL         & Carga         & Función lambda    \\ \hline
  POST    & /saveAccel  & AccelSession  & doGuardarAccel()  \\
}\todo{poner la vuena ruta}

Tal y como se menciona en el apartado \ref{sec:req:nube} la plataforma en la nube se implementa usando los servicios de AWS. Para esta función crea un microservicio o \textit{API}(lista de funciones de acceso público que ofrece una biblioteca de programación) ( del tipo \textit{REST}(Arquitectura que implementa una interfaz sobre el protocolo \textit{HTTP}). Este servicio, definido en la tabla \ref{tab:accelcapture:api} se encargará de recibir la llamadas de la aplicación y guardar el contenido en un archivo \textit{JSON} (lenguaje de definición de estructuras de datos que usa la notación de JavaScript).

\subsubsection{Formato de los datos}

La estructura almacenada se muestra en la tabla \ref{tab:accelcapture:accel_session} la denominamos \textit{AccelSession}. Este objeto contiene información sobre el sensor (error y frequencia de muestreo), fecha de la captura así como unos identificadores del usuario y de la sesión además de la información capturada en los tres ejes del acelerómetro.



\begin{table}[h]
  \subtabla[0.48]{tab:accelcapture:accel_session}{Descripción del Objeto AccelSession}{lll}{
Campo           & Tipo Dato     & Descripción \\ \hline
\textit{uid}    & Alfanumérico  & Identificador único de usuario \\
\textit{sid}    & Alfanumérico  & Identificador único de sesión \\
\textit{sensorResolution} & Real & Resolución del sensor (en $m/s^2$) \\
\textit{sensorMaxRange} & Real  & Valor máximo de la medida (en $m/s^2$) \\
\textit{startTime} & Entero     & Tiempo EPOCH del inicio de la sesión (en segundos) \\
\textit{samples}  & Entero      & Número de muestras de la sesión \\
\textit{triggerMethod} & Texto  & Nombre del evento que inició la sesión de captura \\
\textit{sessionData} & SessionData & Estructura con los datos capturados del sensor \\
}{}

%\hspace{\fill}
\subtabla[0.48]{tab:accelcapture:session_data}{Descripción del Objeto SessionData}{lll}{
Campo           & Tipo Dato     & Descripción \\ \hline
\textit{duration} & Entero      & Duración, en $\mu s$, de la sesión \\
\textit{activity} & Texto       & Nombre de la actividad realizada durante la captura \\
\textit{rate}   & Entero        & Ratio de muestreo (en $Hz$) \\
\textit{accelerationX} & Lista<Reales> & Lista de valores de la aceleración capturados en el eje X (en $m/s^2$) \\
\textit{accelerationY} & Lista<Reales> & Lista de valores de la aceleración capturados en el eje Y (en $m/s^2$) \\
\textit{accelerationZ} & Lista<Reales> & Lista de valores de la aceleración capturados en el eje Z (en $m/s^2$) \\
}{right}

\caption{\label{tab:accelcapture:data_types} Estructuras de datos de AccelCapture}
\end{table}


\section{Generación de la base de datos}

Durante un periodo de 6 meses se han realizado capturas de movimiento usando la aplicación AccelCapture en diversos sujetos. Las capturas se realizan llevando el sensor activado 24 horas al día, incluidos los periodos de reposo y sueño, eliminando, de ser necesario, aquellas capturas en las que hubo alguna caida. El objetivo es capturar la mayor cantidad posible de sesiones de actividades diarias ordinarias, así como otros eventos que, a pesar de que el usuario se encuentre estacionario y en reposo, generan una señal de aceleración no constante como puede ser el uso de diferentes modos de transporte (tren, avión, coche, moto). En total se han registrado más de 10 horas de actividad repartidas en 300 sesiones diferentes.

\subsection{Procesado de las tramas}

A la hora de construir la base de datos, se mantiene prácticamente la estructura de información usada para el almacenamiento en el servidor. Los archivos correspondientes a las sesiones registradas se leen y almacenan en un dataframe. No se realiza ningún filtrado, subsampleado o tratamiento de los datos de aceleración.

\subsection{Análisis de las tramas}

Con el fin de entender las características de la señal capturada procedemos a analizar, para las componentes X, Y y Z, así como para el módulo del vector aceleración:
\begin{itemize}
  \item La señal temporal de la aceleración y la señal diferencia $dif(A_i) = |A_i - A_{i-1}|$
    (Figura \ref{fig:dataset:samples})
  \item El comportamiento frecuencial mediante la transformada de Fourirer (Figura \ref{fig:dataset:fftsample})
  \item Autocorrelación de la señal temporal y señal diferencia (Figura \ref{fig:dataset:autocorrsample}
\end{itemize}


\begin{figure}[htb!]
  \centering
  \begin{subfigure}[b]{0.48\textwidth}
      \centering
      \pincludegraphics[1.1]{DatasetXYZModSample}
      \caption{Muestra de la componentes X, Y, Z y $|\vec{A}|$}
      \label{fig:dataset:xyzmodsample}
  \end{subfigure}
  \hfill
  \begin{subfigure}[b]{0.48\textwidth}
      \centering
      \pincludegraphics[1.1]{DatasetAutocorrSample}
      \caption{Autocorrelación de las componentes}
      \label{fig:dataset:autocorrsample}
  \end{subfigure}
  \begin{subfigure}[b]{0.48\textwidth}
    \centering
    \pincludegraphics[1.1]{XYZModFFT}
    \caption{Análisis en frecuencia de X, Y, Z y $|\vec{A}|$}
    \label{fig:dataset:fftsample} 
  \end{subfigure}
  \hfill
  \begin{subfigure}[b]{0.48\textwidth}
    \centering
    \pincludegraphics[1.1]{DatasetAccelSample}
    \caption{Comparación de $dif(|\vec{A}|)$ con $|\vec{A}|$}
    \label{fig:dataset:accelsample}
  \end{subfigure}
  \caption{\label{fig:dataset:samples} Análisis de una de las muestras capturadas}
\end{figure}

El objetivo de este análisis es encontrar comportamientos cíclicos en la señal que permitan determinar el tamaño del enventanado a utilizar. Los resultados obtenidos muestran que la autocorrelación de la señal es muy baja y por tanto no existen comportamientos cíclicos que justifiquen el uso de ventanas de gran duración. Analizando las secuencias completas los resultados tanto del análisis espectral como de la autocorrelación indican que la señal es prácticamente aleatoria, sin ningún tipo de correlación interna. Con el fin de verificar este resultado, repetimos el estudio (figura \ref{fig:dataset:sub:autocor} pero esta vez enventandando con 150 muestras, o 3 segundos, la señal.

\begin{figure}[htb!]
  \centering
  \begin{subfigure}[b]{0.48\textwidth}
      \centering
      \pincludegraphics[1.1]{SubseriesAutocorLow}
      \caption{Autocorrelación de una muestra enventanada}
      \label{fig:dataset:sub:autocorlow}
  \end{subfigure}
  \hfill
  \begin{subfigure}[b]{0.48\textwidth}
      \centering
      \pincludegraphics[1.1]{SubseriesAutocor}
      \caption{Autocorrelación de una muestra enventanada en reposo}
      \label{fig:dataset:sub:autocorhigh}
  \end{subfigure}
  \caption{\label{fig:dataset:sub:autocor} Estudio de la autocorrelación de la señal enventanada}

\end{figure}

Los resultados se repiten en su mayoría, la señal no presenta periodicidades remarcables salvo cuando se analizan muestras en reposo. En estas suele apreciarse un tono de baja frecuencia (entre 4 y 12Hz) que atribuimos a un harmónico del pulso en reposo del sujeto que se introduce por efecto del filtrado digital a 50Hz que realiza el dispositivo de captura.

Al no haber encontrado ninguna componente espectral predominante o ninguna tendencia temporal, entendemos que ampliar el tamaño de la ventana con el único objetivo de incluir contexto para el modelo de redes neuronales no es necesariamente efectivo y no debería en ningún caso elegirse un tamaño tal que lastrara el tiempo de ejecución del algoritmo.

\subsection{Descripción de la base de datos}

La base de datos final consta de 37563 segundos de actividad capturada en 274 sesiones diferentes realizadas por 3 sujetos de edades que varían entre los 34 y 77 años. Los usuarios han realizado sus activiades diarias con el dispositivo de captura situado en la muñeca izquierda (sin especificar si en la parte superior o inferior de la misma). Los tres sujetos son diestros.

\tablas{tab:dataset:descripcion}{Resumen de propiedades de la base de datos}{lll}{
  Atributo      & Valor   & Notas \\ \midrule
  Sesiones      & 274     &   \\
  Sujetos       & 3       & Edades: 34, 40, 77 \\
  Sensor        & Acelerómetro 3ejes & Las componentes X, Y, Z se presentan por separado \\
  Frecuencia Muestreo & 50Hz & \\
  Unidades medida & $m/s^2$ & \\
  Resolución del Sensor & 0,0011 $m/s^2$ &  \\
  Valor Máximo & 60$m/s^2$ & \\
}



% Esta sección añade MUCHOS errores,especialmente captura flujo tex 

\chapter{Desarrollo de la aplicación}\label{chap:desarrollo:algoritmo}
% !eX root = ../tfm.tex
%! TEX root = ../tfm.tex

\begin{comment}
Aportar detalles del proceso de desarrollo incluyendo fases e hitos del proceso, diagramas representativos de la arquitectura y funcionamiento, capturas de pantalla para ilustrar el funcionamiento, etc.
\end{comment}

Tras un primer contacto con los componentes escogidos para implementar la aplicación en el capítulo anterior nos enfrentamos al desarrollo de la plataforma final. Empezaremos recuperando la base de datos \textit{Sisfall} como mencionamos en el capítulo de requisitos (Sección \ref{sec:req:bases_datos}) para implementar un prototipo del algoritmo, lo describiremos, analizaremos los resultados y compararemos con otros métodos ya existentes.

Con el modelo validado nos centraremos en optimizarlo para mejorar el consumo de recursos de cara al siguiente paso: implementar el sistema final. Explicaremos en la estructura del sistema cliente-servidor escogida y entraremos en detalle en el estudio y análisis de la aplicación para dispositivos llevables presentada. Estudiaremos la arquitectura, interfaz de uso y rendimiento antes de abordar, en el capítulo siguiente, la idoneidad y <como de bien se adapta> a los requisitos previamente definidios.

\section{Modelo y Algoritmo de iFell}\label{sec:imp:model}

\subsection{Algoritmo}\label{sub:imp:model:algoritmo}

Con el fin de buscar un equilibrio entre capacidad de detección y requisitos del sistema, optamos por un algoritmo híbrido con dos modelos (uno analítico y otro basado en redes recurrentes), tal y como se presenta en la figura \ref{fig:ifell:algoritmo}. Al utilizar dos etapas permite mantener el costoso modelo neuronal aletargado a la espera de un evento del modelo analítico. Este modelo, más simple, apenas consiste de un comparador, puede ejecutarse sin problema de forma continua sin tener un impacto notable en la duración de la batería del sistema.

\figura[0.5]{deteccionFlujo}{fig:ifell:algoritmo}{Algoritmo de detección de caídas de iFell}

Justificamos ya en la sección \ref{sec:req:modelos} la necesidad de usar modelos que no analicen la postura, posición o inclinación del sensor debido a la falta de coherencia entre la posición del sensor respecto la del cuerpo por su situación en la muñeca. Este factor, que tendrá un impacto en la calidad del modelo obtenido infiere a la vez de mayor simplicidad y facilitará su posterior uso en sistemas embebidos.

\subsection{Modelo Analítico: Bourke2006}\label{sub:imp:model:analitico}

Presentado por \citeA{Bourke2006} es un algoritmo simple que al no requerir más que del módulo del vector aceleración permite ser usado en nuestro sistema. Usando dos cotas de detección consecutivas (una cota inferior para detectar valles en la aceleración y una superior para la detección del posterior pico como se muestra en la figura\ref{fig:bourke_thresholds}) permite identificar caídas basándose en la forma característica que sigue la aceleración del cuerpo en estos eventos.

Como el resto de modelos analíticos aplicados a la detección de caídas tiene la gran ventaja de ser computacionalmente simple. Su mayor contraprestación es la dificultad de balancear las métricas de Especificidad y Sensibilidad. Al basarse en niveles o cotas, podemos aumentar la sensibilidad situando el nivel de estas de manera que la sensibilidad llegue al 100\%, como requiere el algoritmo por definición, aunque afectará a la especificidad negativamente \cite{Aziz2017}. 

En nuestro algoritmo general, el modelo analítico es similar a una capa de gestión de la atención del modelo RNN. Es por esta razón que la baja especificidad  no resulta un problema ya que lo que nos interesa es que esta etapa tenga una sensibilidad próxima al 100\%. El dispositivo captura información de la aceleración en 3 ejes con una medida máxima de $9G$ y a una frecuencia de muestreo $f_m=50Hz$ \todo{seguro?}. Esta medida la convertimos en el módulo de la aceleración ($|\vec{A}| = |\sqrt{a_{x}^2+a_{y}^2+a_{z}^2}|$).:wa En concreto, comparamos el valor absoluto de la diferencia de dos mediciones consecutivas de la aceleración y comparamos con un umbral de referencia.
\[
ModeloAnalitico\rightarrow |SVTot_n - SVTot_{n-1}|\geq A_{umbral}
\]


Aunque ya hemos presentado los resultados de diferentes estudios que proporcionan valores para las cotas \textit{LFT} y \textit{UFT} del modelo (tabla \ref{tab:bourke_threshold_values}) también hemos comprobado la alta variabilidad de resultados según el conjunto de datos usado. Por esa razón, al no disponer de un estudio que analizase este modelo sobre la base de datos usada, calcularemos los valores usando las muestras de SisFall y analizaremos el comportamiento del modelo.

\subsubsection{Preprocesado de la base de datos}

Para acortar las distancias entre los resultados obtenidos en la simulación con los esperados sobre el dispositivo real realizamos un tratamiento previo de los datos para submuestrear a 25KHz.


\paragraph{Filtrado IIR}

Previo al subsampleado de la señal optamos por realizar un filtrado paso bajo respecto a la frecuencia de corte de Nyquist para señales muestreadas a 25Hz ($f_c=\frac{25}{2}=12,5$). Usaremos un filtro IIR de primer orden, con función de transferencia $H(z)$:

\[
  y_n = \alpha x_n + (1-\alpha) y_{n-1}
\]

%comment = \alpha\times x_n + \(1-\alpha\)\times y_{n-1}

\[
  H(z) = \frac{\alpha}{1-(1-\alpha)z^{-1}}
\]

Donde
\begin{align*}
\alpha = \frac{\delta t}{\delta t + RC} \\
f_c= \frac{1}{2 \pi RC}
\end{align*}

Dado que posteriormente submuestrearemos la señal a 25Hz y que sabemos que en estos primeros 12Hz hay suficiente información sobre las caídas, tomamos como ya hemos dicho $f_c=12,5Hz$. Hay que tener en cuenta que al ser un filtro paso bajo no afectará a la componente continua de la gravedad. Despejando las ecuación precedentes queda:
\begin{align*}
  f_m & = 200{Hz}\\
  \delta t & = 1/f_m = 0.005s\\
  RC & =\frac{1}{2\pi f_c} = \frac{1}{2\pi12,5} = 0,5\\
  \alpha & = \frac{0.02}{0.02 + RC} = 0,015
\end{align*}

\begin{comment}
%VER https://electronics.stackexchange.com/questions/498226/calculate-cutoff-frequency-of-a-digital-iir-filter
%https://en.wikipedia.org/wiki/EWMA_chart
%https://en.wikipedia.org/wiki/Low-pass_filter#Simple_infinite_impulse_response_filter
\end{comment}

El interés de realizar un filtrado prevo al subsampleado puede observarse al compararse los resultados expuestos en la figura \ref{fig:iir} en ella se observa la destrucción de información producida por el muestreado de la señal. Se aprecia claramente como en las curvas de la figura\ref{fig:signalIIRFilter25} en que se ha muestreado la señal sin filtrado previo se ha aplanado en exceso. Resultado especialmente nocivo para nuestro modelo analítico que se basa precisamente en detectar estos picos valles.

\begin{figure}[htb!]
  \centering
  \begin{subfigure}[b]{0.96\textwidth}
      \centering
      \pincludegraphics[0.9]{FilterAndDownsample}
      \caption{Degradación de la señal con submuestreados y filtrados}
      \label{fig:downsample}
  \end{subfigure}
  \centering

  \begin{subfigure}[b]{0.48\textwidth}
      \centering
      \pincludegraphics[1.0]{SignalvsIIRFilter}
      \caption{Señal original (200Hz) y filtrada}
      \label{fig:signalIIRFilter}
  \end{subfigure}
  \hfill
  \begin{subfigure}[b]{0.48\textwidth}
      \centering
      \pincludegraphics[1.0]{SignalvsIIRFilter25Hz}
      \caption{Señal submuestreada a 25Hz y filtrada}
      \label{fig:signalIIRFilter25}
  \end{subfigure}
  \caption{\label{fig:iir}  Efecto del filtro IIR y submuestreado en la señal}
\end{figure}

\paragraph{Submuestreado a 25Hz}

La base de datos SisFall usa un muestreado a 200Hz, el reloj elegido como soporte admite únicamente un muestreado a 50Hz como máximo. Sin embargo a raíz del ya expuesto resultado de \todo{falta referencia de los 12KHz} decidimos reducir el muestreado hasta los 25Hz, que por Nyquist debería ser suficiente para no destruir la información de las caídas de la señal.

El sub-muestreado se realiza tomando una muestra de cada n con $n=\frac{f_m}{f_o}=\frac{25}{200}=8$. En la figura \ref{fig:downsample} se aprecia la degradación que sufre la señal original al aplicarle el filtrado y posterior reducción de muestras. Se observa un doble efecto del filtrado: un suavizado de la señal resultante y un desfase no lineal. El suavizado enmascara parcialmente los valores pico y valle, sin embargo garantiza mayor fidelidad del resultado a la señal original: si bien se pierde la información en picos y valles, la señal mantiene mejor la estructura general. En el fondo se ha reducido el ruido de la señal ya que el filtro en la práctica actúa como un ponderador.

%\figura{FilterAndDownsample}{fig:downsample}{Efecto de filtrar y submuestrear la señal)





\begin{comment}
Los estudios siguientes demuestran que el filtrado de ruido mejora la capacidad de tratamiento posterior
Tian, T.; Sun, S.; Lin, H. Distributed fusion filter for multi-sensor systems with finite-step correlated noises. Inf. Fusion 2019, 46, 128-140.

Luego, para el caso de natación
Xiao, D.; Yu, Z.; Yi, F.; Wang, L.; Tan, C.C.; Guo, B. Smartswim: An infrastructure-free swimmer localization system based on smartphone sensors. In Proceedings of the International Conference on Smart Homes and Health Telematics, Wuhan, China, 25-27 May 2016; pp. 222-234.

decide que una promediado por ventana flotante de tamaño M es el que mejores resultados da: $G_{filter}=\frac{1}{M}\sum_{i=0}^{M}G_i$. (Explicado en Liu \cite{Liu2020} que usa este método con una DeepNN basada en capas CNN + 2xLSTM + Fully Connected + Softmax).

% \figura{capturaFlujo}{fig:capturaFlow}{Flujo de trabajo de la aplicación de captura de datos}
\end{comment}

\subsubsection{Cálculo de parámetros para el modelo Bourke}

Realizamos una primera iteración por todas las secuencias incluidas en el conjunto de datos buscando el valor de valle y de pico para cada una y agrupamos los resultados en dos categorías: \textit{actividad normal} o \textit{caída}. Posteriormente analizamos los histogramas resultantes para obtener la función de densidad de la distribución de ambos valores. Si se observa la figura \ref{fig:bourke:hist} a primera vista la capacidad de segregación entre caídas y actividades de la distribución de los picos de aceleración es mayor que la de los valles. En parte previsible dado lo compacto del rango ( 1g) comparado con el de los valores de pico que es potencialmente ilimitada, o limitado únicamente por la capacidad del sensor. Analizando los datos, en el caso de los valores de pico, el percentil 0\% de las caídas se sitúa en 4,5g que se corresponde con el percentil 23\% de las actividades: si establezco la cota de pico en 4,5g conseguimos detectar el 100\% de la caídas aunque detectemos también como tales el 73\% de las actividades normales. Este porcentaje baja hasta el 59\% si aceptamos no detectar el 1\% de las caídas poniendo la cota en 5.7g. Para el caso de los valles si queremos detectar el 100\% de las caídas debemos situar la cota en 0,97g lo que apenas filtraría el 4\% de las actividades normales, aunque mejora hasta el 27\% si aceptamos no identificar el 1\% de las caídas.

\begin{figure}[htb!]
  \centering
  \begin{subfigure}[b]{0.48\textwidth}
      \centering
      \pincludegraphics[1.1]{BourkeLowThresholdsHistogram}
      \caption{Histograma valores \textit{valle} modelo Bourke}
      \label{fig:bourke:hist:low}
  \end{subfigure}
  \hfill
  \begin{subfigure}[b]{0.48\textwidth}
      \centering
      \pincludegraphics[1.1]{BourkeUpperThresholdsHistogram}
      \caption{Histograma valores \textit{pico} modelo Bourke}
      \label{fig:bourke:hist:high}
  \end{subfigure}
  \caption{\label{fig:bourke:hist} Histogramas de valores pico y valle}
\end{figure}

Es evidente a la luz de estos datos que los resultados del clasificador sobre SisFall son mediocres. Sin embargo dada su función de mecanismo atencional y la particularidad del conjunto de datos, enfocado al estudio de caídas y con muchos eventos similares a caídas etiquetados como actividades, es un resultado esperable.


\subsubsection{Evaluación del modelo Bourke sobre SisFall}

De forma aleatoria apartamos un 20\% (901 muestras) de las muestras de SisFall como conjunto de datos de validación y usamos las 3604 restantes como conjunto de entrenamiento. Recordamos que SisFall está compuesto por 4505 muestras, 1798 de las cuales se corresponden con caídas (el 39,7\%). Las caídas están por tanto sobrerrepresentadas con respecto a su ocurrencia en la vida real.

Tomando el conjunto de datos de entrenamiento previamente filtrado y submuestreado a 25Hz, iteramos sobre los datos de la aceleración para buscar los valores de pico y valle de cada ejercicio. Para establecer las cotas, Bourke propone establecer los niveles de tal forma que el 100\% de las caídas entren dentro del espacio, consiguiendo una sensibilidad del mismo valor por definición. 

\tabla[0.4\linewidth]{tab:bourke:valores}{Valores UFT y LFT del modelo Bourke según percentil}{lllll}{
         & \multicolumn{4}{c}{Valor Percentil} \\ \cmidrule(lr){2-5}
    Cota & p. 0\%         & p. 1\% & p. 3\% & p. 10\%\\ \midrule
    UFT  & \textbf{4,16g} & 5,77g  & 6,67g  & 8,69g\\
    LFT  & \textbf{0,97g} & 0,86g  & 0,81g  & 0,68g\\
  }{3}

Este modelo tiene la desventaja de tener una especificidad muy baja. Los resultados sobre SisFall nos arrojan un valor para las cotas de 0,971g para el valle y 4,164g para el pico (en la tabla \ref{tab:bourke:valores} se presentan los resultados para otros percentiles). Con estos valores se consigue una especificidad de tan solo el 28,7\% mientras que la sensibilidad se queda en 97,2\% usando Bourke estándar (lo designaremos como modelo \textit{BourkeUL} en adelante). Repetimos el experimento iterando sobre dos variaciones del modelo de Bourke: \textit{BourkeU} que busca únicamente los casos que superan la cota de pico y \textit{BourkeL} que hace lo propio con los valles. También variamos el valor de las cotas según el porcentaje de caídas que cumplen dicha cuota en cada caso. En la figura \ref{fig:bourke_cfmatrix} se muestran las matrices de confusión y los valores de sensibilidad y especificidad para cada combinación.

\figura[0.9]{BourkeCONF_Matrix}{fig:bourke_cfmatrix}{Matrices de confusión para modelos Bourke}

Al interpretar los resultados obtenidos interesa entender principalmente la causa por la que \textit{BourkeUL} no alcanza el 100\% de sensibilidad. Lo achacamos al hecho de que para ser detectado por el modelo implementado, la caída ha de cumplir ambas cotas en una ventana deslizante de 200 muestras. El siguiente resultado, y de mayor interés para el trabajo es el buen resultado obtenido por la variante \textit{BourkeU}: Alcanza el 100\% de sensibilidad y la especificidad es la mejor de todos los modelos a igual sensibilidad (por ejemplo, si bajamos la sensibilidad hasta el 97,2\% tiene una especificidad del 43,6\%, mucho mayor que el 28,7\% de \textit{BourkeUL}). Al aunar simplicidad y resultados, optaremos por el modelo BourkeU con un valor de cota de 4,16g para el algoritmo. 



\subsection{Modelo con Redes Neuronales Recurrentes}\label{sub:imp:model:rnn}

La principal novedad de este trabajo es el uso de una red recurrente con una arquitectura codificador/decodificador para la detección de anomalías, o caídas en este caso. La detección de caídas pretende realizarse mediante un modelo de detección de anomalías. Dada una secuencia de medidas de la aceleración $X=\{x_1,x_2,\cdots,x_n\}$ de longitud $n$, definimos la secuencia $Y=\{y_1, y_2, \cdots,y_m\}$ de longitud $m$ como el resultado de aplicar el modelo RNN $RNN()$ sobre la secuencia $X$: $Y=RNN(X)$. Dado un operador $distancia(a,b)$ la detección de anomalía se realiza si la distancia entre la entrada y salida es mayor que un determinado nivel. 
\[
  isFall(X)=\left\{
    \begin{array}{lcl}
      cierto & si & distancia(X,Y) > \lambda \\
      falso & si & distancia(X,Y) \leq  \lambda
    \end{array}
    \right.
\]
\\
Donde:\\
$Y = RNN(X)$ : Salida de la red recurrente codificador/decodificador al tomar $X$ como entrada.
Al contrario que la mayoría de sistemas existentes para la detección de caídas, iFell no se basa en la clasificación de actividades, dado que dicho método requiere de extensos corpus con gran cantidad de instancias de las actividades y caídas a detectar y clasificar. Si bien existen otros algoritmos \cite{referencia} para la detección de caídas usando ya sea clasificadores de categoría única o técnicas de agrupamiento (KNN por ejemplo) que requieren de bases de datos extensas y bien etiquetadas para su entrenamiento, que para el el caso que aquí aplica, no están disponibles. Al usar una técnica de codificador/decodificador para comprimir la señal de entrada y luego reconstruirla, reducimos las necesidades del conjunto de datos para generar el modelo a secuencias no etiquetadas de secuencias de actividad que no contengan caídas para conseguir un codificador/decodificador eficiente en esta tarea, pero que no lo sea cuando la secuencia comprimida y reconstruida sea una caída, para la cual no ha sido entrenado y por tanto competa un mayor error en la tarea.

\subsubsection{Modelo Recurrente basado en codificador/decodificador}

Los modelos codificador/decodificador (véase la figura \ref{fig:modelo:encoderdecoder}), también referidos a veces como secuencia-a-secuencia se pueden entender como dos sistemas independientes: el \textit{codificador} y el \textit{decodificador}. El codificador realiza una conversión y reducción del espacio dimensional de una secuencia de entrada, intentando mantener intacta la información presente. Realiza por tanto una extracción de características de la secuencia de entrada. Evidentemente, a mayor reducción en las dimensiones del espacio de salida del codificador, mayor será la pérdida de información. Por su parte, el decodificador realiza la tarea inversa. Tomando como secuencia de entrada una de las representaciones obtenidas realiza una interpretación de las características para generar una secuencia de salida en otro espacio de resultados. Las aplicaciones son muy variadas, desde el análisis de sentimiento, compresión de series o traducción del lenguaje, a la generación de espacios de representación multimodo (por ejemplo, la descripción de una imagen o la generación de una imagen desde una descripción). 

En nuestro caso optamos por una estructura en que el espacio de entrada y salida de la red es el mismo, y por tanto realizamos una compresión y descompresión de la señal. A partir de la serie de muestras $X$de longitud $n$, obtenemos un vector de características $C=codificador(X)=\{c_1,\cdots,c_l\}$ de longitud $l<n$. Posteriormente inyectamos este vector de características como entrada a una segunda red recurrente, el decodificador, que reconstruye una aproximación de la secuencia original $Y=decodificador(C)=\hat{X}$.

% \figura[0.65]{modeloRNNencoderdecoder}{fig:modelo:encoderdecoder}{Esquema de una red RNN codificador/decodificador}

Los principales parámetros a determinar para nuestro modelo son las longitudes de las secuencias de entrada y salida, así como la de la secuencia codificada o vector de características. Este ratio entre la longitud de la señal a reconstruir y la dimensión del espacio de características es determinante para definir la capacidad del sistema de detectar anomalías. Un ratio demasiado alto, y el error de reconstrucción será tan alto que no servirá como estimador. Un ratio demasiado bajo por contra permitiría a la red enviar toda la información de la entrada y de nuevo resultar poco apta para la detección de anomalías. Los valores tomados finalmente son:
\begin{itemize}
  \item $n=m=100$ Tomamos una ventana de entrada y salida de 100 muestras o 4 segundos. La mayoría de las caídas tienen una duración de $\plusminus1s$ centrado en el pico del impacto. Los dos segundos extra permiten añadir contexto a la señal.
  \item $l=20$ La dimensión del espacio de características intermedio de 1/5 parte del tamaño del espacio de entrada y salida. Como veremos en el apartado siguiente, experimentalmente demuestra ser suficiente para la reconstrucción de la señal sin caer en la sobreadaptación.
\end{itemize}

\paragraph{Estructura de la red}
En una primera instancia optamos por el uso de capas bidireccionales\cite{Schuster1997}. Estas capas son en realizad una composición de dos capas recurrentes de igual tamaño. Una procesa la entrada en un sentido y la otra en el inverso y se pondera el resultado de ambas para obtener la salida. Estas capas han demostrado en varios estudios \cite{Zaho2017, Su2018} su mayor eficiencia procesando series temporales, llegando a mejorar en un 30\% a las capas recurrentes que procesan en un único sentido. Su uso fue finalmente descartado por el soporte inestable de TensorFlow Lite de este tipo de capas, como trataremos en el punto \ref{par:desc:modelo:rnn:limitaciones}.

Descartadas las capas bidireccionales, optamos por una arquitectura de red simétrica para codificador y decodificador, con tres capas unidireccionales de celdas GRU de 300 unidades cada una. Como hemos mencionado, la dimensión del espacio o vector de características intermedio es 20 y es el resultado de recuperar el estado interno de una última capa GRU con 20 unidades (una por característica) al codificador. Este vector se utiliza para alimentar la red del decodificador que tiene una estructura reflejo de la del codificador, con tres capas GRU de 300 unidades que reconstruyen una señal de 100 elementos.

La elección de cada componente y parámetro se realiza buscando maximizar el rendimiento de la red resultante. Así pues se escogen celdas GRU por tener capacidades similares a las celdas LSTM a la hora de tratar con series temporales \cite{Chung2014} siendo menos complejas que estas. Usamos RELU como función de activación. Dado que estamos usando el módulo de la aceleración sin normalizar a la entrada, y que el valor del mismo es siempre positivo, es una elección aceptable. Además, su simplicidad facilita tanto el entrenamiento como su optimización y ejecución posterior.  

\figura[0.4]{modeloRNNFinal}{fig:modelo:rnnFinal}{Modelo RNN de iFell}

\paragraph{Limitaciones de la plataforma TensorFlow Lite}\label{par:desc:modelo:rnn:limitaciones}
Para generar, entrenar y ejecutar el modelo de redes neuronales recurrentes optamos por la biblioteca TenorFlow/TensorFlow Lite por ser de uso generalizado y disponer de un intérprete de modelos para plataformas llevables sobre WearOS. Este cliente tiene sin embargo ciertas limitaciones que imponen restricciones a la estructura de la red entrenada. Los dos más notables es la ausencia de compatibilidad con celdas GRU\cite{tfliteGru}, que están soportadas parcialmente en la versión experimental del intérprete. Incluso así, la red recurrente generada tiene que ser una red sin estado, no podríamos implementar una red que leyese de forma contínua las muestras de la aceleración. Como nuestro modelo es episódico, esta limitación no supone un gran problema, mas allá del tener que usar un intérprete poco estable para la implementación. 
La segunda limitación por contra si tuvo un impacto: TensorFlow Lite no soporta las redes bidireccionales directamente\cite{tfliteBidir} ni pueden usarse posteriormente técnicas de \textit{pruning} para reducir su tamaño\cite{tfPruneBidir}. Sin embargo es posible realizar implementaciones personalizadas para suplir estas carencias. Con ese fin se implementan las soluciones propuestas, siendo las más destacable el uso de una capa personalizada para simular una capa bidireccional que implemente las interfaz \textit{Prunable} y consista de dos capas GRU, una normal y otra regresiva. El código de esta solución, sugerida en los foros citados, se recoge en el apéndice  \ref{app:code:ifell:prunebidir}. Sin embargo el resultado obtenido es poco satisfactorio dado lo aleatorio de los resultados obtenidos: mientras que aplicar pruning parece funcionar sin problemas, la cuantización y conversión a TensorFlow Lite arrojaba errores de forma aleatoria resultando en modelos inutilizables tras realizar el entrenamiento. Por esta razón y dado que su uso se justificaba paradójicamente en función de la optimización de rendimiento del modelo se descartó su uso y se optó por capas progresivas tradicionales pero con una arquitectura más compleja para compensar la pérdida en la capacidad de inferencia.

\subsubsection{Entrenamiento}

Antes de iniciar el entrenamiento debemos preprocesar la base de datos. Aislamos todas las secuencias de actividades que no sean caídas del conjunto SisFall. De estas, Separamos un 20\% para validación y usaremos el 80\% restante para entrenar el modelo. Entrenamos durante un máximo de 100 iteraciones sobre el conjunto reservado para ello. Usaremos una tasa de aprendizaje variable, que nos permitirá realizar las primeras iteraciones con un valor alto para acelerar el aprendizaje. Tras cada iteración, si el error RMSE entre la entrada y salida del modelo es no se ha reducido respecto a la iteración precedente, reduciremos la tasa de aprendizaje. Así mismo usamos un mecanismo de parada anticipada del aprendizaje si durante 5 iteraciones seguidas no se consigue reducir el error RMSE. Usamos Adam como optimizador y normalización por \textit{dropout} en cada capa. 

\figura{evolucionLR}{fig:LRvariable}{Evolución del valor LR exponencial}
\todo{gráfica de la evolución del entrenamiento}

\todo{comentario sobre el resultado obtenido}

\subsubsection{Optimización del modelo}

En la sección \ref{sec:req:optimizacion} hemos hablado ya de las técnicas de pruning y cuantización. Aplicaremos ambas, primero haciendo pruning del modelo entrenado, reentrenaremos el modelo con el mismo subconjunto usado para el primer entrenamiento para reajustar los pesos y corregir el error introducido al eliminar los nodos de menor interés. Con este modelo ya aligerado convertiremos el modelo para su uso en TensorFlow Lite. Realizaremos 4 conversiones: 1 sin aplicar cuantización y otras 3 usando diferentes técnicas (Conversión a enteros, dinámica y conversión a float16). Analizaremos las posibles variaciones en la precisión de la red, mejoras en el tiempo de procesado y en consumo de memoria.

\paragraph{Pruning}
Para realizar el cribado de nodos superfluos de la red usaremos el módulo \texttt{tfmot} de Keras que ofrece herramientas para realizar la evaluación y eliminación de nodos. Internamente genera un contador para cada nodo entrenable de la red donde se almacena el valor, impacto o importancia de dicho nodo en el resultado final usando algoritmos basados en el gradiente. Así pues es necesario añadir varios ciclos de entrenamiento para poder evaluar y eliminar nodos.

Realizamos 10 ciclos extra de entrenamiento sobre el total del conjunto de datos. Usamos, al igual que en el entrenamiento una tasa de aprendizaje con decaimiento no lineal. La eliminación de nodos se realizará también de forma no lineal, decreciente. Tras una primera época para analizar la importancia de cada neurona, realizamos un cribado del 30\% de ellos, y batch por batch y de forma decreciente vamos eliminando pesos durante 4 épocas hasta llegar a eliminar el 50\% de ellos. El interés de usar una tasa de aprendizaje decreciente sirve a la tarea de poder realizar grandes correcciones en la red en las primeras etapas, cuando las modificaciones en su morfología son mayores e ir reduciendo la brusquedad de los ajustes, especialmente durante las últimas 5 iteraciones durante la cuales no se elimina ningún nodo.

Como puede apreciarse en la gráfica de la función RMSE durante el entrenamiento de pruning, a pesar de la reducción de complejidad de la red, el comportamiento o error de la misma permanece prácticamente inalterado.

\todo{falta gráfica del entrenamiento/pruning}

\paragraph{Cuantización y conversión a TensorFlow Lite}
El modelo ya comprimido es no sirve para convertirlo en un modelo TensorFlow Lite. Para ello hemos de recuperar la estructura del modelo original y aplicar los pesos del modelo comprimido, poniendo a 0 los pesos de los nodos y uniones eliminados. Este cambio parece anular los beneficios de aplicar pruning, pero como veremos en las pruebas, no es así. 
Con la estructura original restituida, convertimos el modelo a TensorFlow Lite con y sin cuantización. En las tablas \ref{tab:tflite_weights_activations, tab:tflite_quantization}  introducimos los diferentes tipos de cuantización, su efecto en la representación de pesos y activaciones, velocidad, precisión y tamaño de la red. Analizamos los modelos usando la conversión a enteros (con rango dinámico y estático) y la reducción en la precisión decimal a \texttt{Float16} o 16bits.

\tablan{tab:quant:resultados}{Resultados de los modelos optimizados}{lrrrrr}{
            &             &       & \multicolumn{3}{c}{Latencia}  \\ \cmidrule(lr){4-6}
  Modelo    &  Tamaño(MB) & RMSE  &  CPU  &  GPU  & Wearable  \\ \midrule
  Base      & 11,1        & 0,532 & -     &       &           \\
  Sparse\tnote{1} & 8.1   & 0,725 & -     &       &           \\
  Int8      & 3.9         & 1.121 & -     &       &           \\
  Float16   & 7,6         & 0,726 & -     &       &           \\
  Dynamic   & 3.9         & 1.110 & -     &       &           \\
}{
\item [1] Tras restitución de la red.
}{3}

Como se observa en los resultados de la tabla \ref{ŧab:quant:resultados}, podemos obtener modelos con menos de la mitad del tamaño de la red original,con un impacto controlado y mínimo en la capacidad predictiva de la red y una mejora de un orden de magnitud en el tiempo de ejecución. Si bien esta mejora depende de la presencia o no de una unidad GPU o de cálculo en coma flotante. Si se da alguno de estos supuestos, los modelos con cuantización a coma flotante de precisión reducida mantienen las ventajas de la aceleración de estas unidades de cálculo con una degradación mínima de la precisión a costa de una menor compresión en tamaño. De no poseer de estas unidades especializadas de cálculo, los dos modelos con conversión de parámetros a enteros de 8 bits ofrecen el mejor rendimiento, aunque la degradación en la precisión del modelo puede ser excesiva según la aplicación. 

\section{Desarrollo de iFell}

\subsection{Arquitectura del Sistema}

\subsection{Arquitectura de la aplicación}

El sistema está compuesto por dos bloques funcionales tal y como se muestra en \ref{fig:clasesUml}. Tenemos un bloque o paquete que incluye los servicios encargados del registro de datos de los sensores, gestión de entradas y notificación de eventos, implementación del algoritmo de detección y comunicación con el servidor. Este primer bloque contiene a su vez una aplicación de gestión del servicio que permite realizar tareas de mantenimiento y configuración. En un segundo bloque tenemos la interfaz del usuario principal, encargada de lanzar la aplicación, alertar en caso de caída y capturar la respuesta del usuario en caso de necesitarla.

\figura{classUML}{fig:clasesUml}{Diagrama de clases de la aplicación}

El proceso que contiene la lógica principal, descrita en la figura\ref{fig:deteccionFlow}, se encuentra en la clase \textit{AccelSensorRead}. Este servicio es lanzado automáticamente al ejecutarse cualquiera de las dos interfaces de usuario provistas, y se mantiene ejecutándose de fondo. Se encarga de:

\begin{itemize}
  \item Recupera configuración previa
  \item Configurar y leer las muestras del acelerómetro
  \item Realizar el primer proceso de detección (algoritmo basado en cotas)
  \item Lanzar el proceso de análisis usando el modelo ML y recuperar el resultado
  \item Alertar y notificar del evento tanto a los clientes locales como a los servidores remotos
\end{itemize}

El proceso de detección o filtrado usando el modelo implementado basado en redes recurrentes es el único que se exporta a una clase propia: \textit{CrashDetectService}. Toma de nuevo la forma de un servicio asíncrono que es invocado únicamente cuando el modelo analítico ha detectado un positivo. De esta forma se pretende reducir drásticamente las necesidades de cómputo. Este servicio a su vez se subdivide en unos módulos que son los modelos de detección propiamente dichos. Como se aprecia en el diagrama UML de la arquitectura\ref{fig:clasesUml}, es la clase \textit{TFLiteModelDetector} la que provee la cumunicación con el modelo de TFLite previamente generado y entrenado en Colab.


\subsection{Interfaz de Usuario}
Desde el punto de vista del usuario la aplicación provée dos puntos de entrada. La interfaz principal toma la forma de una \textit{watchface}. En \textit{WearOS}, una \textit{watchface} se corresponde con un tipo específico de aplicación que realiza principalmente la función de mostrar la hora al usuario. Realiza la función de "escritorio" del sistema y es por tanto el punto inicial de toda interacción del usuario con el sistema. Este acercamiento permite solventar dos problemas:

Facilita la experiencia de usuario al no requerir de ninguna acción por parte del usuario para poner en marcha la aplicación. Al convertirse en la aplicación principal del reloj con  \textit{WearOS} garantizamos que el propio sistema operativo lanzará en el arranque y mantendrá activa la actividad en todo momento.

\begin{figure}[!ht]
  \centering
  \begin{subfigure}[b]{0.4\textwidth}
      \centering
      \includegraphics[width=\linewidth]{appActivity.png}
      \caption{Aplicación de gestión}
      \label{fig:uiActivity}
  \end{subfigure}
  \hfill
  \begin{subfigure}[b]{0.4\textwidth}
      \centering
      \includegraphics[width=\linewidth]{appWatchface.png}
      \caption{\textit{Watchface}}
      \label{fig:uiWatchface}
  \end{subfigure}
  \caption{\label{fig:ifell:UI} Interfaz de usuario de iFell}
\iffalse
\subfloat{\label{fig:uiActivity} Aplicación de gestión}{\includegraphics[width=0.4\linewidth]{appActivity.png}}
     \caption{\label{fig:uiApps} Interfaces de usuario}
     \hfill
\subfloat{\label{fig:uiWatchface} Watchface}{\includegraphics[width=0.4\linewidth]{appWatchface.png}}
\fi
\end{figure}


La aplicación toma forma de un objeto cotidiano e interactúa con el usuario utilizando un concepto conocido para este: un reloj digital. La única información que se muestra al usuario es la hora (adicionalmente el propio sistema operativo sobreimprime indicaciones de batería baja, ausencia de conexión a internet y existencia de notificaciones de otros servicios y aplicaciones). De esta forma, para el usuario, el dispositivo se convierte en un objeto conocido con un uso muy extendido que de forma adicional a su función tradicional realiza el proceso de detección de caídas.

En esta aplicación se han reducido al máximo las interacciones requeridas por parte del usuario en todo momento, hasta el punto de no requerir ninguna. La aplicación funciona en todo momento como un reloj tradicional mostrando la hora de forma analógica mediante unas manecillas. En el caso de detectarse una caída o evento similar, emitirá de forma automática una serie de avisos visuales, acústicos y hápticos (según las capacidades del dispositivo sobre el que se ejecute) que podrán ser desactivados si se detecta actividad nuevamente. Aunque también existe la posibilidad de desactivarlos tocando la pantalla.

La segunda interfaz que provee la aplicación se encarga de las tareas de administración y configuración. Permite introducir un nombre del usuario y forzar el estado de funcionamiento de los servicios de captura y detección. Si bien ninguno de estos procesos es necesario para el funcionamiento, se ofrece para facilitar la puesta en marcha y prueba del sistema.

\subsection{Comunicación con el Servidor}


% ########################################

El desafío de consolidar en un sistema portable una aplicación autónoma de detección de caídas debe hacer frente a una serie de limitaciones.

\section{Desafíos}
\todo{esto son requisitos, movel al punto anterior} El objetivo de todo modelo de detección de eventos es lograr un sistema que consiga capturar la totalidad de las realizaciones del mismo con el menos número posible de falsos positivos. En otras palabras, buscamos un sistema con una especificidad y sensibilidad de 100\%\cite{Noury2007}. \todo{Añadir referencias a papers, comentarios sobre sensibilidad y especificidad}.

\subsection{Usabilidad}\todo{de nuevo un requisito, al apartado anterior}
El primer reto de toda aplicación es conseguir una experiencia de usuario adaptada al cliente final. De nada sirve lograr implementar una plataforma que cumpla perfectamente con todos los requisitos y objetivos funcionales si el producto resultante se utiliza.

\subsubsection{Público objetivo}
Como se ha mencionado en la introducción del trabajo, los daños relacionados con las caídas son una de las principales causas de mortaldad entre las personas mayores de 65 años \todo{cita requerida}. Es propio de este grupo de población la desafección por la tecnología y la carencia o desinterés por su uso. Esta condición ha de ser tenida en cuenta para el desarrollo de cualquier producto.

Las personas de edad avanzada suelen padecer así mismo de otras condiciones que pueden limitar su grado de movilidad, atención o memoria que impidan o reduzcan la posibilidad de adaptarse o incorporar nuevas rutinas. Los problemas motores y de percepción reducen notablemente la capacidad de mostrar información así como de interactuar con el usuario cuando se necesite una acción por su parte.

Se entiende por tanto que si se ha de realizar un producto para esta población, es requisito que sea lo menos obtrusivo posible \todo{de nuevo un requisito, al punto anterior}, siendo recomendable incorporar la funcionalidad a un objeto de uso cotidiano para evitar la modificación de rutinas o la reticencia a incorporar nuevos procesos o elementos en su vida diaria. La interfaz de usuario debe ser mínima, usando un lenguaje visual que resulte familiar alejado de los estándares de las aplicaciones modernas. Así mismo, reducir o eliminar los procesos de configuración y manipulación, con un sistema que funcione al salir de la caja \todo{mala traducción de \textit{out of the box}}.

\subsubsection{Localización}

Una de las decisiones con mayor impacto sobre la funcionalidad del prototipo es la elección de la posición del dispositivo de captura ya que influencia en gran medida a la capacidad de detección de caídas \cite{Kangas2008}. Diversos estudios muestran que el mejor lugar para posicionar un sistema de medición de la aceleración para detectar caídas es la cintura, seguida de la cabeza siendo posible también usar un medidor en la muñeca\cite{Chen2005, Kangas2008, Noury2007}. Si bien estos resultados se basan en el análisis de métodos analíticos basados en cotas, se desprende de ellos la actitud o posición del cuerpo es un buen indicador para la predicción de actividades, razón por la que realizar la captura en muñecas o tobillos, las extremidades más alejadas del tronco, sufren de mayores penalizaciones para conseguir buenas estimaciones.

Al optarse por un reloj o \textit{pulsera de actividad} como plataforma para la implementación las opciones para posicionar la unidad de medida quedan reducida a una: la muñeca.

\subsection{Conectividad}

Hasta la fecha, los sistemas de detección de caídas se basan en arquitecturas cliente servidor para disociar el módulo de cómputo del de captura y conseguir que el dispositivo a llevar en si sea del menor tamaño posible. Este acercamiento que permite solventar el problema que derivaría de tener que llevar un obtrusivo sistema sobre si añade el problema de la necesidad de un enlace o comunicación con el módulo de cálculo. Ganamos en usabilidad pero perdemos en portabilidad.\todo{Cita y referencias a sistemas comerciales}

El principal problema de la conectividad radica en el hecho de que a pérdida de comunicación entre ambos módulos deriva en un sistema con ninguna capacidad. El módulo de cálculo, privado de datos no es capaz de realizar ninguna detección. Por su parte el aparato de captura, por si mismo, no tiene capacidad alguna para realizar nada más que la lectura.

El sistema implementado opta por una arquitectura cliente-pesado y servidor ligero. El cliente tiene suficiente capacidad para realizar captura, cómputo y capacidad de alerta como para funcionar de forma aislada. El servidor realiza funciones de expansión de la capacidad de alerta de la plataforma, así como de distribución de actualizaciones o mejoras en modelos y parámetros, pero de ninguna manera resulta imprescindible para el funcionamiento del dispositivo portable. Este es el principio básico y director de este trabajo: implementar una plataforma autónoma de detección de caídas capaz de ejecutarse en un dispositibo vestible.


\section{Plataforma}
\todo{justificar cada una de estas decisiones}
Para el servidor optamos por una arquitectura de microservicios usando la plataforma de AWS Lambda con almacenamiento en S3

Para el dispositivo móvil usamos un reloj inteligente Fossil Sport con sistema operativo WearOS y por tanto compatible con el ecosistema Android.

El la generación, entrenamiento, análisis y evaluación de modelos se realiza usando Keras/Tensorflow corriendo en la plataforma Google Colab.

\section{Arquitectura}\label{desc_archi}
\warn{según la profe punto muy interesante, a pulir }

\subsection{Arquitectura del sistema}

\warn{hablando de AWS: (ver siguiente páraffo)}
Resumiendo la estructura del sistema, \textit{AWS S3} (\url{https://aws.amazon.com/es/s3/?c=ser&sec=srv}) es el sistema de almacenamiento de datos en la red. Un disco duro en la nube, escalable en capacidad y fácilmente accesible para poder recuperar los datos. \textit{AWS Lambda} (\url{https://aws.amazon.com/es/lambda/?c=ser&sec=srv}) permite implementar funciones de código y definir una serie de eventos para lanzar su ejecución, al formar parte del ecosistema AWS es fácil conectar estas funciones con el sistema S3 y API Gateway. Finalmente \textit{AWS API Gateway} permite definir unos puntos de entrada, o URLs que conformarán la API del servicio. Cuando una de estas direcciones URL es invocada, inmediatamente se ejecuta la 



\todo{show me the data!!! cómo llegamos a 3G?}
El valor de $A_{umbral}$ se obtiene experimentalmente gracias al análisis de los datos obtenidos. Se fija en $A_{umbral} = 3G$ que resulta un buen equilibrio ya que conseguimos una cantidad reducida de eventos manteniendo un nivel bajo\todo{citar trabajos y medidas. Bourke usa 3,5G, otros usan valores más altos, la mayoría de caídas tienen niveles instantáneos de SVTot superiores a 6G}.\todo{incluir trabajo estadístico de los datos de aceleración del dataset} \todo{incluir gráfica temporal explicativa del algoritmo}

Si se sobrepasa el umbral en $t=0$. El sistema envía instantáneamente al siguiente modelo las muestras entre $t=-225$ y $t=-25$ (4 segudnos previos al evento). En segundo plano sigue capturando muestras hasta $t=25$. Este segundo bloque de 50 muestras (un segundo centrado en el evento) se envía también a la siguiente etapa. La decisión de separar el envío de datos a la siguiente etapa está motivada por la reducción de la latencia del sistema y obtener una clasificación o resultado con la menor dilación posible.


\subsubsection{Modelo RNN}


El modelo predictivo, se basa en la capacidad de las redes RNN y particularmente las basadas en celdas LSTM y GRU para generar modelos predictivos de calidad para series temporales de señales estacionarias \cite{Qin2019}.




\figura{trainingModeloRNN}{fig:rnnTrainAlt}{Evolución de RMSE durante el entrenamiento}

\figura{prediction.png}{fig:prediccionAceleracion}{Predicción de la aceleración}

\paragraph*{Detección}

Este segundo bloque recibe por un lado la señal $SVTot(-25:25)$\todo{no es un formato aceptable, usar t-25, t-24, .... t+25 o algo que sea aceptable} del acelerómetro y por el segundo la predicción $\hat{SVTot}(-25:25)$ \todo{unificat nomenclatura!!!}. Para clasificar el evento como una caída, usaremos un umbral sobre el error de predicción del modelo recurrente. La métrica de error empleada es la \textit{Raíz del Error Cuadrático Medio} definida como: \[
RECM=\sqrt{1/T\sum_{t_1}^{t_2}|y-\hat{y}|^2 }
\]Donde $t_1 = -25$, $t_2 = 25$ y $T=t_2-t_1=50$.

Comparando el resultado del error obtenido con un nuevo umbral al que denominaremos umbral de detección $U_{d}$ obtendremos finalmente la clasificación del evento en \textit{Caída} o \textit{no-Caida}. Dicho umbral se obtiene de nuevo de forma experimental.

En este caso el valor utilizado se calcula mediante el análisis del RECM usado durante la validación del modelo durante el entrenamiento. \todo{mostrar valores}

 
%Añade errores pero sobretodo son referencias

\chapter{Evaluación}\label{chap:eval}
% !TeX root = ../tfm.tex
%! TEX root = ../tfm.tex
\documentclass[../tfm.tex]{subfiles}
\begin{document}

\info{al menos una mínima evaluación de usabilidad de la herramienta y su aplicabilidad para resolver el problema resuelto.}

\section{Evaluación del modelo}\label{eval_modelo}

Un sistema de detección de caídas es en el fondo un clasificador con una única clase. Todos los resultados pueden por tanto agruparse en los que pertenecen o no a dicho conjunto. Con el fin de evaluar y comparar los resultados obtenidos usaremos dos métricas estadísticas:
\begin{itemize}
  \item Sensitividad (Capacidad de identificar las caídas), también conocida como \textit{recall}.
  \[
    Sensitividad = \frac{TP}{TP+FN}
  \]
  \item Especificidad o Selectividad (Capacidad de discernir únicamente las caídas).
  \[
    Especificidad = \frac{TN}{TN+FP}
  \]
\end{itemize}

Estas métricas son usadas habitualmente en otros trabajos similares\cite{Noury2007,Chen2005, Bourke2006}
\info{añadir el resto de citas}.



\subsection{Modelos basados en Machine Learning}
\info{De Anita 2020 sabemos que
Putra2017: An Event-triggered machine learning approach for accelerometer based fall detection (sistema híbrido: Eventos + ML)

Hussain2019 "Activity Aware falldetection and recognition based on wearable sensors" IEEE sensors 19 vol 12, solo suscripción :(

}

En la tabla \ref{tab:MLResults} Destacan los buenos resultados obtenidos por los modelos que usan técnicas de aprendizaje automático. Si bien en lo referente a la latencia, \cite{Liu2020} obtiene tiempos que superan el segundo en la mayoría de modelos, llegando incluso a los 8,87s obtenidos con un clasificador de Bayes. Tanto en los trabajos de \cite{Liu2020} como \cite{Musci2020} y \cite{Torti2018} se subraya el hecho de la no uniformidad de los resultados. Los mejores resultados se obtienen con población joven mientras que con población adulta las métricas pueden perder hasta 5 puntos, en parte debido a la falta de datos de entrenamiento.

\tablas{tab:MLResults}{Resultados de sistemas basados en ML}{l|c|c|c|ccc}{
   & Yo & Musci2020 & Torti2018  & Liu2020  & Liu2020 & Liu2020 \\
   & GRU & RNN(LSTM) & RNN(LSTM)  & FD-DNN   & LSTM    & CNN     \\ \midrule
Sensitividad (\%) &  &   & 98,73  & 94,09  & 81,47   & 87,50   \\
Especificidad (\%)&   &   & 97,93  & 99,94  & 99,57   & 99,88   \\
Accuracy (\%)     & 91,1 &   & 98,33  & 99,17  & 96,88   & 98,13   \\
Tiempo(s)         &     &     &     &     &     &     \\
}

\figura{CompositeFallNormalRMSHistogram_t-25}{fig:GRU_predictionRMS_Histogram}{Histograma de los errores de predicción del modelo GRU\todo{Se han mezclado muestras de caidas en el entrenamiento y el modelo. Repitiendo el entrenamiento.}}

En la tabla\ref{tab:analiticResults} se muestran los resultados de los modelos basados en métodos analíticos. Su importancia para este estudio proviene del hecho de que son los métodos más extendidos en los sistemas disponibles hoy en día y establecen por tanto el nivel a superar. Se muestra también los resultados obtenidos para el clasificador bourke utilizado a modo comparativo y de validación del resultado obtenido.
\tablas{tab:analiticResults}{Resultados de sistemas basados en métodos analíticos}{l|c|c|c|c}{
              & Yo        & Yo      & SisFALL & SisFALL \\
              & Bourke    & Hibrido & Cotas(SumVect)  & Cotas(SV) 100\%Sens \\ \midrule
Sensitividad (\%) & 99,4  &    91,13   & 94,28 & 100  \\
Especificidad (\%) & 29,7 &   42,88    & 96,13 & 32,9 \\
Accuracy (\%) & 64,55 &     67  & 95,21 & 66,43 \\
Tiempo (s)    & 0     & 2,5     & 0       & \\
}
\warn{Falta extraer Lim2014 con sisfall y comparar}

\subsection{Mixto Bourke + GRU}


\todo{Validar la mejora del acercamiento en cuanto a especificidad respecto a Bourke simple. Comparar con los resultados de la tabla anterior}


\end{document}


\chapter{Conclusiones y Trabajo Futuro}\label{chap:conclusiones}
% !TeX root = ../tfm.tex
%! TEX root = ../tfm.tex

\section{Conclusiones}

A lo largo de este ejercicio hemos introducido, definido y realizado la implementación de \textit{iFell}, una plataforma de detección de caídas que puede ejecutarse de forma completamente autónoma
sobre un dispositivo llevable. Hemos demostrado la viabilidad de un algoritmo híbrido basado que ejecute un modelo basado en redes de neuronas recurrentes de forma episódica para reducir los requisitos tanto a nivel de cómputo como de consumo energético logrando un sistema que puede permanecer activo durante días con una latencia inferior a un segundo. Hemos introducido un nuevo modelo de detección de anomalías en series temporales usando una arquitectura codificador/decodificador con redes recurrentes y el RMSE que mejora en hasta un 50\% especificidad de un modelo Bourke, manteniendo el 100\% de sensibilidad. Hemos demostrado que la extracción de características de la señal $|\vec{A}|$ muestreada a únicamente 50Hz tiene suficiente información para segregar cídas del resto de actividades, incluso llevando el sensor en la muñeca. En el camino hemos analizado variaciones del modelo codificador/decodificador para detección de anomalías usando tanto la predicción de múltiples pasos futuros como la capacidad de reconstrucción de la señal y hemos evaluado el impacto en rendimiento y recursos de varias técnicas de optimización de modelos basados en redes de neuronas.

También hemos recolectado una muestra de capturas de actividad cotidiana, para lo cual hemos implementado una aplicación específica: \textit{AccelCapture}. Analizando estas muestras hemos observado la estacioanriedad de la señal aceleración, su inexistente autocorrelación en la mayoría de actividades y la distribución de picos y valles de la aceleración para actividades normales y caídas.




\section{Lineas de trabajo futuro}

Durante el trabajo se han ido comentando varias opciones que por una razón u otra no fue viable explorar en su momento, como por ejemplo el uso de capas bidireccionales o técnicas de optimización de la discretización de pesos avanzadas. Sin embargo estas lineas de mejora entregarán un beneficio marginal.

De especial interés de cara a futuras iteraciones del modelo codificador/decodificador aquí presentado, es el posible uso de técnicas de clústering como K-means sobre el espacio de salida del codificador para la detección de anomalías reemplazando el codificador y comparador RMSE. Si se pretende proseguir por el uso de técnicas de aprendizaje no supervisado, el uso de redes generativas antagónicas (\textit{GAN}) para entrenar un discriminador que sepa reconocer actividad normal y cotidiana de caídas puede ser una vía alternativa ya sea al uso de un decodificador o directamente reemplazar todo el sistema. Finalmente, se puede estudiar cambiar el codificador o extractor de característica y reemplazarlo por una red convolucional que analice la señal de los sensores de entrada.

De cara a mejorar la calidad del algoritmo híbrido, en especial del modelo analítico, puede ser interesante el uso de información de otros sensores que habitualmente se encuentran en pulseras de actividad y otros dispositivos llevables como puede ser un giroscopio o sensor del campo magnético o pulsómetro. Con la llegada de nuevas generaciones de microprocesadores con unidades de cómputo especializadas en modelos neuronales se facilita la posibilidad de aumentar la complejidad de estos y permitir el uso de entradas compuestas de varias fuentes. Este incremento en la diversidad de fuentes de datos debería ayudar a paliar el problema de la baja calidad de las medidas de la aceleración realizadas en la muñeca como estimador de la actividad.





% \bibliographystyle{unsrt}
% \bibliographystyle{apalike}
\newpage %necesario? parece que con apacite si
\bibliographystyle{apacite} % usa apacite y parece no funcionar, pero es el bueno
\bibliography{tfm}

\lstlistoflistings

\appendix

% \chapter{Captura de datos y construcción del dataset}\label{app:dataset}
% \subfile{seccion/app-accelcapture}

\chapter{Entrenamiento y Evaluación de modelos}\label{app:modelos}
% !TeX root = ../tfm.tex
%! TEX root = ../tfm.tex



\section{Entrenamiento de Modelos}
\figura{evolucionLR}{fig:LRvariable}{Evolución del valor LR exponencial}

\section{Modelos \ifell/}


\section{Matrices de confusión}

\subsection{Modelos Bourke}
Entrenados y validados usando \sisfall/
\figura[0.7]{BourkeCONF_Matrix}{fig:app:confmatrix:bourke}{Matrices de confusión para BourkeU, BourkeL y BourkeUL}

\subsection{Modelos RNN e IFELL}
La denominación sigue el esquema \textit{<RNN|IFELL>-<tipo>(<unidades>)[<atributos>]}. Donde:
\begin{itemize}
  \item \textit{RNN} Denomina a los modelos entrenados y validados con \sisfall/
  \item \textit{IFELL} Designa a los modelos entrenados con \accelcapture/ y validados con \sisfall/
  \item \textit{<tipo>} \textbf{P} para modelos de predicción y \textbf{R} para modelos de reconstrucción
  \item \textit{<celdas>} Es el número de unidades que componen cada celda GRU
  \item \textit{<atributos>} El número de atributos o dimensiones del espacio intermedio
\end{itemize}

\paragraph{Modelos RNN de 175 unidades}
\begin{figure}[H]
  \centering
  \begin{subfigure}[t]{0.48\textwidth}
      \centering
      \pincludegraphics[1.0]{RNNP(175)30CONF.pgf}
      \caption{\footnotesize \label{fig:app:confmatrix:P17530}Modelo RNN-P(175)[30]}
  \end{subfigure}
  \hfill
  \begin{subfigure}[t]{0.48\textwidth}
      \centering
      \pincludegraphics[1.0]{RNNR(175)30CONF.pgf}
      \caption{\footnotesize \label{fig:app:confmatrix:R17530}Modelo RNN-R(175)[30]}
  \end{subfigure}
  \begin{subfigure}[t]{0.48\textwidth}
      \centering
      \pincludegraphics{RNNP(175)50CONF.pgf}
      \caption{\footnotesize \label{fig:app:confmatrix:P17550}Modelo RNN-P(175)[50]}
  \end{subfigure}
  \hfill
  \begin{subfigure}[t]{0.48\textwidth}
      \centering
      \pincludegraphics[1.0]{RNNR(175)50CONF.pgf}
      \caption{\footnotesize \label{fig:app:confmatrix:R17550}Modelo RNN-R(175)[50]}
  \end{subfigure}
  \caption{\label{fig:app:confmatrix:175} Matrices confusión para modelos RNN de 175 unidades por celda}
\end{figure}

\paragraph{Modelos RNN de 350 unidades}

\begin{figure}[H]
  \centering
  \begin{subfigure}[t]{0.48\textwidth}
      \centering
      \pincludegraphics[1.0]{RNNP(350)30CONF.pgf}
      \caption{\footnotesize \label{fig:app:confmatrix:P35030}Modelo RNN-P(350)[30]}
  \end{subfigure}
  \hfill
  \begin{subfigure}[t]{0.48\textwidth}
      \centering
      \pincludegraphics[1.0]{RNNR(350)30CONF}
      \caption{\footnotesize \label{fig:app:confmatrix:R35030}Modelo RNN-R(350)[30]}
  \end{subfigure}
  \begin{subfigure}[t]{0.48\textwidth}
      \centering
      \pincludegraphics[1.0]{RNNP(400)50CONF}
      \caption{\footnotesize \label{fig:app:confmatrix:P35050}Modelo RNN-P(350)[50]}
  \end{subfigure}
  \hfill
  \begin{subfigure}[t]{0.48\textwidth}
      \centering
      \pincludegraphics[1.0]{RNNR(350)50CONF}
      \caption{\footnotesize \label{fig:app:confmatrix:R35050}Modelo RNN-R(350)[50]}
  \end{subfigure}
  \caption{\label{fig:app:confmatrix:350} Matrices confusión para modelos RNN de 350 unidades por celda}
\end{figure}

\paragraph{Modelos IFELL}
\begin{figure}[H]
  \centering
  \begin{subfigure}[t]{0.48\textwidth}
      \centering
      \pincludegraphics[1.0]{IFELLP(175)50CONF}
      \caption{\footnotesize \label{fig:app:confmatrix:ifell:P17550}Modelo IFELL-P(175)[50]}
  \end{subfigure}
  \hfill
  \begin{subfigure}[t]{0.48\textwidth}
      \centering
      \pincludegraphics[1.0]{IFELLR(175)50CONF}
      \caption{\footnotesize \label{fig:app:confmatrix:ifell:R17550}Modelo IFELL-R(175)[50]}
  \end{subfigure}
  \caption{\label{fig:app:confmatrix:ifell:175} Matrices confusión para modelos IFELL de 175 unidades por celda}
\end{figure}


\chapter{Listados de código}\label{app:codigo}
% !TeX root = ../tfm.tex
%! TEX root = ../tfm.tex
\section{\accelcapture/}
\subsection{Servidor/Api Lambda}\label{app:code:accelcapturelambda}
\subsubsection{POST guardarSesions}
% \begin{spacing}{0.5}
\todo{poner el nombre que toca a la función}
\lstinputlisting[style=ES6,caption={Listado de código de POST guardarSessions}]{code/lambdaAccelCapture.js}
% \end{spacing}

\section{\ifell/}
\subsection{Modelo RNN}
\subsubsection{Prunable Bidir}\label{app:code:ifell:prunebidir}
\lstinputlisting[style=ES6,language=Python,caption={Listado de código de implementación de capas PruneBidir}]{code/pruneBidir.py}


\chapter{Plataforma y Dispositivo}\label{app:plataforma}
\subfile{seccion/app-fossil}

\chapter{Artículo}
\includepdf[pages=-]{articulo.pdf} 

% ELIMINAR EN VERSIÓN FINAL
% \listoftodos
\end{document}
